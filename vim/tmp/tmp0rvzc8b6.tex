% Input your problem and solution below.
% Three dashes on a newline indicate the breaking points.

---

Let $a_2$ be any positive integer. We define the infinite sequence $a_2$, $a_3$, $\ldots$ recursively as follows.
\begin{quote}
    If $a_n=0$, then $a_{n+1}=0$. Otherwise, we write $a_n$ in base $n$, then write all exponents of $n$ in base $n$, and so on until all numbers in the expression are at most $n$. Then we replace all instances of $n$ with $n+1$ (including the exponents), subtract $1$, and let $a_{n+1}$ be the result.
\end{quote}
For example, if $a_2=11$ we have
\begin{align*}
    a_2&=2^3+2+1=2^{2+1}+2+1\\
    a_3&=3^{3+1}+3+1-1=3^{3+1}+3\\
    a_4&=4^{4+1}+4-1=4^{4+1}+3\\
    a_5&=5^{5+1}+3-1=5^{5+1}+2
\end{align*}
and so on. Prove that $a_N=0$ for some integer $N>2$.

---

Say the \emph{$n$-ordinal} representation of a number $M$ is the result when all instances of $n$ in the hereditary base-$n$ representation of $M$ are replaced with the first infinite ordinal number $\omega$.

Define a sequence $(b_n)_{n\ge2}$ such that $b_n$ is the $n$-ordinal representation of $a_n$. So, in our example,
\begin{align*}
    b_2&=\omega^{\omega+1}+\omega+1\\
    b_3&=\omega^{\omega+1}+\omega\\
    b_4&=\omega^{\omega+1}+3\\
    b_5&=\omega^{\omega+1}+2\\
    b_6&=\omega^{\omega+1}+1\\
    b_7&=\omega^{\omega+1}\\
    b_8&=7\cdot\omega^\omega+7\cdot\omega^7+\cdots+7.
\end{align*}
Assume for contradiction $(a_n)_{n\ge2}$ is never zero. Note that the $n$-ordinal representation of $a_n$ and the $(n+1)$-ordinal representation of $a_{n+1}+1$ are equal, so $(b_n)_{n\ge2}$ is forever strictly increasing.

But this is absurd, since ordinals are well-founded; i.e.\ each subset of ordinals has a least element. Assuming the axiom of dependent choice, there is no countably infinite strictly decreasing sequence of ordinals.

