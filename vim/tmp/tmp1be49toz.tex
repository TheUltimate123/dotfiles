% Input your problem and solution below.
% Three dashes on a newline indicate the breaking points.

---

Given any positive integer $c$, denote $p(c)$ as the largest prime factor of $c$. A sequence $\{a_n\}$ of positive integers satisfies $a_1>1$ and $a_{n+1}=a_n+p(a_n)$ for all $n\ge1$. Prove that there must exist at least one perfect square in the sequence $\{a_n\}$.

---

We will show there is an element $m$ of the sequence such that $p(m)>\sqrt m$; call these numbers \emph{radioactive}. From there, $p(a_n)$ will always be constant until $a_n$ is the perfect square $p(m)^2$.

Assume for contradiction no element of the sequence is radioactive, and define $p_n=p(a_n)$ and $b_n=a_n/p_n$. Since $(p_n)$ cannot be eventually constant, and it is non-decreasing, it must be unbounded. Since $b_n>p_n$ by assumptuion, $(b_n)$ is also unbounded.

However \[b_{n+1}=\frac{a_{n+1}}{p_{n+1}}\le\frac{a_{n+1}}{p_n}=\frac{a_n+p_n}{p_n}=b_n+1,\]
but $(b_n)$ is unbounded, so it must eventually be prime. Then some element of the sequence is of the form $pq$, where $q$ is in the sequence $(b_n)$. These numbers are not radioactive, contradiction.

