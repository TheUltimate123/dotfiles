% Input your problem and solution below.
% Three dashes on a newline indicate the breaking points.

---

For every positive integer $m=2^tk$, where $k\ge0$ and $t$ is odd, we let $f(m)=t^{1-k}$. Prove that for any positive integers $a\le m$ with $a$ odd, the number $f(1)f(2)\cdots f(m)$ is an integer divisible by $a$.

---

It is sufficient to check that $\nu_p(f(1))+\cdots+\nu_p(f(m))\ge\nu_p(f(a))$ for odd primes $p$. Notice that \[\nu_p(2^kt)=(1-k)\nu_p(t)\implies\nu_p(f(m))=\big(1-\nu_2(m)\big)\nu_p(m).\]
The key computation is that \[\sum_{i=1}^m\nu_p(i)=\sum_{i=1}^m\sum_{j>0}\left(p^j\mid k\right)=\sum_{j>0}\sum_{i=1}^m\left(p^j\mid k\right)=\sum_{j>0}\left\lfloor\frac m{p^j}\right\rfloor=\nu_p(m!),\]
whereas by a similar computation,
\begin{align*}
\sum_{i=1}^m\nu_p(i)\nu_2(i)&=\sum_{i>0}\sum_{j>0}\left\lfloor\frac n{p^i2^j}\right\rfloor\le\sum_{i>0}\sum_{j=0}^{\left\lfloor\log_2(m)\right\rfloor}\frac1{2^j}\left\lfloor \frac n{p^i}\right\rfloor\\
&\le\sum_{i=0}^{\left\lfloor\log_p(m)\right\rfloor}\left(\left\lfloor\frac n{p^i}\right\rfloor-1\right)=\nu_p(n!)-\left\lfloor\log_p(m)\right\rfloor.
\end{align*}
Thus \[\sum_{i=1}^m\nu_p(f(i))=\sum_{i=1}^m\nu_p(i)-\sum_{i=1}^m\nu_p(i)\nu_2(i)\ge\left\lfloor\log_p(m)\right\rfloor,\]
which is sufficient.
