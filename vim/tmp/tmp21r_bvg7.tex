% Input your problem and solution below.
% Three dashes on a newline indicate the breaking points.

---

Let $ABC$ be a triangle with circumcenter $O$. The points $P$ and $Q$ are interior points of the sides $CA$ and $AB$ respectively. Let $K,L,M$ be the midpoints of $\seg{BP}$, $\seg{CQ}$, $\seg{PQ}$, respectively, and let $\Gamma$ be the circumcircle of $\triangle KLM$. Suppose that $\seg{PQ}$ is tangent to $\Gamma$. Prove that $OP=OQ$.

---

\begin{center}
    \begin{asy}
        size(12cm);
        defaultpen(fontsize(10pt));
        pen pri=red+linewidth(0.5);
        pen sec=orange+linewidth(0.5);
        pen tri=fuchsia+linewidth(0.5);
        pen fil=red+opacity(0.05);
        pen sfil=orange+opacity(0.05);
        pair O, A, B, C, P, Q, K, L, M;
        O=(0, 0);
        A=dir(110);
        B=dir(220);
        C=dir(320);
        Q=extension(O, dir(190), A, B);
        P=intersectionpoints(circle(O, length(Q-O)), A -- C)[1];
        K=(B+P)/2;
        L=(C+Q)/2;
        M=(P+Q)/2;

        draw(A -- B -- C -- A, pri);
        draw(B -- P -- O -- Q -- C, pri);
        draw(K -- L -- M -- K, sec);
        draw(P -- Q, pri);
        filldraw(circumcircle(A, B, C), fil, pri);
        filldraw(circumcircle(K, L, M), sfil, sec);

        dot("$O$", O, N);
        dot("$A$", A, A);
        dot("$B$", B, B);
        dot("$C$", C, C);
        dot("$P$", P, E);
        dot("$Q$", Q, W);
        dot("$K$", K, S);
        dot("$L$", L, NE);
        dot("$M$", M, N);
    \end{asy}
\end{center}
Let $R$ denote the circumradius of $\triangle ABC$. Note that $\seg{MK}$ is the $P$-midsegment of $\triangle PQB$. However, as $(KLM)$ is tangent to $\seg{PQ}$ at $M$, \[\da AQP=\da BQP=\da KMP=-\da MLK.\]
Similarly, $\da APQ=-\da MKL$, so $\triangle APQ\sim\triangle MKL$. It follows that \[\frac{PA}{QA}=\frac{MK}{ML}=\frac{QB}{PC}\implies PA\cdot PC=QA\cdot QB.\]
However, \[OP^2-R^2=\pow(P,(ABC))=\pow(Q,(ABC))=OQ^2-R^2,\]
whence $OP=OQ$, as desired.

