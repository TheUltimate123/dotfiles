% Input your problem and solution below.
% Three dashes on a newline indicate the breaking points.

---

Determine all functions $f:\mathbb Z\to\mathbb Z$ such that for all integers $a$ and $b$, \[f(2a)+2f(b)=f(f(a+b)).\]

---

The answer is $f\equiv0$ and $f(x)=2x+q$ for all $q$. Let $P(a,b)$ denote the assertion. Notice that
\begin{itemize}
    \item $P(1,x)$ yields $f(2)+2f(x)=f(f(x+1))$ for all $x$.
        \vspace{-0.5em}
    \item $P(0,x+1)$ yields $f(0)+2f(x+1)=f(f(x+1))$ for all $x$.
\end{itemize}
Combining these, $f(2)+2f(x)=f(0)+2f(x+1)$. Hence, $f(0)\equiv f(2)\pmod2$ and for all $x$, \[f(x+1)-f(x)=\frac{f(2)-f(0)}2.\]
The right hand side is constant, so $f$ is linear. Now, set $f(x)=px+q$ and $c=a+b$. Substituting,
\begin{align*}
    2pa+q+2pb+2q&=p(p(a+b)+q)+q\\
    \iff2p(a+b)+3q&=p^2(a+b)+pq+q\\
    \iff2q&=p(c(p-2)+q).
\end{align*}
Let $c=0$; then, either $q=0$ or $p=2$.
\begin{itemize}
    \item If $q=0$, then $0=cp(p-2)$ for all $c$, implying $p\in\{0,2\}$.
        \vspace{-0.5em}
    \item If $p=2$, then $2q=2q$, so all such functions work.
\end{itemize}
It is easy to check that $f\equiv0$ and $f(x)=2x+q$ work, so we are done.

