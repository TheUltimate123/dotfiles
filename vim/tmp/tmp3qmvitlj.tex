% Input your problem and solution below.
% Three dashes on a newline indicate the breaking points.

---

Circles $\Gamma_1$, $\Gamma_2$, and $\Gamma_3$ are pairwise externally tangent and have diameters of length $70$, $99$, and $55$, respectively. Suppose that $\Gamma_2$ and $\Gamma_3$ touch at $X$, $\Gamma_3$ and $\Gamma_1$ touch at $Y$, and $\Gamma_1$ and $\Gamma_2$ touch at $Z$. A point $A$ is chosen on the minor arc $YZ$ of $\Gamma_1$. Ray $AZ$ intersects $\Gamma_2$ again at $B$, ray $BX$ intersects $\Gamma_3$ again at $C$, ray $CY$ intersects $\Gamma_1$ again at $D$, ray $DZ$ intersects $\Gamma_2$ again at $E$, ray $EX$ intersects $\Gamma_3$ again at $F$, and ray $FY$ intersects $\Gamma_1$ again at $G$. Find the maximum possible value of $AB+BC+CD+DE+EF+FG$.

---

Let the centers of $\Gamma_1,\Gamma_2,\Gamma_3$ be $P,Q,R$, respectively. Notice that if central angles are directed, by homothety,
\begin{align*}
    \widehat{DZ}&=\widehat{YZ}+\widehat{DY}=\widehat{YZ}+\widehat{CY}=\widehat{YZ}+\widehat{XY}+\widehat{CX}=\widehat{YZ}+\widehat{XY}+\widehat{BX}\\
    &=\widehat{YZ}+\widehat{XY}+\widehat{ZX}+\widehat{BZ}=\widehat{YZ}+\widehat{XY}+\widehat{ZX}+\widehat{AZ}=\widehat{AZ}.
\end{align*}
Hence, $\overline{AD}$ is a diameter of $\Gamma_1$, and as a result, $G=A$.
\begin{center}
    \begin{asy}
        size(8cm);
        defaultpen(fontsize(10pt));

        pen pri=royalblue;
        pen sec=blue;
        pen tri=Cyan;
        pen fil=royalblue+opacity(0.05);
        pen tfil=Cyan+opacity(0.05);

        pair P, Q, R, I, X, Y, Z, A, B, C, D, EE, F;
        P=(0, 120);
        Q=(-119, 0);
        R=(35, 0); // c=169, a=154, b=125 -> s=224
        I=incenter(P, Q, R);
        X=foot(I, Q, R);
        Y=foot(I, R, P);
        Z=foot(I, P, Q);
        A=P+70*dir(270-aCos(119/169)-aSin(119/169)-4aTan(3/7));
        B=Z-99/70*(A-Z);
        C=X-55/99*(B-X);
        D=2P-A;
        EE=2Q-B;
        F=2R-C;

        filldraw(P -- Q -- R -- cycle, tfil, tri);
        draw(A -- B -- C -- D -- EE -- F -- A, sec);
        filldraw(circle(P, 70), fil, pri);
        filldraw(circle(Q, 99), fil, pri);
        filldraw(circle(R, 55), fil, pri);

        dot("$P$", P, N);
        dot("$Q$", Q, SW);
        dot("$R$", R, SE);
        dot("$X$", X, E);
        dot("$Y$", Y, dir(60));
        dot("$Z$", Z, dir(30));
        dot("$D$", A, N);
        dot("$E$", B, S);
        dot("$F$", C, dir(33));
        dot("$A$", D, SW);
        dot("$B$", EE, N);
        dot("$C$", F, S);
    \end{asy}
\end{center}
Let $a=70,b=99,c=55$ be the diameters of $\Gamma_1,\Gamma_2,\Gamma_3$, respectively, and also let $2\theta=\widehat{BX}=\widehat{CX}$. It follows that if $2\beta=\angle Q$ and $2\gamma=\angle R$, then $\widehat{AZ}=\widehat{BZ}=2(\theta-\beta)$ and $\widehat{CY}=\widehat{DY}=2(\theta+\gamma)$. Hence, by the Extended Law of Sines, \[AB=AZ+BZ=(a+b)\sin(\theta-\beta).\]
Similarly, $BC=(b+c)\sin\theta$ and $CD=(c+a)\sin(\theta+\gamma)$. boxremark that \[DE=(a+b)\cos(\theta-\beta).\]
Similarly, $EF=(b+c)\cos\theta$ and $FA=(c+a)\cos(\theta+\gamma)$.

We will now determine $\beta$ and $\gamma$. Scale $\triangle PQR$ up by a factor of $2$, so that angles are preserved. It is easy to determine that $PQ=169,QR=154,RP=125$, and if $S$ denotes the foot from $P$ to $\overline{QR}$, $PS=120,QS=119,RS=35$. Then, $\cos Q=\tfrac{119}{169}$ and $\cos R=\tfrac7{25}$. By the Half Angle Formulae, \[\sin\beta=\frac5{13},\;\cos\beta=\frac{12}{13},\;\sin\gamma=\frac35,\;\cos\gamma=\frac45,\]
and by the Cauchy-Schwarz Inequality,
\begin{align*}
    &AB+BC+CD+DE+EF+FA\\
    &=169\sin(\theta-\beta)+154\sin\theta+125\sin(\theta+\gamma)\\
    &\qquad+169\cos(\theta-\beta)+154\cos\theta+125\cos(\theta+\gamma)\\
    &=(156\sin\theta-65\cos\theta)+154\sin\theta+(100\sin\theta+75\cos\theta)\\
    &\qquad+(156\cos\theta+65\sin\theta)+154\cos\theta+(100\cos\theta-75\sin\theta)\\
    &=420\sin\theta+400\cos\theta=20(21\sin\theta+20\cos\theta)\\
    &\le 20\sqrt{(21^2+20^2)(\sin^2\theta+\cos^2\theta)}=580.
\end{align*}
Scaling back down, $AB+BC+CD+DE+EF+FA\le 580$, which is easily achievable by taking the equality case of the Cauchy-Schwarz Inequality (shown in the diagram).
\begin{alternative}
    In the above solution, we use the Cauchy-Schwarz Inequality to maximize the expression $21\sin\theta+20\cos\theta$. Alternatively, note that if $\zeta=\tan^{-1}(\tfrac{20}{21})$, \[21\sin\theta+20\cos\theta=29(\cos\zeta\sin\theta+\sin\zeta\cos\theta)=29\sin(\zeta+\theta)\le 29,\]
    with equality easily achievable.
\end{alternative}
\begin{boxremark}
    It appears that $\overline{AP}\perp\overline{QR}$, but this is not true. In fact, if we solve the equality case for $\theta$, we find that $\theta=2\tan^{-1}(\tfrac37)$. Hence, if $S$ is the foot from $P$ to $\overline{QR}$, then (oriented clockwise) \[\angle APS=4\tan^{-1}\left(\frac37\right)-90^\circ\approx 2.79436205459274^\circ.\]
\end{boxremark}

---

580
