% Input your problem and solution below.
% Three dashes on a newline indicate the breaking points.

---

Arrange all square-free positive integers in ascending order $a_1$, $a_2$, \ldots. Prove that there are infinitely many positive integers $n$ such that $a_{n+1}-a_n=2020$.

---

To begin, let $p_1$, $p_2$, \ldots, $p_{2019}$ be primes, with $P=p_1p_2\cdots p_{2019}$, and let $c$ be an integer (that exists by Chinese Remainder theorem) such that $c\equiv -i\pmod{p_i^2}$ for $i=1,\ldots,2019$. Then for each $k$ of the form $k=P^2i+c$, the integers $k+1$, \ldots, $k+2019$ are all divisible by squares.

It will suffice to show there are infinitely many $k$ in the arithmetic sequence $i\mapsto P^2i+c$ such that $k$ and $k+2020$ are both squarefree.
\begin{remark}
    In what follows, we will cite the fact that $\sum_p\frac1{p^2}<\frac12$, where the sum ranges over primes $p$. Here is an easy proof: first note that
    \[\sum_{n\text{ odd}}\frac1{n^2}=\sum_n\frac1{n^2}-\sum_{n\text{ even}}\frac1{n^2}=\frac34\cdot\frac{\pi^2}6<\frac54\]
    by $\pi^2<10$. Then
    \[\sum_p\frac1{p^2}<-\frac1{1^2}+\frac1{2^2}+\sum_{n\text{ odd}}\frac1{n^2}<\frac12.\]
\end{remark}
Observe that for fixed primes $q$, the density of positive integers $i$ for which $q^2\mid P^2i+c$ is $\le1/q^2$. Similarly $i$ with $q^2\mid P^2i+c+2020$ have density $\le1/q^2$, so the density of $i$ for which $q^2$ divides either $P^2i+c$ or $P^2i+c+2020$ is $\le2/q^2$.

But the density of $i$ for which there exists $q$ that divide either $P^2i+c$ or $P^2i+c+2020$ is at most
\[\sum_q\frac2{q^2}<1,\]
so there are infinitely many $i$ for which $P^2i+c$ and $P^2i+c+2020$ are squarefree, as desired.

