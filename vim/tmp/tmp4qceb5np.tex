% Input your problem and solution below.
% Three dashes on a newline indicate the breaking points.

---

In the Cartesian coordinate plane define the strips $S_n=\{(x,y):n\le x<n+1\}$ for $n\in\mathbb Z$ and color each strip red or blue. Prove that any rectangle which is not a square can be placed in the plane so that its vertices have the same color.

---

Let this rectangle have dimensions $a\times b$, and for each $i$, let strip $S_i$ have color $c_i$.
\setcounter{claim}0
\begin{claim}
    If $a\not\in\mathbb Z$, then we are done.
\end{claim}
\begin{proof}
    Assume for contradiction an $a\times b$ rectangle may not be placed in the plane in such a way. Only consider axis-aligned rectangles with horizontal edge $a$ and vertical edge $b$. Then,
    \begin{itemize}[itemsep=0em]
        \item $c_n\ne c_{n+\left\lfloor a\right\rfloor}$, but
        \item $c_n\ne c_{n+\left\lceil a\right\rceil}$;
    \end{itemize}
    thus $c_{n+\left\lfloor a\right\rfloor}=c_{n+\left\lceil a\right\rceil}$. As we vary $n$, all strips have the same color, so we are clearly done.
\end{proof}

Henceforth $a$, $b$ are integers. The goal is to tilt the rectangle slightly in order to apply the above argument again. Again, assume for contradiction the problem is false.
\begin{claim}
    $\nu_2(a)=\nu_2(b)$.
\end{claim}
\begin{proof}
    Assume $\nu_2(a)<\nu_2(b)$. Then we have
    \begin{itemize}
        \item $c_0\ne c_a\ne c_{2a}\ne\cdots\ne c_{\lcm(a,b)}$; since $\lcm(a,b)/a$ is even, $c_0=c_{\lcm(a,b)}$.
        \item $c_0\ne c_b\ne c_{2b}\ne\cdots\ne c_{\lcm(a,b)}$; since $\lcm(a,b)/b$ is odd, $c_0\ne c_{\lcm(a,b)}$.
    \end{itemize}
    This is a clear contradiction.
\end{proof}
\begin{claim}
    $c_n\ne c_{n+\gcd(a,b)}$; in particular, $c_n=c_{n+2\gcd(a,b)}$.
\end{claim}
\begin{proof}
    Clearly $c_n\ne c_{n+a}$ and $c_n\ne c_{n+b}$ for all $a$, $b$. By B\'ezout's lemma we can select integers $m$, $n$ so that $am+bn=\gcd(a,b)$. Then $m+n$ is odd by parity, so $c_n$ and $c_{n+am+bn}$ have different colors.
\end{proof}
\begin{center}
    \begin{asy}
        size(4cm); defaultpen(fontsize(10pt));
        pair A,B,C,D,WW,X,Y,Z;
        A=(0,5);
        B=(10,29);
        C=(22,24);
        D=(12,0);
        WW=(0,0);
        X=(0,29);
        Y=(22,29);
        Z=(22,0);

        draw(WW--X--Y--Z--cycle,gray);
        draw(A--B--C--D--cycle);
        label("$a$",B--C,unit(A-B));
        label("$b$",D--C,unit(B-C));
        label("$t$",D--Z,S);
        label("$\frac ab\sqrt{b^2-t^2}$",( (2B+Y)/3)--Y,N);
        dot("$A$",A,W);
        dot("$B$",B,NW);
        dot("$C$",C,E);
        dot("$D$",D,S);
    \end{asy}
\end{center}
Consider a tilted $a\times b$ rectangle as shown above, with $t=2\gcd(a,b)$. Then $A$, $B$ are in strips with the same color, and so are $C$, $D$, so $B$, $C$ must be in strips of different color.

Let $s:=\frac ab\sqrt{b^2-t^2}$ be the vertical distance between $B$, $C$. If $s\in\mathbb Q$, then $\sqrt{b^2-t^2}\in\mathbb Z$; however, \[\sqrt{b^2-t^2}=\gcd(a,b)\cdot\sqrt{\left(\frac b{\gcd(a,b)}\right)^2-4}\notin\mathbb Z,\]
so $s$ is not an integer.

Then for each $n$,
\begin{itemize}[itemsep=0em]
    \item $c_n\ne c_{n+\left\lfloor s\right\rfloor}$, but
    \item $c_n\ne c_{n+\left\lceil s\right\rceil}$;
\end{itemize}
thus $c_{n+\left\lfloor s\right\rfloor}=c_{n+\left\lceil s\right\rceil}$. As we vary $n$, all strips have the same color, so we are done.

