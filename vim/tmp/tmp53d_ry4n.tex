% Input your problem and solution below.
% Three dashes on a newline indicate the breaking points.
% vim: tw=72

---

In acute triangle $ABC$, let $H_1$, $H_2$, $H_3$ be the feet of the altitudes from $A,B,C$, respectively, and let $T_1$, $T_2$, and $T_3$ be the points where the incircle touches $\overline{BC}$, $\overline{CA}$, and $\overline{AB}$, respectively. Prove that the reflections of $\overline{H_1H_2}$, $\overline{H_2H_3}$, and $\overline{H_3H_1}$ over $\overline{T_1T_2}$, $\overline{T_2T_3}$, and $\overline{T_3T_1}$, respectively, are the sides of a triangle that is inscribed in the incircle of $\triangle ABC$.

---

\begin{center}
    \begin{asy}
        size(10cm);
        defaultpen(fontsize(10pt));
        pair A, B, C, H1, H2, H3, I, T1, T2, T3, X, Y, Z, P, Yp;
        A=dir(110);
        B=dir(200);
        C=dir(340);
        H1=foot(A, B, C);
        H2=foot(B, C, A);
        H3=foot(C, A, B);
        I=incenter(A, B, C);
        T1=foot(I, B, C);
        T2=foot(I, C, A);
        T3=foot(I, A, B);
        X=2*foot(T2+T3-T1, T2, T3)-(T2+T3-T1);
        Y=2*foot(T3+T1-T2, T3, T1)-(T3+T1-T2);
        Z=2*foot(T1+T2-T3, T1, T2)-(T1+T2-T3);
        P=extension(T1, Y, T2, T3);
        Yp=2*foot(Y, T2, T3)-Y;

        filldraw(A -- B -- C -- A -- cycle, springgreen+opacity(0.05), springgreen);
        filldraw(incircle(A, B, C), springgreen+opacity(0.05), springgreen);
        filldraw(T1 -- T2 -- T3 -- cycle, red+opacity(0.05), red); filldraw(X -- Y -- Z -- cycle, blue+opacity(0.05), blue);
        draw(T1 -- P -- T2, purple);
        draw(H1 -- H2 -- H3 -- H1, dotted+orange);
        draw(H2 -- Yp, dotted+orange); draw(Y -- Yp, dashed+blue);
        draw(H2 -- P, dotted+orange);
        draw(arc((B+C)/2, C, B), orange);
        fill(arc((B+C)/2, C, B) -- cycle, orange+opacity(0.05));
        draw(B -- P, dashed+orange);

        dot("$X$", X, S);
        dot("$Y$", Y, SE);
        dot("$Z$", Z, W);
        dot("$Y'$", Yp, N);
        dot("$A$", A, N);
        dot("$B$", B, SW);
        dot("$C$", C, SE);
        dot("$T_1$", T1, S);
        dot("$T_2$", T2, N/2);
        dot("$T_3$", T3, dir(150));
        dot("$H_1$", H1, S);
        dot("$H_2$", H2, N);
        dot("$H_3$", H3, NW);
        dot("$P$", P, E);
    \end{asy}
\end{center}
Let $Y$ be the point on the incircle of $\triangle ABC$ such that $\overline{T_2Y}\parallel\overline{T_1T_3}$. I claim that $Y$ is one of the vertices of this triangle. Let $P=\overline{T_1Y}\cap\overline{T_2T_3}$. Since $BT_1=BT_3$ and $PT_1=PT_3$, $\overline{BP}$ bisects $\angle ABC$. By the Iran Lemma, $P$ lies on $(BCH_2H_3)$ and $(CT_1T_2)$. Notice that \[\measuredangle H_2PB=\measuredangle H_2CB=\measuredangle T_2PY.\]However, $BT_1PT_3$ is a kite, so $\overline{PB}$ and $\overline{PH_2}$ are reflections across $\overline{T_2T_3}$; moreover, if $Y'$ denotes the reflection of $Y$ across $\overline{T_2T_3}$, \[\measuredangle H_2PT_2=\measuredangle BPY=\measuredangle Y'PH_2.\]Since $PY'=PY=PT_2$, by SAS, $\triangle PH_2Y'\cong\triangle PH_2T_2$, so $Y'$ is the reflection of $T_2$ across $\overline{PH_2}$. Finally, \[\measuredangle PH_2H_3=\measuredangle PBH_3=\measuredangle CBP=\measuredangle CH_2P=\measuredangle PH_2Y',\]so $Y'$ lies on $\overline{H_2H_3}$. By symmetry, we are done. 

