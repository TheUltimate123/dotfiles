% Input your problem and solution below.
% Three dashes on a newline indicate the breaking points.

---

Players $A$ and $B$ play a ``paintful'' game on the real line. Player $A$ has a pot of paint with four units of black ink. A quantity $p$ of this ink suffices to blacken a (closed) real interval of length $p$. In every round, player $A$ picks some positive integer $m$ and provides $1/2^m$ units of ink from the pot. Player $B$ then picks an integer $k$ and blackens the interval from $k/2^m$ to $(k+1)/2^m$ (some parts of this interval may have been blackened before). The goal of player $A$ is to reach a situation where the pot is empty and the interval $[0,1]$ is not completely blackened.

Decide whether there exists a strategy for player $A$ to win in a finite number of moves.

---

Player $B$ can guarantee player $A$ never wins by playing greedily: at every step, player $B$ selects the least nonnegative integer $k$ such that the interval from $k/2^m$ to $(k+1)/2^m$ is fully white. (If no such $k$ exists, just throw away the paint.)

Now this is a one player game, so assume for contradiction $A$ has a winning sequence of moves by selecting $m$ equal to $m_1$, $m_2$, \ldots, $m_n$, for $n$ minimal. Recall that
\[\sum_{i=1}^n\frac1{2^{m_i}}=4,\]
so for binary reasons, the sequence $m_1$, $m_2$, \ldots, $m_n$ does not have a unique maximum.

Let $m_i=m_j$ be maximal among $m_1$, $m_2$, \ldots, $m_n$, with $i$, $j$ minimal. When player $B$ is given $m_i$, suppose he colors the interval from $\ell/2^{m_i}$ to $(\ell+1)/2^{m_i}$. Then since $m_{i+1}$, \ldots, $m_{j-1}$ are all less than $m_i$, the interval from $(\ell+1)/2^{m_i}$ to $(\ell+2)/2^{m_i}$ will remain white until player $B$ is given $m_j$, so $B$ will fill that interval.

Then this strategy is equivalent to having a single instance of $m_i-1$ rather than two instances of $m_i$; that is, the sequence
\[m_1,\;m_2,\;\ldots,\;m_{i-1},\;m_i-1,\;m_{i+1},\;\ldots,\;m_{j-1},\;m_{j+1},\;\ldots,m_n\]
works just as well, contradicting the minimality of $n$.

