% Input your problem and solution below.
% Three dashes on a newline indicate the breaking points.

---

Let $P_1P_2\cdots P_{100}$ be a cyclic $100$-gon, and let $P_i=P_{i+100}$ for all $i$. Define $Q_i$ as the intersection of diagonals $P_{i-2}P_{i+1}$ and $P_{i-1}P_{i+2}$ for all integers $i$.

Suppose there exists a point $P$ satisfying $\seg{PP_i}\perp\seg{P_{i-1}P_{i+1}}$ for all integers $i$. Prove that the points $Q_1$, $Q_2$, $\ldots$, $Q_{100}$ are concyclic.

---

Replace $100$ with $n$. Shown below is an example for $n=8$. Let $R_i=\seg{PP_i}\cap\seg{P_{i-1}P_{i+1}}$ be the foot from $P$ to $\seg{P_{i-1}P_{i+1}}$, and let $S_i=\seg{P_{i-1}P_{i+1}}\cap\seg{P_iP_{i+2}}$.
\begin{center}
    \begin{asy}
        size(13cm); defaultpen(fontsize(10pt));
        pen pri=blue;
        pen sec=deepcyan;
        pen sec2=heavygreen;
        pen tri=royalblue;
        pen tri2=heavycyan;
        pen qua=lightblue;
        pen qua2=mediumblue;
        pen fil=pri+opacity(0.05);
        pen sfil=sec+opacity(0.05);
        pen sfil2=sec2+opacity(0.05);
        pen tfil=tri+opacity(0.05);
        pen tfil2=tri2+opacity(0.05);
        pen qfil=qua+opacity(0.05);
        pen qfil2=qua2+opacity(0.05);

        real t=3.5;
        pair O,PP,RR,QQ;
        pair[] P,Q,R,SS,T;
        O=(0,0);
        PP=0.15*dir(110+t);
        P[0]=dir(90+t);
        P[1]=dir(57.4+t);
        for (int i = 2; i <= 8; i += 1)
            P[i]=reflect(0,O+P[i-1]-PP)*P[i-2];
        for (int i = 0; i < 8; i += 1)
            Q[i]=extension(P[(i+6)%8],P[(i+1)%8],P[(i+7)%8],P[(i+2)%8]);
        for (int i = 0; i < 8; i += 1)
            R[i]=foot(PP,P[(i+7)%8],P[(i+1)%8]);

        RR=circumcenter(R[0],R[1],R[2]);
        QQ=2RR-PP;

        for (int i = 0; i < 8; i += 1)
            SS[i]=extension(P[(i+7)%8],P[(i+1)%8],P[i],P[(i+2)%8]);
        for (int i = 0; i < 8; i += 1)
        T[i]=foot(QQ,P[(i+7)%8],P[(i+1)%8]);

        for(int i = 0; i < 8; i += 1) {
            draw(P[(i-1)%8]--P[(i+1)%8],sec);
            draw(PP--P[i],sec+dashed);

            draw(P[(i-1)%8]--P[(i+2)%8],tri);
            draw(QQ--T[i],qua+dashed);
        }

        filldraw(circumcircle(T[0],T[1],T[2]),qfil2,qua2);
        filldraw(Q[0]--Q[1]--Q[2]--Q[3]--Q[4]--Q[5]--Q[6]--Q[7]--cycle,tfil2,tri2+linewidth(0.8));
        filldraw(circumcircle(Q[0],Q[1],Q[2]),tfil2,tri2);
        filldraw(SS[0]--SS[1]--SS[2]--SS[3]--SS[4]--SS[5]--SS[6]--SS[7]--cycle,sfil2,sec2+linewidth(0.8));
        filldraw(circle(O,1),fil,pri+linewidth(0.8));
        filldraw(P[0]--P[1]--P[2]--P[3]--P[4]--P[5]--P[6]--P[7]--cycle,fil,pri+linewidth(0.8));
        dot("$P$", PP, S);
        dot("$Q$", QQ, N);

        for (int i = 0; i < 8; i += 1)
            dot("$P_" + string(i+1) + "$", P[i], P[i]);

        dot("$Q_1$", Q[0], N);
        dot("$Q_2$", Q[1], dir(15));
        dot("$Q_3$", Q[2], dir(-30));
        dot("$Q_4$", Q[3], dir(-40));
        dot("$Q_5$", Q[4], dir(240));
        dot("$Q_6$", Q[5], dir(235));
        dot("$Q_7$", Q[6], dir(185));
        dot("$Q_8$", Q[7], N);

        dot("$R_1$", R[0], unit(P[1]-P[7])*dir(45));
        dot("$R_2$", R[1], unit(P[2]-P[0])*dir(135));
        dot("$R_3$", R[2], unit(P[3]-P[1])*dir(135));
        dot("$R_4$", R[3], unit(P[4]-P[2])*dir(45));
        dot("$R_5$", R[4], unit(P[5]-P[3])*dir(45));
        dot("$R_6$", R[5], unit(P[6]-P[4])*dir(45));
        dot("$R_7$", R[6], unit(P[7]-P[5])*dir(45));
        dot("$R_8$", R[7], unit(P[0]-P[6])*dir(135));

        dot("$S_1$", SS[0], unit(SS[0]-incenter(SS[0],SS[7],SS[1])));
        dot("$S_2$", SS[1], unit(SS[1]-incenter(SS[1],SS[0],SS[2])));
        dot("$S_3$", SS[2], unit(SS[2]-incenter(SS[2],SS[1],SS[3])));
        dot("$S_4$", SS[3], unit(SS[3]-incenter(SS[3],SS[2],SS[4])));
        dot("$S_5$", SS[4], unit(SS[4]-incenter(SS[4],SS[3],SS[5])));
        dot("$S_6$", SS[5], unit(SS[5]-incenter(SS[5],SS[4],SS[6])));
        dot("$S_7$", SS[6], unit(-SS[6]+incenter(SS[6],SS[5],SS[7])));
        dot("$S_8$", SS[7], unit(-SS[7]+incenter(SS[7],SS[6],SS[0])));

        dot("$T_1$", T[0], unit(P[1]-P[7])*dir(135));
        dot("$T_2$", T[1], dir(100));
        dot("$T_3$", T[2], unit(P[3]-P[1])*dir(45));
        dot("$T_4$", T[3], unit(P[4]-P[2])*dir(90));
        dot("$T_5$", T[4], unit(P[5]-P[3])*dir(90));
        dot("$T_6$", T[5], unit(P[6]-P[4])*dir(135));
        dot("$T_7$", T[6], unit(P[7]-P[5])*dir(135));
        dot("$T_8$", T[7], unit(P[0]-P[6])*dir(90));
    \end{asy}
\end{center}
\setcounter{iclaim}0
\begin{iclaim}
    $R_1R_2\cdots R_n$ is cyclic.
\end{iclaim}
\begin{proof}
    Since $\angle P_iR_iP_{i+1}=\angle P_iR_{i+1}P_{i+1}=90\dg$ for each $i$, we have $PP_i\cdot PR_i=PP_{i+1}\cdot PR_{i+1}$, meaning inversion at $P$ with radius $\sqrt{PP_i\cdot PR_i}$, which is fixed, sends the circumcircle of $P_1P_2\cdots P_n$ to that of $R_1R_2\cdots R_{100}$.
\end{proof}
\begin{iclaim}
    $\seg{Q_iQ_{i+1}}\parallel\seg{R_iR_{i+1}}$ for all $i$.
\end{iclaim}
\begin{proof}
    By Reim's theorem on $(P_iR_iR_{i+1}P_{i+1})$ and $(P_{i-1}P_iP_{i+1}P_{i+2})$, $\seg{Q_iQ_{i+1}}=\seg{P_{i-1}P_{i+2}}\parallel\seg{R_iR_{i+1}}$.
\end{proof}

Since $P$ has a pedal circle in $S_1S_2\cdots S_{100}$, the reflection $Q$ over the center of the pedal circle is the isogonal conjugate of $P$ in $S_1S_2\cdots S_{100}$. Let $T_i$ be the foot from $Q$ to $\seg{P_{i-1}P_{i+1}}$, so that there is a fixed circle through both $R_i$ and $T_i$ for all $i$.
\begin{iclaim}
    $\triangle P_{i-1}P_iP_{i+1}$ and $\triangle T_{i-1}T_iT_{i+1}$ are homothetic.
\end{iclaim}
\begin{proof}
    By Reim's theorem on $(P_iR_iR_{i-1}P_{i-1})$ and $(R_iT_iT_{i-1}R_{i-1})$, we have $\seg{P_iP_{i-1}}\parallel\seg{T_iT_{i-1}}$, and similarly $\seg{P_iP_{i+1}}\parallel\seg{T_iT_{i+1}}$. Also $P_iS_{i-1}\cdot P_iR_{i-1}=P_iR_i\cdot P_iP=P_iS_{i+1}\cdot P_iR_{i+1}$, so $R_{i-1}S_{i-1}S_iR_{i+1}$ is cyclic, and $\seg{P_{i-1}P_{i+1}}\parallel\seg{R_{i-1}R_{i+1}}$ by Reim's theorem on $(R_{i-1}S_{i-1}S_iR_{i+1})$ and $(R_{i-1}T_{i-1}T_{i+1}R_{i+1})$.
\end{proof}
\begin{iclaim}
    $\seg{QD_i}\perp\seg{Q_iQ_{i+1}}$ for all $i$.
\end{iclaim}
\begin{proof}
    Since $S_i$ is the orthocenter of $\triangle PP_iP_{i+1}$, we know $\seg{PS_i}\perp\seg{P_iP_{i+1}}$. Note that $\seg{S_iP}$ and $\seg{S_iQ}$ are isogonal wrt.\ $\angle P_iS_iP_{i+1}$, and $\seg{P_iP_{i+1}}$ and $\seg{P_{i-1}P_{i+2}}$ are isogonal, so $\seg{QS_i}\perp\seg{P_{i-1}P_{i+2}}$, as claimed.
\end{proof}
\begin{iclaim}
    $Q$, $Q_i$, $T_i$ are collinear for all $i$.
\end{iclaim}
\begin{proof}
    By the parallel lines from Claim 3, \[\frac{T_iS_{i-1}}{S_{i-1}P_{i-1}}=\frac{T_iT_{i-1}}{P_iP_{i-1}}=\frac{T_iT_{i+1}}{P_iP_{i+1}}=\frac{T_iS_i}{S_iP_{i+2}}.\]
    Hence a homothety at $S$ sends $\triangle Q_iP_{i-1}P_{i+1}$ to $\triangle H_iS_{i-1}S_i$ for some point $H_i$ with $\seg{QS_i}\perp\seg{H_iS_{i-1}}$ and $\seg{QS_{i-1}}\perp\seg{H_iS_i}$. It follows that $H_i$ is the orthocenter of $\triangle Q_iS_{i-1}S_i$, so $H_i$ lies on $\seg{QT_i}$. This is sufficient, since $H_i\in\seg{T_iQ_i}$ by the homothety.
\end{proof}

From this, $\triangle PR_iR_{i+1}$ and $\triangle QQ_iQ_{i+1}$ for each $i$, so $R_1R_2\cdots R_n\sim Q_1Q_2\cdots Q_n$. Since the former is cyclic, so is the latter, thus concluding the proof.

