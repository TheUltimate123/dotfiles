% Input your problem and solution below.
% Three dashes on a newline indicate the breaking points.

---

Let $p$ be an odd prime. Denote by $A$ the set of all integers $a$ such that $1\le a<p$, and both $a$ and $4-a$ are quadratic non-residues modulo $p$. Calculate the remainder when the product of the elements of $A$ is divided by $p$.

---

The answer is $2$, regardless of the choice of $p$.
First check that
\begin{align*}
    4|A|&\equiv\sum_{a=0}^{p-1}\left(a^{(p-1)/2}-1\right)\left( (4-a)^{(p-1)/2}-1\right)\\
    &\equiv\sum_{a=0}^{p-1}[a(4-a)]^{(p-1)/2}
    \equiv\sum_{a=0}^{p-1}\left(-a^2\right)^{(p-1)/2}\\
    &\equiv-(-1)^{(p-1)/4}\pmod p.
\end{align*}
where the third equality is since all terms with degree less than $p-1$ disappear upon summation.

\bigskip

\textbf{Proof for $p\equiv3\pmod4$:}
Observe that for all $a\in A$, we also have $\frac{16}a\in A$.

Since $|A|<p$, we have $|A|=\frac{p+1}4$. If $p\equiv7\pmod8$, then $|A|$ is even, so pairing up the elements of $A$, \[\prod_{a\in A}a\equiv16^{|A|/2}\equiv2\cdot2^{(p-1)/2}=2\pmod p.\]
Otherwise $p\equiv3\pmod8$ and $-4\in A$ is paired with itself, so \[\prod_{a\in A}a\equiv(-4)\cdot16^{(|A|-1)/2}\equiv(-4)\frac{2^{(p-1)/2}}2\equiv2\pmod p.\]

\bigskip

\textbf{Proof for $p\equiv1\pmod4$:}
Since $|A|<p$, we have $|A|=\frac{p-1}4$.

Let $B$ denote the set of $b$ such that $b$ and $4-b$ are both quadratic residues; assume $0,4\notin B$.

Check that
\[4|A|\equiv\sum_{a=0}^{p-1}[a(4-a)]^{(p-1)/2}\equiv2(|A|+|B|)-2\pmod p,\]
so if $p=4k+1$, then $|A|=k$ and $|B|=k-1$.
\begin{claim*}
    $x\in A\cup B\setminus\{2\}$ if and only if $(2-x)^2\in B$.
\end{claim*}
\begin{proof}
    The first condition is equivalent to $x(4-x)=4-(2-x)^2$ being a square. Additionally note that $(2-x)^2\equiv(2-y)^2$ if and only if $x\equiv y$ or $x+y\equiv4$.
\end{proof}

Now let $B=\{b_1^2,b_2^2,\ldots,b_{k-1}^2\}$, and assume $b_1^2+b_2^2\equiv b_3^2+b_4^2\equiv\cdots\equiv4\pmod p$. If $k$ is even, then there's an additional term $b_{k-1}^2\equiv2\pmod p$. We can write
\begin{align*}
    b_1^2&=(2-b_2)(2+b_2)\\
    b_2^2&=(2-b_1)(2+b_1)\\
    b_3^2&=(2-b_4)(2+b_4)\\
    b_4^2&=(2-b_3)(2+b_3)\\
    &\;\vdots
\end{align*}
Multiplying these together, we have
\[\frac12\prod_{x\in A\cup B}x=\prod_{x\in A\cup B\setminus\{2\}}x=\prod_{i=0}^{k-1}(2-b_i)(2+b_i)=\prod_{i=0}^{k-1}b_i^2=\prod_{b\in B}b,\]
which implies the desired conclusion.

