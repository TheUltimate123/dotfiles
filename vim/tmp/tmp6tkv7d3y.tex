% Input your problem and solution below.
% Three dashes on a newline indicate the breaking points.

---

Points $A_1$, $B_1$, $C_1$ are chosen on sides $BC$, $CA$, $AB$ of a triangle $ABC$ respectively. The circumcircles of triangles $AB_1C_1$, $BC_1A_1$, $CA_1B_1$ intersect the circumcircle of triangle $ABC$ again at points $A_2$, $B_2$, $C_2$ respectively. Points $A_3$, $B_3$, $C_3$ are symmetric to $A_1$, $B_1$, $C_1$ with respect to the midpoints of sides $BC$, $CA$, $AB$ respectively. Prove that triangles $A_2B_2C_2$ and $A_3B_3C_3$ are similar.

---

\begin{center}
    \begin{asy}
        size(6cm); defaultpen(fontsize(10pt));

        pen pri=red;
        pen sec=orange;
        pen tri=fuchsia;
        pen fil=pri+opacity(0.05);
        pen sfil=sec+opacity(0.05);
        pen tfil=tri+opacity(0.05);

        pair A,B,C,A1,B1,C1,A2,B2,C2,A3,B3,C3;
        A=dir(110);
        B=dir(210);
        C=dir(330);
        A1=(2B+C)/3;
        B1=(2C+A)/3;
        C1=(2A+B)/3;
        A2=reflect( (0,0),circumcenter(A,B1,C1))*A;
        B2=reflect( (0,0),circumcenter(B,C1,A1))*B;
        C2=reflect( (0,0),circumcenter(C,A1,B1))*C;
        A3=B+C-A1;
        B3=C+A-B1;
        C3=A+B-C1;

        filldraw(A3--B3--C3--cycle,tfil,tri);
        filldraw(circumcircle(A,B1,C1),sfil,sec);
        filldraw(circumcircle(B,C1,A1),sfil,sec);
        filldraw(circumcircle(C,A1,B1),sfil,sec);
        filldraw(A2--B2--C2--cycle,sfil,sec);
        filldraw(circumcircle(A,B,C),fil,pri);
        filldraw(A1--B1--C1--cycle,fil,pri);
        filldraw(A--B--C--cycle,fil,pri);

        dot("$A$",A,dir(120));
        dot("$B$",B,SW);
        dot("$C$",C,dir(-15));
        dot("$A_1$",A1,S);
        dot("$B_1$",B1,E);
        dot("$C_1$",C1,NW);
        dot("$A_2$",A2,NE);
        dot("$B_2$",B2,NW);
        dot("$C_2$",C2,dir(255));
        dot("$A_3$",A3,S);
        dot("$B_3$",B3,E);
        dot("$C_3$",C3,W);
        dot(reflect(circumcenter(B,C1,A1),circumcenter(C,A1,B1))*A1);
    \end{asy}
\end{center}
We begin with this observation.
\begin{iclaim*}
    $\triangle A_3BC\sim\triangle AC_3B_3$, and similarly $\triangle B_3CA\sim\triangle BA_3C_3$ and $\triangle C_3AB\sim\triangle CB_3A_3$.
\end{iclaim*}
\begin{proof}
    Clearly $\da CA_3B=\da B_3AC_3$. Note that by spiral similarity, \[\frac{A_3B}{A_3C}=\frac{BC_1}{CB_1}=\frac{AC_3}{AB_3},\]
    so the conclusion follows from SAS.
\end{proof}

From here,
\begin{align*}
    \da C_2A_2B_2&=\da C_2A_2A+\da AA_2B_2=\da C_2BA+\da ACB_2\\
    &=\da CA_3B_3+\da C_3A_3B=\da C_3A_3B_3.
\end{align*}
Cyclically applying this, we conclude $\triangle A_2B_2C_2\sim\triangle A_3B_3C_3$, as desired.

