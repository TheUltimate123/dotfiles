% Input your problem and solution below.
% Three dashes on a newline indicate the breaking points.

---

In triangle $ABC$ with $AB\ne AC$, let its incircle be tangent to sides $BC$, $CA$, $AB$ at $D$, $E$, $F$, respectively. The internal angle bisector of $\angle BAC$ intersects lines $DE$ and $DF$ at $X$ and $Y$, respectively. Let $S$ and $T$ be distinct points on side $BC$ such that $\angle XSY=\angle XTY=90^\circ$. Finally, let $\gamma$ be the circumcircle of $\triangle AST$.
\begin{enumerate}[label=(\alph*),itemsep=0em]
    \item Show that $\gamma$ is tangent to the circumcircle of $\triangle ABC$.
    \item Show that $\gamma$ is tangent to the incircle of $\triangle ABC$.
\end{enumerate}

---

\begin{center}
    \begin{asy}
        size(8cm); defaultpen(fontsize(10pt));
        pen pri=blue;
        pen sec=purple;
        pen tri=fuchsia;
        pen fil=pri+opacity(0.05);
        pen sfil=sec+opacity(0.05);
        pen tfil=tri+opacity(0.05);

        pair A,B,C,I,D,EE,F,X,Y,SS,T,P;
        A=dir(150);
        B=dir(220);
        C=dir(320);
        I=incenter(A,B,C);
        D=foot(I,B,C);
        EE=foot(I,C,A);
        F=foot(I,A,B);
        X=extension(A,I,D,EE);
        Y=extension(A,I,D,F);
        SS=intersectionpoints(B--C,circle( (X+Y)/2,length(X-Y)/2))[0];
        T=2*foot(circumcenter(X,Y,SS),B,C)-SS;
        P=extension(A,X,B,C);

        filldraw(circumcircle(A,SS,T),tfil,tri);
        draw(B--X,tri); draw(C--Y,tri);
        filldraw(circumcircle(X,Y,T),sfil,sec);
        draw(EE--D,sec); draw(F--Y,sec);
        filldraw(circumcircle(A,B,C),fil,pri);
        filldraw(incircle(A,B,C),fil,pri);
        draw(A--Y,pri);
        draw(A--B--C--A,pri);

        dot("$A$",A,A);
        dot("$B$",B,B);
        dot("$C$",C,C);
        dot("$D$",D,dir(240));
        dot("$E$",EE,NE);
        dot("$F$",F,dir(150));
        dot("$X$",X,dir(170));
        dot("$Y$",Y,SE);
        dot("$S$",SS,SW);
        dot("$T$",T,SE);
        dot("$P$",P,SW);
        dot("$I$",I,dir(60));
    \end{asy}
\end{center}
Let $I$ be the incenter, and without loss of generality assume $AB<AC$. By the Iran lemma, $\seg{BX}$ and $\seg{CY}$ are perpendicular to $\seg{AI}$. Let $P=\seg{AI}\cap\seg{BC}$ and $Q$ the foot of the external angle bisector. Then \[-1=(BC;PQ)=(XY;PA)\]
by projection onto $\seg{AI}$ (perspectivity at infinity), but $\angle XSY=\angle XTY=90\dg$, so $\seg{SX}$ and $\seg{SY}$ bisect $\angle AST$ and $\seg{TX}$ and $\seg{TY}$ bisect $\angle ATS$. It follows that $X$ and $Y$ are the incenter and $A$-excenter of $\triangle AST$, so $\seg{AS}$ and $\seg{AT}$ are isogonal, and part (a) follows.

For part (b), note that \[\da IDX=\da IDE=90\dg+\da DFE=\da(\seg{DF},\seg{AI})=\da DYI,\]
whence $IX\cdot IY=ID^2$, and $(I)$ and $(XY)$ are orthogonal. Thus inversion at $(XY)$ fixes $(I)$ and sends $(AST)$ to $\seg{BC}$, which is tangent to $(I)$, as desired.

