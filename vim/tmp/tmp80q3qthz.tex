% Input your problem and solution below.
% Three dashes on a newline indicate the breaking points.
% vim: tw=72

---

Circles $\omega_1$ and $\omega_2$ intersect at points $P$ and $K$. Line $XY$ is tangent to $\omega_1$ at $X$ and $\omega_2$ at $Y$. Let lines $YP$ and $XP$ meet $\omega_1$ and $\omega_2$, respectively, again at $B$ and $C$, respectively, and let lines $BX$ and $CY$ meet at $A$. Prove that if the circumcircles of $\triangle ABC$ and $\triangle AXY$ meet again at $Q$, then $\angle QXA=\angle QKP$.

---

\begin{center}
    \begin{asy}
        size(12cm);
        defaultpen(fontsize(10pt));

        pen pri=deepgreen+linewidth(0.5);
        pen sec=royalblue+linewidth(0.5);
        pen tri=rgb(41, 207, 255)+linewidth(0.5);
        pen fil=deepgreen+opacity(0.05);
        pen sfil=royalblue+opacity(0.05);
        pen tfil=rgb(41, 207, 255)+opacity(0.05);

        pair X, Y, O1, O2, P, K, B, C, A, Q, L;
        X=(-1, 0);
        Y=(1, 0);
        O1=(-1, -0.9);
        O2=(1, -1.3);
        P=intersectionpoints(circle(O1, length(X-O1)), circle(O2, length(Y-O2)))[0];
        K=intersectionpoints(circle(O1, length(X-O1)), circle(O2, length(Y-O2)))[1];
        B=intersectionpoint(circle(O1, length(X-O1)), (P+(P-Y)*0.01) -- (P+(P-Y)*100));
        C=intersectionpoint(circle(O2, length(Y-O2)), (P+(P-X)*0.01) -- (P+(P-X)*100));
        A=extension(B, X, C, Y);
        Q=intersectionpoints(circumcircle(A, B, C), circumcircle(A, X, Y))[1];
        L=intersectionpoint(K -- (K+(K-P)*100), circumcircle(B, C, P));

        X=(X-B)/(C-B);
        Y=(Y-B)/(C-B);
        O1=(O1-B)/(C-B);
        O2=(O2-B)/(C-B);
        P=(P-B)/(C-B);
        K=(K-B)/(C-B);
        A=(A-B)/(C-B);
        Q=(Q-B)/(C-B);
        L=(L-B)/(C-B);
        B=(0,0);
        C=(1,0);

        filldraw(circle(O1, length(X-O1)), fil, pri);
        filldraw(circle(O2, length(Y-O2)), fil, pri);
        draw(B -- Y -- X -- C, pri);
        draw(B -- A -- C -- B, sec);
        filldraw(circumcircle(A, B, C), sfil, sec);
        filldraw(circumcircle(A, X, Y), sfil, sec);
        draw(P -- L, tri);
        draw(B -- L -- C, tri);
        filldraw(circumcircle(B, P, C), tfil, tri);

        dot("$X$", X, dir(100));
        dot("$Y$", Y, dir(80));
        dot("$P$", P, dir(100));
        dot("$K$", K, dir(-30));
        dot("$B$", B, dir(210));
        dot("$C$", C, dir(-30));
        dot("$A$", A, N);
        dot("$Q$", Q, dir(120));
        dot("$L$", L, S);
    \end{asy}
\end{center}
Note that $K$ is the Miquel point of $BXCY$ and $Q$ is the Miquel point of $BXYC$. Let the spiral similarity at $Q$ sending $\overline{XY}$ to $\overline{BC}$ send $K$ to $L$, so $\triangle BLC\sim\triangle XKY$. Note that
\begin{align*}
    \measuredangle BLC&=\measuredangle XKY=\measuredangle XKP+\measuredangle XKP+\measuredangle PKY\\
    &=\measuredangle YXP+\measuredangle PYX=\measuredangle YPX=\measuredangle BPC,
\end{align*}
so $BPCL$ is cyclic. Moreover, \[\measuredangle BPL=\measuredangle BCL=\measuredangle XYK=\measuredangle YCK=\measuredangle BXK=\measuredangle BPK,\]
whence $L$ lies on $\overline{PK}$. It follows that $\measuredangle QKP=\measuredangle QKL=\measuredangle QXB=\measuredangle QXA$, as required.

