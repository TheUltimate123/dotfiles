% Input your problem and solution below.
% Three dashes on a newline indicate the breaking points.
% vim: tw=72

---

Karl starts with $n$ cards labeled $1,2,3,\ldots,n$ lined up in a random order on his desk. He calls a pair $(a,b)$ of these cards swapped if $a>b$ and the card labeled $a$ is to the left of the card labeled $b$. For instance, in the sequence of cards $3,1,4,2$, there are three swapped pairs of cards, $(3,1)$, $(3,2)$, and $(4,2)$.

He picks up the card labeled $1$ and inserts it back into the sequence in the opposite position: if the card labeled $1$ had $i$ cards to its left, then it now has $i$ cards to its right. He then picks up the card labeled $2$ and reinserts it in the same manner, and so on until he has picked up and put back each of the cards $1,2,\ldots,n$ exactly once in that order. (For example, the process starting at $3,1,4,2$ would be $3,1,4,2\to3,4,1,2\to2,3,4,1\to2,4,3,1\to2,3,4,1$.)

Show that, no matter what lineup of cards Karl started with, his final lineup has the same number of swapped pairs as the starting lineup.

---

At the end of every step, increment the label of the moved card by $n$. For instance, the process involving $3142$ is now \[3142\to3452\to6345\to6475\to6785.\]
Each card in the final configuration is just $n$ more than the corresponding card in the final configuration of the original procedure, and hence has the same number of inversions. However each step here preserves the number of inversions in the sequence, so the number of inversions is invariant.
