% Input your problem and solution below.
% Three dashes on a newline indicate the breaking points.
% vim: tw=72

---

Given a sequence of real numbers, a move consists of choosing two terms and replacing each with their arithmetic mean. Show that there exists a sequence of 2015 distinct real numbers such that after one initial move is applied to the sequence\textemdash no matter what move\textemdash there is always a way to continue with a finite sequence of moves so as to obtain in the end a constant sequence.

---

We claim that $-1007,-1006,\ldots,-1,0,1,\ldots,1006,1007$ works. Suppose the first move averages $a$ and $b$. Since sum is invariant, we need to transform this sequence into $0,0,\ldots,0$.
\begin{itemize}
    \item \textit{Case 1:} $a=-b$. Then, for all $c>0$ such that $c\ne |a|$, move $(c,-c)$ to $(0,0)$. Then, the sequence is $0,0,\ldots,0$, and we are done.
    \item \textit{Case 2:} $a\ne -b$ and $a\ne 0$. Then, move $(-a,-b)$ and subsequently $\big(\tfrac12(a+b),-\tfrac12(a+b)\big)$. Then, for all $c>0$ such that $c\notin\{|a|,|b|\}$, move $(c,-c)$ to $(0,0)$. Our sequence only contains $0$, and we are done.
    \item \textit{Case 3:} $b=0$. Then, for all $c>0$ such that $c\ne |a|$, move $(c,-c)$, so our sequence contains $-a,\tfrac a2,\tfrac a2$, and $2012$ zeros. Move $(-a,0)$ to $(-\tfrac a2,-\tfrac a2)$, and move the two $(\tfrac a2,-\tfrac a2)$'s so that our sequence only contains zeros, and we win.
\end{itemize}
We have covered all possible $(a,b)$, so we are done.
