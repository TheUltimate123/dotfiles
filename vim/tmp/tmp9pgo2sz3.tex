% Input your problem and solution below.
% Three dashes on a newline indicate the breaking points.

---

Ten points are marked in the plane. Is it always possible to place ten non-intersecting unit disks that cover all ten points?

---

The answer is yes. Randomly draw a hexagonal packing of unit circles.
\begin{center}
    \begin{asy}
        size(4cm); defaultpen(fontsize(10pt));
        pair A,B,C,D,EE,F;
        A=(0,0);
        B=(2,0);
        C=(4,0);
        D=(1,sqrt(3));
        EE=(3,sqrt(3));

        filldraw(circle(A,1),yellow+opacity(0.1),orange);
        filldraw(circle(B,1),yellow+opacity(0.1),orange);
        filldraw(circle(C,1),yellow+opacity(0.1),orange);
        filldraw(circle(D,1),yellow+opacity(0.1),orange);
        filldraw(circle(EE,1),yellow+opacity(0.1),orange);
        fill(B--D--EE--cycle,lightred+white);
        fill(D--( (D+EE)/2) -- arc(D,1,0,-60,CW) --  ( (D+B)/2)--cycle,lightblue+white);
        fill(EE--( (D+EE)/2) -- arc(EE,1,180,240,CCW) --  ( (EE+B)/2)--cycle,lightblue+white);
        fill(B--( (B+EE)/2) -- arc(B,1,60,120,CCW) --  ( (B+D)/2)--cycle,lightblue+white);
        draw(arc(D,1,0,-60,CW),orange);
        draw(arc(EE,1,180,240,CCW),orange);
        draw(arc(B,1,60,120,CCW),orange);
        draw(B--D--EE--B,cyan+blue+linewidth(1));
    \end{asy}
\end{center}
The probability a point is covered by one of the circle is \[\mathbb P(\text{covered})=\frac{\pi/2}{\sqrt3}=\frac{\pi}{2\sqrt3}>0.9,\]
hence if we draw $10$ points, the expected number of them that is covered is \[\mathbb E[\text{\# covered}]=10\cdot\mathbb P(\text{covered})>9.\]
Therefore one such hexagonal packing covers all $10$ points, and since each point is in the interior of at most one circle, all but at most $10$ of the circles may be deleted.
\begin{boxremark}
    Source: \url{https://2012.cccg.ca/papers/paper13.pdf}.

    The paper also proves that the minimum $k$ such that the smallest point set not coverable by disjoint unit disks satisfies $13\le k\le45$.
\end{boxremark}


