% Input your problem and solution below.
% Three dashes on a newline indicate the breaking points.
% vim: tw=72

---

Given a cyclic quadrilateral $ABCD$, let the diagonals $AC$ and $BD$ meet at $E$ and the lines $AD$ and $BC$ meet at $F$. The midpoints of $\overline{AB}$ and $\overline{CD}$ are $G$ and $H$, respectively. Show that $\overline{EF}$ is tangent at $E$ to the circle through the points $E$, $G$ and $H$.

---

\begin{center}
    \begin{asy}
        size(12cm);
        defaultpen(fontsize(10pt));

        pen pri=deepgreen;
        pen fil=springgreen+opacity(0.05);
        pen sec=chartreuse;
        pen sfil=chartreuse+opacity(0.05);

        pair A, B, C, D, EE, F, G, H, X, P, Q, I;
        A=dir(140);
        D=dir(85);
        C=dir(350);
        B=dir(190);
        EE=extension(A, C, B, D);
        X=extension(A, B, C, D);
        F=extension(A, D, B, C);
        G=(A+B)/2;
        H=(C+D)/2;
        P=extension(EE,F,A,B);
        Q=extension(EE,F,C,D);
        I=(EE+F)/2;

        filldraw(circumcircle(A,B,C),fil,pri);
        filldraw(circumcircle(EE,G,H),sfil,sec);
        draw(C--A--B--D--C--F--D--X--A,pri);
        draw(F--Q,pri);
        fill(F--C--D--cycle,fil);
        fill(X--B--C--cycle,fil);
        draw(H--I,sec);

        dot("$A$",A,NW);
        dot("$B$",B,SW);
        dot("$C$",C,SE);
        dot("$D$",D,NE);
        dot("$E$",EE,S);
        dot("$F$",F,W);
        dot("$X$",X,N);
        dot("$G$",G,SE);
        dot("$H$",H,NE);
        dot("$P$",P,NE);
        dot("$Q$",Q,NE);
        dot("$I$",I,S);

        clip((-100, -1.1) -- (100, -1.1) -- (100, 100) -- (-100, 100) -- cycle);
    \end{asy}
\end{center}
Let line $EF$ intersect $\overline{AB}$ and $\overline{CD}$ at $P$ and $Q$ respectively, and let lines $AB$ and $CD$ meet at $X$. By Ceva-Menelaus, $-1=(XP;AB)=(XQ;DC)$, so by the Midpoint of Harmonic Bundles Lemma, $XP\cdot XG=XA\cdot XB=XD\cdot XC=XQ\cdot XH$, thus $PGHQ$ is cyclic.

Denote by $I$ the midpoint of $\overline{EF}$. Check that $G$, $H$, $I$ lie on the Gauss line of $ACBD$. However it is well-known that $-1=(EF;PQ)$, so by the Midpoint of Harmonic Bundles Lemma, $IE^2=IP\cdot IQ=IG\cdot IH$. This completes the proof.

