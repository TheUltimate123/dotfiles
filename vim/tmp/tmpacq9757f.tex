% Input your problem and solution below.
% Three dashes on a newline indicate the breaking points.

---

Let $ABC$ be a triangle with incenter $I$ and contact triangle $DEF$. Let $M$ be the foot of the perpendicular from $D$ to $\seg{EF}$ and let $P$ be the midpoint of $\seg{DM}$. If $H$ is the orthocenter of triangle $BIC$, prove that $\seg{PH}$ bisects $\seg{EF}$.

---

This is the birthplace of the so-called ``Iran lemma,'' thus it seems most appropriate to provide its proof here.
\begin{boxlemma*}[Iran lemma]
    Let $ABC$ be a triangle. Then the $A$-intouch chord, the $B$-angle bisector, and the $C$-midline concur on the circle with diameter $\seg{BC}$.
\end{boxlemma*}
\begin{proof}\ 
    \begin{center}
        \begin{asy}
            size(4cm); defaultpen(fontsize(10pt));

            pair A,B,C,I,D,EE,F,M,NN,K;
            A=dir(110);
            B=dir(210);
            C=dir(330);
            I=incenter(A,B,C);
            D=foot(I,B,C);
            EE=foot(I,C,A);
            F=foot(I,A,B);
            M=(B+C)/2;
            NN=(C+A)/2;
            K=extension(B,I,EE,F);

            draw(B--K--C,gray);
            draw(M--K,gray);
            draw(circumcircle(C,D,EE),gray);
            draw(arc(M,length(C-M),0,180));
            draw(A--B--C--A);
            draw(incircle(A,B,C));
            draw(F--K);

            dot("$A$",A,N);
            dot("$B$",B,SW);
            dot("$C$",C,SE);
            dot("$D$",D,SW);
            dot("$E$",EE,2*N);
            dot("$F$",F,NW);
            dot("$M$",M,S);
            dot("$N$",NN,SW);
            dot("$K$",K,NE);
            dot("$I$",I,NW);
        \end{asy}
    \end{center}
    Let $I$ be the incenter, $DEF$ the contact triangle, and $M_AM_BM_C$ the medial triangle. Let $\seg{BI}$ intersect $(BC)$ again at $K$. Then $K\in(CI)$, so \[\da IEK=\da DIB+\da ICD=90\dg+\da CBI+\da ICB=\da IAF=\da IEK,\]
    whence $K\in\seg{EF}$.

    Now $\da KM_AC=2\da KBC=\da ABC$, so $\seg{KM_A}\parallel\seg{AB}$ and $K\in\seg{M_AM_B}$, completing the proof.
\end{proof}
\begin{center}
\begin{asy}
    size(8cm); defaultpen(fontsize(10pt));

    pen pri=deepblue;
    pen sec=Cyan;
    pen tri=deepcyan;
    pen fil=pri+opacity(0.05);
    pen sfil=sec+opacity(0.05);
    pen tfil=tri+opacity(0.05);
    
    pair A,B,C,I,D,EE,F,M,P,Q,R,NN,H;
    A=dir(130);
    B=dir(210);
    C=dir(330);
    I=incenter(A,B,C);
    D=foot(I,B,C);
    EE=foot(I,C,A);
    F=foot(I,A,B);
    M=foot(D,EE,F);
    P=(D+M)/2;
    Q=foot(C,B,I);
    R=foot(B,C,I);
    NN=(EE+F)/2;
    H=orthocenter(B,I,C);

    filldraw(circle(I,length(NN-I)),tfil,tri);
    fill(D--Q--R--cycle,tfil);
    draw(H--P,tri+dashed);
    draw(Q--D--R,tri);
    draw(B--H--C--R,tri);
    draw(B--Q,tri);
    filldraw(incircle(A,B,C),sfil,sec);
    draw(D--M,sec+dashed);
    filldraw(A--B--C--cycle,fil,pri);
    draw(A--I,pri+dashed);
    draw(Q--F,pri);

    dot("$A$",A,N);
    dot("$B$",B,SW);
    dot("$C$",C,SE);
    dot("$I$",I,S);
    dot("$D$",D,S);
    dot("$E$",EE,dir(80));
    dot("$F$",F,dir(170));
    dot("$M$",M,dir(110));
    dot("$P$",P,E);
    dot("$Q$",Q,NE);
    dot("$R$",R,NW);
    dot("$N$",NN,dir(150));
    dot("$H$",H,N);
\end{asy}
\end{center}
Let $Q=\seg{BI}\cap\seg{EF}\cap(BC)$ and $R=\seg{CI}\cap\seg{EF}\cap(BC)$ by the Iran lemma. Since $DQR$ is the orthic triangle of $\triangle BIC$, $I$ is its incenter and $H$ is its $D$-excenter. If $N$ denotes the midpoint of $\seg{EF}$, $N$ is tangency point of the incircle of $\triangle DQR$ and $\seg{QR}$, so the collinearity is well-known.

To spell it out, if $L$ is the foot from $H$ to $\seg{QR}$ and $K$ is the reflection $L$ over $H$, then the homothety at $D$ sending the incircle of $\triangle DQR$ to the $D$-excicle sends $N$ to $K$, so $D$, $N$, $K$ are collinear. Since $M$, $N$, $L$ are collinear, $P$, $N$, $H$ are collinear as well, completing the proof.

