% Input your problem and solution below.
% Three dashes on a newline indicate the breaking points.

---

For each integer $a_0>1$, define the sequence $a_0$, $a_1$, $a_2$ by: \[a_{n+1}=\begin{cases}
        \sqrt{a_n}&\text{if $\sqrt{a_n}$ is an integer,}\\
        a_n+3&\text{otherwise,}
    \end{cases}\quad\text{for each $n\ge0$.}
\]
Determine all values of $a_0$ for which there is a number $A$ such that $a_n=A$ for infinitely many values of $n$.

---

The answer is $3\mid a_0$. The proof is easy, but rigorizing it is substantially harder. Here is one among many ways of doing so.

\bigskip

\textbf{Proof of sufficiency:} Let $m$ be the minimum value attained by the sequence. If $3\mid a_0$, then all elements of the sequence are divisible by $3$.

I claim $m=3$. Indeed, if $m\ge6$, then $(m-3)^2\ge m$, so the sequence reaches $k^2$ for some $k\le m-3$ and subsequently $k\le m-3<m$. This contradicts the minimality of $m$, as required.

Then the sequence repeats $3\to6\to9\to3\to\cdots$, end proof.

\bigskip

\textbf{Proof of necessity:} Firstly if $a_k\equiv2\pmod3$ for some $k$, then $a_{k+i}=a_k+3i$ for all $i$ since $2$ is not a quadratic residue modulo $3$; thus the sequence is unbounded and the problem does not hold. Let $m$ be the minimum value attained by the sequence.

I claim $m\equiv2\pmod3$, which proves the problem. Assume for contradiction $m\equiv1\pmod3$; then $m\ge4$ implies $(m-2)^2\ge m$, so the sequence reaches $k^2$ for some $k\le(m-2)^2$ and subsequently $k\le m-2<m$. This contradicts the minimality of $m$, as required.


