% Input your problem and solution below.
% Three dashes on a newline indicate the breaking points.
% vim: tw=72

---

Let $ABCD$ be a convex quadrilateral with $BA\ne BC$. Denote the incircles of triangles $ABC$ and $ADC$ by $\omega_1$ and $\omega_2$ respectively. Suppose that there exists a circle $\omega$ tangent to the extension of ray $BA$ past $A$, to the extension of ray $BC$ past $C$, to the line $AD$, and to the line $CD$. Prove that the common external tangents to $\omega_1$ and $\omega_2$ intersect on $\omega$.

---

\begin{center}
    \begin{asy}
        size(12cm);
        defaultpen(fontsize(10pt));

        pen pri=deepblue;
        pen sec=royalblue;
        pen tri=blue;
        pen qua=Cyan;
        pen qui=deepcyan;
        pen fil=pri+opacity(0.05);
        pen sfil=sec+opacity(0.05);
        pen tfil=tri+opacity(0.05);
        pen qfil=qua+opacity(0.05);
        pen qifil=qui+opacity(0.05);

        pair O, T1, T2, T3, T4, A, B, C, D, I, J, K, L, Kp, Lp, T;
        O=(0, 0);
        T1=dir(70);
        T2=dir(290);
        T3=dir(30);
        T4=dir(320);
        A=2*(T4+T1)/length(T4+T1)^2;
        B=2*(T1+T2)/length(T1+T2)^2;
        C=2*(T2+T3)/length(T2+T3)^2;
        D=2*(T3+T4)/length(T3+T4)^2;
        I=incenter(B, A, C);
        J=incenter(D, A, C);
        K=foot(I, A, C);
        L=foot(J, A, C);
        Kp=2I-K;
        Lp=2J-L;
        T=extension(O, foot(O, A, C), I, J);


        filldraw(circle(O, 1), sfil, sec);
        filldraw(incircle(B, A, C), tfil, tri);
        filldraw(incircle(D, A, C), tfil, tri);
        draw(T1 -- B -- T2, pri);
        draw(extension(D, A, B, C) -- D -- extension(D, C, B, A), pri);
        draw(A -- B -- C -- D -- A -- C, pri+linewidth(2));
        draw(K -- Kp, qua);
        draw(L -- Lp, qua);
        draw(K -- T -- B, qui);

        clip(T1 -- (T1+(100, 0)) -- (T2+(100, 0)) -- T2 -- cycle);

        dot("$A$", A, N);
        dot("$B$", B, E);
        dot("$C$", C, SE);
        dot("$D$", D, dir(170));
        dot("$K$", K, NW);
        dot("$L$", L, SE);
        dot("$K'$", Kp, SE);
        dot("$L'$", Lp, dir(110));
        dot("$T$", T, W);
    \end{asy}
\end{center}
Let $\omega$ touch $\overline{AB},\overline{BC},\overline{CD},\overline{DA}$ at $T_1,T_2,T_3,T_4$, respectively. Also let $\omega_1$ and $\omega_2$ touch $\overline{AC}$ at $K$ and $L$, respectively, and let $K'$ and $L'$ be their respective antipodes. Denote by $T$ the point on $\omega$ closest to $\overline{AC}$. Check that
\begin{align*}
    AB+AD&=BT_1-AT_1+AT_4-DT_4=BT_2-DT_3\\
    &=BT_2-CT_2+CT_3-DT_3=CB+CD.
\end{align*}
Note that \[AK=\frac{AC+AB-CB}2=\frac{AC+CD-AD}2=CL,\]
so $L$ is the $B$-extouch point of $\triangle ABC$. Analogously, $K$ is the $D$-extouch point of $\triangle ADC$, so $K'\in\overline{BL}$ and $L'\in\overline{DK}$. The homothety at $B$ between $\omega$ and $\omega_1$ maps $T$ to $K'$, so $T\in\overline{BK'L}$. Similarly, the negative homothety at $D$ between $\omega$ and $\omega_2$ maps $T$ to $L'$, so $T\in\overline{DL'K}$. It follows that $T=\overline{K'L}\cap\overline{KL'}$. However, since $\overline{KK'}\parallel\overline{LL'}$, $T$ is the center of homothety between $\overline{KK'}$ and $\overline{LL'}$, and since $\overline{KK'}$ and $\overline{LL'}$ are diameters of $\omega_1$ and $\omega_2$, $T$ is the exsimilicenter of $\omega_1$ and $\omega_2$, which lies on $\omega$, as desired. 

