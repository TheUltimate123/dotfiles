% Input your problem and solution below.
% Three dashes on a newline indicate the breaking points.

---

Let $AXYZB$ be a convex pentagon inscribed in a semicircle with diameter $\seg{AB}$, and let $K$ be the foot of the altitude from $Y$ to $\seg{AB}$. Let $O$ denote the midpoint of $\seg{AB}$ and $L$ the intersection of $\seg{XZ}$ and $\seg{YO}$. Select a point $M$ on line $KL$ with $MA=MB$, and finally, let $I$ be the reflection of $O$ across $\seg{XZ}$. Prove that if quadrilateral $XKOZ$ is cyclic then so is quadrilateral $YOMI$.

---

Let $\Gamma$ be the circle with diameter $\seg{AB}$, and let $T=\seg{AB}\cap\seg{XZ}$. Denote by $N$ and $V$ the images of $M$ and $L$ under inversion at $\Gamma$. Let $S$ be the midpoint of $\seg{YN}$.

Since $TK\cdot TO=TX\cdot TZ=TA\cdot TB$, we have $-1=(AB;KT)$ by Midpoints of Harmonic Bundles lemma. In particular, $T$ and $K$ are inverses, and since $\seg{YK}$ is a symmedian of $\triangle YAB$, $\seg{TY}$ is tangent to $\Gamma$. Furthermore $TONV$ is cyclic with diameter $\seg{TN}$, and $\seg{YN}$ intersects $\Gamma$ again at a point $W$ on $(OYM)$.
\begin{center}
    \begin{asy}
        size(9cm); defaultpen(fontsize(10pt));

        pair O,A,B,X,Z,K,Y,T,L,Lp,M,U,NN,SS,P,Q,WW,I;
        O=(0,0);
        A=dir(180);
        B=dir(0);
        X=dir(163);
        Z=dir(65);
        K=2*foot(circumcenter(X,K,Z),A,B)-O;
        Y=intersectionpoint(circle(O,1),K--K+(0,50));
        T=extension(X,Z,A,B);
        L=extension(O,Y,X,Z);
        Lp=reflect(A,B)*(1/L);
        M=extension(K,L,O,O+(0,1));
        U=circumcenter(O,T,Lp);
        NN=2*foot(U,O,M)-O;
        SS=(Y+NN)/2;
        P=intersectionpoint(circle(O,1),K--M);
        Q=2*foot(O,P,K)-P;
        WW=2*foot(O,Y,SS)-Y;
        I=reflect(X,Z)*O;

        draw(circumcircle(X,O,Z),gray);
        draw(O--M,gray);
        draw(T--NN,gray);
        draw(circumcircle(O,Y,WW),Dotted);
        draw(O--I,Dotted);
        draw(circle(U,abs(U)),dashed);
        draw(WW--Y--T);
        draw(O--Lp);
        draw(circle(O,1));
        draw(B--T--Z);
        //draw(Q--M);

        dot("$O$",O,dir(290));
        dot("$A$",A,SW);
        dot("$B$",B,SE);
        dot("$X$",X,X);
        dot("$Z$",Z,NE);
        dot("$K$",K,dir(300));
        dot("$Y$",Y,dir(60));
        dot("$T$",T,SW);
        dot("$L$",L,dir(160));
        dot("$V$",Lp,NW);
        dot("$M$",M,N);
        dot("$U$",U,S);
        dot("$N$",NN,NE);
        dot("$S$",SS,NE);
        //dot("$P$",P,dir(110));
        //dot("$Q$",Q,SW);
        dot("$W$",WW,dir(-15));
        dot("$I$",I,NW);
    \end{asy}
\end{center}
Since $\seg{YK}\perp\seg{AB}$ and $\seg{NO}\perp\seg{AB}$, we have $SK=SO$. Furthermore if $U$ denotes the center of $(TONV)$, $\seg{SU}$ is the $N$-midsegment of $\triangle MYU$, thus $\seg{SU}\parallel\seg{TT}$. But $\seg{YT}\perp\seg{YO}$, which is the radical axis of $(XKOZ)$ and $(TONV)$, so $S$ is the center of $(XKOZ)$.

Finally $S$ and $I$ are inverses, but $Y$, $S$, $N$ are collinear, so $YOMI$ is cyclic, and we are done.

