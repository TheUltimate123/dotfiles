% Input your problem and solution below.
% Three dashes on a newline indicate the breaking points.

---

Suppose that an arrow is drawn on each edge of a convex polyhedron, giving each edge a direction, in such a way that every vertex of the polyhedron has at least one arrow coming out of it and at least one arrow going into it. Prove that under these conditions, it is always possible to find a face of the polyhedron such that the directions of the boundary edges of that face go in a cycle.

---

Say an angle of a face is \emph{good} if it can be traversed along the drawn arrows, and \emph{bad} otherwise. Let $V$, $E$, $F$ be the sets of vertices, edges, and faces, respectively.

Assume for contradiction every vertex has positive indegreee and outdegree, but no face forms a cycle. We rely on two orthogonal estimates, based on parity issues:
\begin{itemize}[itemsep=0em]
    \item Each face with $S$ sides contains at most $S-2$ good angles.
    \item At each vertex of the cube, there are at least two good angles.
\end{itemize}
Let $I$ be the number of good angles. Observe \[I\le\sum_{f\in F}(\text{\# sides}-2)=2|E|-2|F|=2|V|-4<2|V|\le I,\]
which is absurd.
\begin{boxremark}
    The version of the problem with a cube is from Po-Shen Loh's lecture ``Graph theory: introduction'' at MOP 2019, which is based on his 21-228 Discrete Mathematics course at Carnegie Mellon University.
\end{boxremark}

