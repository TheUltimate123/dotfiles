% Input your problem and solution below.
% Three dashes on a newline indicate the breaking points.

---

A set $S$ of points from the space is called \emph{completely symmetric} if it has at least three elements and fulfills the following condition: for every two distinct points $A$ and $B$ in $S$, the perpendicular bisector plane of segment $AB$ is a plane of symmetry of $S$. Prove that if a completely symmetric set is finite, then it consists of the vertices of either a regular polygon, a regular tetrahedron, or a regular octahedron.

---

First, we verify the planar case:
Let $G$ be the centroid of the points in $S$; then $G$ lies on each perpendicular bisector $AB$, so $GA=GB$ for all $A$, $B$ in $S$; i.e.\ the points in $S$ lie on a circle centered at $G$. Let the points of $S$ lie on the circle in the order $A_1$, $A_2$, \ldots, $A_n$. We can verify for each $i$ by looking at the perpendicular bisector of $\seg{A_{i-1}A_{i+1}}$ that $A_iA_{i-1}=A_iA_{i+1}$, so the polygon is regular, as needed.

Then assume the points are not all planar. Let $G$ be the centroid again, so we can verify analogously that the points in $S$ lie on a sphere centered at $G$, and consider the polyhedron formed by the points in $S$.

\paragraph{First finish}
Observe that any plane containing at least three points must form a regular polygon. We will verify the following claim:
\begin{claim*}
    If we have a plane containing at least four points, and there is a point $P$ not on the plane, then there is a point $P$ on the other side of the plane as $P$.
\end{claim*}
\begin{proof}
    Let the plane contain regular polygon $A_1A_2\cdots A_k$ with side length $s$. If there is a point $P$ ``above'' the plane, then we can select a point $Q$ on the plane containing $\triangle PA_1A_2$ yet still ``above'' the plane with $QA_1=s$.

    By $k\ge4$, we have $QA_1=A_1A_2<A_1A_3$, so the perpendicular bisector plane of $\seg{QA_3}$ intersects segment $A_1A_3$; that is, the reflection of $A_1$ over the perpendicular bisector plane lies ``below'' the plane, and it must also be in $S$.
\end{proof}

It follows that every face of the polyhedron is an equilateral triangle, so the polyhedron is either a tetrahedron, octahedron, or icosahedron (by platonic solids). To rule out the icosahedron, take the space diagonal; the two pentagons do not pair up. 

\paragraph{Second sketch of finish}
Take a vertex $P$ with neighbors $A_1$, $A_2$, \ldots, $A_k$. Evidently $PA_1=PA_2=\cdots=PA_k$, so $A_1$, \ldots, $A_k$ lie on a sphere centered at $P$; but they also lie on a sphere centered at $G$, so they lie on a circle (i.e.\ they're coplanar). But also recall that $A_1A_2=A_2A_3=\cdots=A_kA_1$, so this proves that all faces adjacent to $P$ have the same number of sides.

Then we have reduced the problem to verifying with platonic solids work, a finite case check we shall omit.
