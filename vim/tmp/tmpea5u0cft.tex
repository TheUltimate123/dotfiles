% Input your problem and solution below.
% Three dashes on a newline indicate the breaking points.
% vim: tw=72

---

In convex cyclic quadrilateral $ABCD$, we know that lines $AC$ and $BD$ intersect at $E$, lines $AB$ and $CD$ intersect at $F$, and lines $BC$ and $DA$ intersect at $G$. Suppose that the circumcircle of $\triangle ABE$ intersects line $CB$ at $B$ and $P$, and the circumcircle of $\triangle ADE$ intersects line $CD$ at $D$ and $Q$, where $C,B,P,G$ and $C,Q,D,F$ are collinear in that order. Prove that if lines $FP$ and $GQ$ intersect at $M$, then $\angle MAC=90^\circ$.

---

\begin{customenv}{First solution, by Pappus' Theorem}\
    \begin{center}
        \begin{asy}
            size(12cm);
            defaultpen(fontsize(10pt));
            pen pri=royalblue+linewidth(0.5);
            pen sec=deepgreen+linewidth(0.5);
            pen tri=springgreen+linewidth(0.5);
            pen fil=royalblue+opacity(0.05);
            pen sfil=deepgreen+opacity(0.05);
            pen tfil=springgreen+opacity(0.05);
            pair A, B, C, D, EE, F, G, P, Q, T, M;
            A=dir(165);
            B=dir(85);
            C=dir(350);
            D=dir(190);
            EE=extension(A, C, B, D);
            F=extension(A, B, C, D);
            G=extension(A, D, B, C);
            P=intersectionpoint(G -- (B+(G-B)*0.01), circumcircle(A, B, EE));
            Q=intersectionpoint(C -- (D+(C-D)*0.01), circumcircle(A, D, EE));
            T=extension(B, Q, D, P);
            M=extension(F, P, G, Q);

            draw(B -- F -- C, pri); draw(C -- G -- D, pri); draw(A -- C, pri); draw(B -- D, pri); filldraw(circumcircle(A, B, C), fil, pri);
            filldraw(circumcircle(A, B, EE), tfil, tri); filldraw(circumcircle(A, D, EE), tfil, tri);

            draw(F -- P, tri); draw(G -- Q, tri);
            draw(B -- T -- P, sec); draw(P -- Q, sec); filldraw(circumcircle(B, D, P), sfil, sec);
            draw(T -- M, cyan);

            dot("$A$", A, dir(155));
            dot("$B$", B, NE);
            dot("$C$", C, SE);
            dot("$D$", D, SW);
            dot("$E$", EE, dir(120));
            dot("$F$", F, SW);
            dot("$G$", G, N);
            dot("$P$", P, NE);
            dot("$Q$", Q, SE);
            dot("$T$", T, S);
            dot("$M$", M, dir(350));
        \end{asy}
    \end{center}
    By Power of a Point, $CB\cdot CP=CA\cdot CE=CD\cdot CQ$, so $BDQP$ is cyclic. Furthermore, \[\measuredangle AEP=\measuredangle ABP=\measuredangle ABC=\measuredangle ADC=\measuredangle ADQ=\measuredangle AEQ,\]
    whence $E\in\overline{PQ}$. Then, $A$ is the Miquel Point of $BDQP$. If $T=\overline{BQ}\cap\overline{DP}$, then it is well-known that $\angle TAC=90^\circ$. However, by Pappus' Theorem on $\overline{BPG}$ and $\overline{DQF}$, $M\in\overline{AT}$, so $\measuredangle MAC=\measuredangle TAC=90^\circ$, and we are done. 
\end{customenv}
\begin{customenv}{Second solution, by harmonic bundles}
    First since \[\da BAC=\da BDC=\da EDQ=\da EAQ=\da CAQ,\]
    $\seg{AC}$ bisects $\angle BAQ$ and similarly $\angle DAP$. Let their external angle bisectors intersect $\seg{DC}$ and $\seg{BC}$ at $X$ and $Y$ respectively. Since $-1=(CX;FQ)=(CY;PG)$, lines $XY$, $FP$, $GQ$ concur. But $\seg{FP}$ and $\seg{GQ}$ intersect at $M$, so $\seg{AM}\perp\seg{AC}$, as desired.
\end{customenv}

