% Input your problem and solution below.
% Three dashes on a newline indicate the breaking points.

---

For each positive integer $n$, the Bank of Cape Town issues coins of denomination $\frac1n$. Given a finite collection of such coins (of not necessarily different denominations) with total value at most most $99+\frac12$, prove that it is possible to split this collection into $100$ or fewer groups, such that each group has total value at most $1$.
---

We prove that if we have a finite collection of coins with total value at most $N-1/2$, then it is possible to split the coins in to $N$ or fewer groups. For any subset of coins with sum $1$, put them in a group and forget about them. Henceforth assume no subset of the coins sum to $1$.

Let $\chi(n)=n\cdot 2^{-\nu_2(n)}$ be the greatest odd divisor of $n$. For all coins $1/m$ with $\chi(m)\le 2N-1$, put $1/m$ in group $\tfrac12(\chi(m)+1)$. We will deal with the other coins later.
\begin{iclaim*}
    The sum of the coins in any group is less than $1$.
\end{iclaim*}
\begin{proof}
    Assume for the sake of contradiction the coins sum to at least $1$; we show there is a subset of the coins with sum exactly $1$, proving the claim. Combine any two identical coins of the form $\tfrac1{2^ak}$, with $a>1$, until this operation can no longer be done. Then for all $a\ge 1$, there is at most one coin of this form, and thus these coins sum to less than $\sum_{a=1}^\infty\frac1{2^ak}=\frac1k$. Hence for the sum to be at least $1$, there must be a coin of value $1$, contradiction.
\end{proof}

Finally iterate through all remaining coins and find a suitable home for them. If any coin $\tfrac1j$, $j>2N-1$ is left out, then the total value of the coins we have is greater than \[N\left(1-\frac1j\right)+\frac1j=N-\frac{N-1}j>N-\frac12,\]
absurd.
