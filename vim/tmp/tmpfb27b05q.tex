% Input your problem and solution below.
% Three dashes on a newline indicate the breaking points.

---

Let $a$ be the smallest real number such that the roots of the polynomial \[P(x)=ax^3-x^2-(a^2+1)x+a^2-1\]
are all real. Then, the largest of these roots can be expressed in the form $\tfrac mn$, where $m$ and $n$ are relatively prime positive integers. Find $m+n$.

---

We know that \[0=-P(x)=(x-1)a^2-x^3a+(x^2+x+1).\]
Applying the quadratic formula with respect to $a$, we find that \[a=\frac{x^3\pm\sqrt{x^6-4(x-1)(x^2+x+1)}}{2(x-1)}=\frac{x^3\pm\sqrt{x^6-4x^3+4}}{2(x-1)}=\frac{x^3\pm(x^3-2)}{2(x-1)}.\]
Hence, $a=\frac1{x-1}$ or $a=x^2+x+1$. However, $a=x^2+x+1$ only has roots for $x$ if $a\ge\tfrac34$, so the smallest such $a$ is $\tfrac34$. Solving for $x$, we find that $x\in\{\frac73,\;-\frac12\}$, so the largest root of $P$ is $\tfrac73$, and the requested sum is $7+3=10$.
\begin{boxremark}
    The values of $a$ in terms of $x$ do in fact imply that our polynomial can be factored as \[P(x)=\big(ax-(a+1)\big)\big(x^2+x-(a-1)\big).\]
\end{boxremark}

---

010
