% Input your problem and solution below.
% Three dashes on a newline indicate the breaking points.

---

Let $m\ne0$ be an integer. Find all polynomials $P(x)$ with real coefficients such that
\[(x^3-mx^2+1)P(x+1)+(x^3+mx^2+1)P(x-1)=2(x^3-mx+1)P(x)\]
for all real number $x$.

---

The linear $P$ that work are $P(x)\equiv cx$. We will show these are the only solutions.

We can easily verify the following via substitution:
\begin{itemize}[itemsep=0em]
    \item for any root $r$ of $x^3-mx^2+1$, we have $rP(r-1)=(r-1)P(r)$; and
    \item for any root $r$ of $x^3+mx^2+1$, we have $rP(r+1)=(r+1)P(r)$.
\end{itemize}
Hence we consider the polynomial
\[Q(x):=(x-1)P(x)-xP(x-1),\]
with the properties that
\begin{itemize}[itemsep=0em]
    \item $Q(x+1)+Q(x)=P(x+1)-2P(x)-xP(x-1)$ and
    \item $Q(x+1)-Q(x)=x(P(x+1)-2P(x)+P(x-1))$.
\end{itemize}

After some brief computation, the given functional equation rewrites as
\begin{align*}
    mx\big(xP(x+1)-2P(x)-xP(x-1)\big)&=\left(x^3+1\right)\big(P(x+1)-2P(x)+P(x-1)\big)\\
    mx\cdot\big(Q(x+1)+Q(x)\big)&=\left(x^3+1\right)\cdot\big(Q(x+1)-Q(x)\big)\\
    Q(x+1)\cdot\left(x^3-mx^2+1\right)&=Q(x)\cdot\left(x^3+mx^2+1\right).\tag{$*$}
\end{align*}
\setcounter{claim}0
\begin{claim}
    There is a root $r$ of $x^3-mx^2+1$ such that no element of $r+\mathbb Z$ is a root of $x^3+mx^2+1$.
\end{claim}
\begin{proof}
    For $m=2$, take $r=1$; then $x^3+2x^2+1$ has no integer root, as needed.

    Otherwise, by rational root theorem, $x^3-mx^2+1$ will have three irrational roots $r$, $s$, $t$; in particular, $r$, $s$, $t$ are Galois conjugates. Then for any integer $a$, the numbers $r+a$, $s+a$, $t+a$ are also Galois conjugates, so is $r+a$ is a root of $x^3+mx^2+1$, then so are $s+a$, $t+a$.

    It follows that
    \begin{align*}
        x^3-mx^2+1&=(x-r)(x-s)(x-t)\\
        x^3+mx^2+1&=(x-r-a)(x-s-a)(x-t-a).
    \end{align*}
    By comparing the quadratic terms, we have $r+s+t=m$ and $r+s+t+3a=-m$, so $3a=-2m$. But by comparing the constant terms, we have
    \[-1=(r+a)(s+a)(t+a)=-P(-a)=a^3+ma^2-1,\]
    so $a=0$ or $a=m$. Either way, combining with $3a=-2m$ gives $a=m=0$, contradiction.
\end{proof}
\begin{claim}
    $Q\equiv0$.
\end{claim}
\begin{proof}
    Assume for contradiction $Q\not\equiv0$, i.e.\ $Q$ has finitely many roots. Take $s$ a root of $Q(x)$. Then
    \begin{itemize}
        \item I claim there exists $a\in\mathbb Z$ such that $s+a$ is a root of $x^3-mx^2+1$; otherwise, we will show by induction on $i\ge0$ that $Q(s+i)=0$. Indeed, if $Q(s+i)=0$, then since $s+i$ is not a root of $x^3-mx^2+1$, by $x=s+i$ in $(*)$, we have $Q(s+i+1)=0$ as well.
        \item I claim there exists $b\in\mathbb Z$ such that $s-b$ is a root of $x^3+mx^2+1$; otherwise, we will show by induction on $i\ge0$ that $Q(s-i)=0$. Indeed, if $Q(s-i)=0$, then since $s-i$ is not a root of $x^3+mx^2+1$, by $x=s-i-1$ in $(*)$, we have $Q(s-i-1)=0$ as well.
    \end{itemize}
    The above two conditions in tandem contradict the first claim.
\end{proof}

Therefore, $(x-1)P(x)\equiv xP(x-1)$; that is,
\[\frac{P(x)}x=\frac{P(x-1)}{x-1}\]
for all $x\notin\{0,1\}$, so $P$ is of the form $P(x)\equiv cx$ for some $c$.
