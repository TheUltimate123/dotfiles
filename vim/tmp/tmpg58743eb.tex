% Input your problem and solution below.
% Three dashes on a newline indicate the breaking points.
% vim: tw=72

---

Let $ABC$ be an isosceles triangle with $AB=AC$. Suppose that the center of circle $\omega$ is the midpoint of the $\overline{BC}$, and $\overline{AB}$ and $\overline{AC}$ are tangent to $\omega$ at points $E$ and $F$ respectively. There is a point $G$ that lies on $\omega$ such that $\angle AGE=90^\circ$. Show that if the tangent to $\omega$ at $G$ meets $\overline{AC}$ at $K$, then line $BK$ bisects $\overline{EF}$.

---

\begin{center}
    \begin{asy}
        size(8cm);
        defaultpen(fontsize(10pt));

        pen pri=red;
        pen sec=orange;
        pen fil=pri+opacity(0.05);
        pen sfil=sec+opacity(0.05);

        pair A, B, C, O, EE, F, G, K, Ep, T;
        A=dir(90);
        B=dir(200);
        C=dir(340);
        O=(B+C)/2;
        EE=foot(O,A,B);
        F=foot(O,A,C);
        G=foot(EE,A,2O-EE);
        K=extension(A,C,G,G+rotate(90)*(G-O));
        Ep=2O-EE;
        T=2*foot(O,Ep,K)-Ep;

        filldraw(A--B--C--cycle,fil,pri);
        filldraw(circle(O,length(F-O)),sfil,sec);
        draw(A--Ep,pri);
        draw(Ep--G--K,pri);
        draw(EE--F,pri);
        draw(B--K,pri);
        draw(C--Ep--F,pri);
        draw(G--EE--O--F,sec);
        draw(O--Ep--K,sec);

        dot("$A$",A,N);
        dot("$B$",B,W);
        dot("$C$",C,E);
        dot("$O$",O,SW);
        dot("$E$",EE,NW);
        dot("$F$",F,NE);
        dot("$G$",G,NE);
        dot("$K$",K,NE);
        dot("$L$",Ep,SE);
        dot("$T$",T,SW);
        dot("$S$",(EE+F)/2,dir(120));
    \end{asy}
\end{center}
Let line $AG$ intersect $\omega$ again at $L$, $\overline{LK}$ intersect $\omega$ again at $T$, and $\overline{BK}$ intersect $\overline{EF}$ at $S$. Note that since $\angle EGL=90^\circ$, $L$ is the antipode of $E$ on $\omega$, so $F$ and $L$ are reflections over $\overline{BC}$. It follows that $\overline{CL}$ is tangent to $\omega$, whence \[-1=(GF;TL)\stackrel L=(AF;KC)\stackrel B=(EF;S\infty_{BC}).\]
This implies that $S$ is the midpoint of $\overline{EF}$, and we are done. 

