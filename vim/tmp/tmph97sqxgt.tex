% Input your problem and solution below.
% Three dashes on a newline indicate the breaking points.

---

Let $m>1$ be a fixed positive integer. For a nonempty string of base-ten digits $S$, let $c(S)$ be the number of ways to split $S$ into contiguous nonempty strings of digits such that the base-ten number represented by each string is divisible by $m$. (For example, if $m=2$, then $c(1234)=2$ due to $1234$ and $12|34$.) These strings are allowed to have leading zeroes. 

In terms of $m$, what are the possible values that $c(S)$ can take?

---

The answer is $0$ and powers of $2$. It is easy to achieve these: $c(1)=0$ and $c(0\cdots0)=2^n$ for $n+1$ zeros. Now we prove only these are attainable.

Let $(\|)$ denote concatenation. Note that if $m\nmid S$, then $c(S)=0$. It is easy to see $c(x\|y)=2c(x)c(y)$ for $m\mid x,y$, and if $S$ cannot be expressed as $x\|y$ for $m\mid x,y$, then either $m\nmid S$ or $c(S)=1$.

In all cases, $c(S)$ is either $0$ or a power of $2$.

