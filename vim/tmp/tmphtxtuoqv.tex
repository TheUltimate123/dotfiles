% Input your problem and solution below.
% Three dashes on a newline indicate the breaking points.

---

Solve over $\mathbb R$ the functional equation \[f(f(x)+y)^2=(x-y)(f(x)-f(y))+4f(x)f(y).\]

---

The answer is $f\equiv0$ and $f(x)=x$, which work. Henceforth assume $f$ is not constant. Otherwise $f\equiv0$.
\begin{iclaim*}
    $f$ is injective.
\end{iclaim*}
\begin{proof}
    Say $f(a)=f(b)$. Then 
    \begin{align*}
        f(f(a)+y)^2&=(a-y)(f(a)-f(y))+4f(a)f(y),\\
        f(f(b)+y)^2&=(b-y)(f(b)-f(y))+4f(b)f(y).
    \end{align*}
    Hence $(a-y)(f(a)-f(y))=(b-y)(f(b)-f(y))$. Take $y$ such that $f(y)\ne f(a)$, which exists by our assumption that $f$ is not constant, and conclude $a=b$.
\end{proof}

Apply the substitution $y=x-f(x)$ to obtain \[f(x)^2=f(x)\big(f(x)-f(x-f(x))\big)+4f(x)f(x-f(x))=f(x)^2+3f(x)f(x-f(x)),\]
whence $f(x)=0$ or $f(x-f(x))=0$ for all $x$. By injectivity $f(x)=x$ for all $x$, as desired.
