% Input your problem and solution below.
% Three dashes on a newline indicate the breaking points.

---

Let $ABC$ be a triangle with incenter $I$, and let $D$ be a point on line $BC$ satisfying $\angle AID=90^\circ$. Let the excircle of triangle $ABC$ opposite the vertex $A$ be tangent to $\overline{BC}$ at $A_1$. Define points $B_1$ on $\overline{CA}$ and $C_1$ on $\overline{AB}$ analogously, using the excircles opposite $B$ and $C$, respectively.

Prove that if quadrilateral $AB_1A_1C_1$ is cyclic, then $\overline{AD}$ is tangent to the circumcircle of $\triangle DB_1C_1$.

---

\begin{customenv}{First solution, by angle chasing}
    Let $V$ be the antipode of $A$ on $(AB_1C_1)$. Let the incircle of $\triangle ABC$ touch $\overline{CA}$ and $\overline{AB}$ at $E$ and $F$, respectively, and let $J$ be the Miquel point of $BCEF$. Furthermore, let $M$ be the midpoint of $\overline{AD}$.

    We claim that $\overline{AA_1}$ is a diamter of $(AB_1A_1C_1)$. Note that $V$ is the Bevan point of $\triangle ABC$, so $\overline{VA_1}\perp\overline{BC}$. Furthermore, if $V\ne A_1$, then $\overline{VA_1}\perp\overline{AA_1}$, which would require that $A\in\overline{BC}$, which is absurd.
    \begin{center}
        \begin{asy}
            size(12cm);
            defaultpen(fontsize(10pt));

            pen pri=royalblue+linewidth(0.5);
            pen sec=Cyan+linewidth(0.5);
            pen tri=springgreen+linewidth(0.5);
            pen qua=chartreuse+linewidth(0.5);
            pen fil=pri+opacity(0.05);
            pen sfil=sec+opacity(0.05);
            pen tfil=tri+opacity(0.05);
            pen qfil=qua+opacity(0.05);

            pair B, C, L, I, A, A1, B1, C1, EE, F, J, D, M, K, IB, IC;
            B=dir(190); C=dir(350); L=dir(270);
            I=intersectionpoint(arc(L, length(B-L), 90, 180, CCW), (-B) -- (-C));
            A=intersectionpoint(I -- (I+100*(I-L)), circle((0, 0), 1));
            A1=B+C-foot(I, B, C);
            B1=foot(A1, A, C);
            C1=foot(A1, A, B);
            EE=foot(I, A, C);
            F=foot(I, A, B);
            J=intersectionpoints(circumcircle(A, B, C), circumcircle(A, EE, F))[1];
            D=extension(A, J, B, C);
            M=(A+D)/2;
            K=dir(90);
            IB=extension(B, I, A, K);
            IC=extension(C, I, A, K);

            draw(A -- B -- C -- A, pri);
            filldraw(circumcircle(A, B, C), fil, pri);
            draw(IB -- A1 -- IC -- IB, sec);
            filldraw(circumcircle(A, B1, C1), tfil, tri);
            draw(B -- D -- A, pri);
            filldraw(circumcircle(A, EE, F), sfil, sec);
            filldraw(circumcircle(B, I, C), qfil, qua);
            draw(B -- IB, tri+dashed); draw(C -- IC, tri+dashed);
            draw(M -- B1, pri);
            draw(D -- I -- J, sec); draw(A -- L, pri);

            clip((L + (100, -1/8)) -- (L+(-100, -1/8)) -- (-100, 100) -- (100, 100) -- cycle);

            dot("$A$", A, NW);
            dot("$B$", B, SW);
            dot("$C$", C, SE);
            dot("$L$", L, S);
            dot("$I$", I, NE);
            dot("$A_1$", A1, SE);
            dot("$B_1$", B1, NE);
            dot("$C_1$", C1, SW);
            dot("$E$", EE, NE);
            dot("$F$", F, NW);
            dot("$J$", J, dir(150));
            dot("$D$", D, SW);
            dot("$M$", M, dir(195));
            dot("$K$", K, N);
            dot("$I_B$", IB, NE);
            dot("$I_C$", IC, W);
        \end{asy}
    \end{center}
    By Pappus' Theorem on $\overline{BA_1C}$ and $\overline{I_CAI_B}$, $I$ lies on $\overline{B_1C_1}$, and by the Radical Axis Theorem on $(AI)$, $(ABC)$, and $(BIC)$, $J$ lies on $\overline{AD}$. Since $\triangle JBF\sim\triangle JCE$, \[\frac{AC_1}{AB_1}=\frac{JF}{JE}=\frac{JB}{JC},\]
    and also $\measuredangle C_1AB_1=\measuredangle BAC=\measuredangle BJC$, so $\triangle AC_1B_1\sim\triangle JBC$. This implies that \[\measuredangle DAB=\measuredangle JAF=\measuredangle JEF=\measuredangle JCB=\measuredangle AB_1C_1,\]
    so $\overline{AD}$ is tangent to $(AB_1C_1)$. Moreover, \[\measuredangle MIA=\measuredangle IAM=\measuredangle IAC_1+\measuredangle C_1AM=\measuredangle B_1AI+\measuredangle C_1B_1A=\measuredangle B_1IA,\]
    whence $M$ lies on $\overline{B_1IC_1}$. Hence, $MD^2=MA^2=MB_1\cdot MC_1$, and the desired result follows. 
\end{customenv}
\begin{customenv}{Second solution, by harmonic bundles}
    Like above, check that $\overline{AA_1}$ is a diameter of $(AB_1C_1)$ and that $I$ lies on $\overline{B_1C_1}$. Denote by $X$ the foot of the altitude from $A$. Notice that \[-1=(A,\overline{BC}\cap\overline{I_BI_C};I_B,I_C)\stackrel{A_1}=(AX;B_1C_1).\]
    Let the tangents to $(AB_1C_1)$ at $A$ and $X$ meet at $M$. Since $I$ is the foot of the $A$-angle bisector of $\triangle AB_1C_1$ and $AB_1XC_1$ is harmonic, $(AIX)$ is the $A$-Apollonius circle of $\overline{B_1C_1}$.

    Since $\angle DIA=90^\circ=\angle DXA$, $AIXD$ is cyclic. However $M$ is the circumcenter of $\triangle AIX$, so $M$ is the midpoint of $\overline{AD}$, and $MD^2=MA^2=MB_1\cdot MC_1$. This completes the proof. 
\end{customenv}
