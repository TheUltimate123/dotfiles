% Input your problem and solution below.
% Three dashes on a newline indicate the breaking points.

---

Given an acute-angled triangle $ABC$ with orthocenter $H$, the reflection of the nine-point circle about $AH$ intersects circumcircle at points $X$ and $Y$. Prove that $\seg{AH}$ is the external bisector of $\angle XHY$.

---

\begin{center}
    \begin{asy}
        /* Geogebra to Asymptote conversion, documentation at artofproblemsolving.com/Wiki go to User:Azjps/geogebra */
        size(8cm); defaultpen(fontsize(10pt));
        real labelscalefactor = 0.5; /* changes label-to-point distance */
        pen rvwvcq = rgb(0.08235294117647059,0.396078431372549,0.7529411764705882); pen wrwrwr = rgb(0.3803921568627451,0.3803921568627451,0.3803921568627451);

        draw((-2.7610213408027877,5.933356659746497)--(-5.28,1.09)--(2.46,1.09)--cycle, rvwvcq);
        /* draw figures */
        draw((-2.7610213408027877,5.933356659746497)--(-5.28,1.09), rvwvcq);
        draw((-5.28,1.09)--(2.46,1.09), rvwvcq);
        draw((2.46,1.09)--(-2.7610213408027877,5.933356659746497), rvwvcq);
        draw(circle((-1.41,2.15397929378948), 4.0135958861864465), wrwrwr);
        draw(circle((-2.085510670401394,2.979688682978509), 2.0067979430932237), wrwrwr);
        draw(circle((-4.112042681605576,2.15397929378948), 4.0135958861864465), wrwrwr);
        draw(circle((-3.4365320112041817,2.979688682978509), 2.0067979430932237), wrwrwr);
        draw((-4.5578028381004945,4.64401729150342)--(-0.12037089975989268,2.5729164294431177), wrwrwr);
        draw((-5.4016717818456845,2.572916429443112)--(-0.9642398435050779,4.644017291503414), wrwrwr);
        draw(circle((-2.3842613779589934,3.5116783298732486), 2.45081088682496), linetype("2 2") + wrwrwr);
        draw(circle((-3.1377813036465847,3.5116783298732486), 2.4508108868249603), linetype("2 2") + wrwrwr);
        draw((-2.7610213408027877,5.933356659746497)--(-2.7610213408027877,-1.6253980721675363), wrwrwr);
        /* dots and labels */
        dot("$A$",(-2.7610213408027877,5.933356659746497),N);
        dot("$B$",(-5.28,1.09),SW);
        dot("$C$",(2.46,1.09),SE);
        dot("$H$",(-2.7610213408027877,3.8053980721675362),dir(300));
        dot("$D$",(-2.7610213408027877,1.09),dir(-60));
        dot("$Y'$",(-0.9642398435050779,4.644017291503414),NE);
        dot("$X'$",(-0.12037089975989268,2.5729164294431177),SE);
        dot("$X$",(-4.5578028381004945,4.64401729150342),NW);
        dot("$Y$",(-5.4016717818456845,2.572916429443112),W);
        dot("$H_A$",(-2.7610213408027877,-1.6253980721675363),S);
        /* end of picture */
    \end{asy}
\end{center}
Let $\Gamma$ be the circumcircle, $\omega$ the nine-point circle, and $\Gamma'$ and $\omega'$ their respective reflections across $\seg{AH}$. Also let $\Psi$ denote negative inversion at $H$ with radius $\sqrt{AH\cdot HD}$.

Note that:
\begin{itemize}[itemsep=0em]
    \item $\Psi$ swaps $\Gamma$ and $\omega$;
    \item $\Psi$ swaps $\Gamma'$ and $\omega'$.
\end{itemize}
Let $\Psi$ send $X$ and $Y$ to $X'$ and $Y'$, respectively. Then $X'$ and $Y'$ are the intersections of $\Gamma'$ and $\omega$. Thus $X$ and $Y$ are the reflections of $Y'$ and $X'$ across $\seg{AH}$, so $XYX'Y'$ is an isosceles trapezoid and $\seg{AH}$ bisects $\angle XHY$. End proof.

