% Input your problem and solution below.
% Three dashes on a newline indicate the breaking points.

---

Let $\pi(n)$ denote the number of prime numbers at most $n$. Prove that for all positive integers $m$ and $n$, the following inequality holds:
\[\pi(m)-\pi(n)\le\frac{(m-1)\varphi(n)}n.\]
When does equality hold?

---

If $m\le n$, the inequality obviously holds, with equality if and only if $(m,n)=(1,1)$.

Henceforth $m>n$. We first verify the bound. The goal is to show there are at most $\frac{m-1}n\vphi(n)$ primes in $\{n+1,\ldots,m\}$. Let $f_n(S)$ denote the number of elements of $S$ relatively prime to $n$. I will prove the stronger inequality
\[f(\{n+1,\ldots,M\})\le\frac{m-1}n\vphi(n).\]

Let $K=\left\lceil\frac{m-n}n\right\rceil\cdot n$, so $K\le m-1$; then 
\begin{align*}
    f(\{n+1,\ldots,M\})&=f(\{1,\ldots,m-n\})\le f(\{1,\ldots,K\})\\
    &=\left\lceil\frac{m-n}n\right\rceil\cdot\vphi(n)\le\frac{m-1}n\vphi(n),
\end{align*}
as claimed.

Now for the equality bound to hold:
\begin{itemize}[itemsep=0em]
    \item all elements of $\{n+1,\ldots,m\}$ must be prime;
    \item no elements of $\{m-n+1,\ldots,K\}$ are relatively prime to $n$; and
    \item $\left\lceil\frac{m-n}n\right\rceil=\frac{m-1}n$, i.e.\ $m\equiv1\pmod n$ and $K=m-1$.
\end{itemize}
If $n\ge3$, for the second condition to hold we must have $\gcd(K-1,n)\ne1$, which is absurd; therefore $n\le2$.
\begin{itemize}
    \item For $n=1$, everything in $\{2,\ldots,m\}$ must be prime, so $m\le3$. Both $(2,1)$ and $(3,1)$ work.
    \item For $n=2$, all odds in $\{3,\ldots,m\}$ need to be prime, so $m\le7$. We can verify $(3,2)$, $(5,2)$, $(7,2)$ all work.
\end{itemize}
Therefore the solution set is $(1,1)$, $(2,1)$, $(3,1)$, $(3,2)$, $(5,2)$, $(7,2)$.
