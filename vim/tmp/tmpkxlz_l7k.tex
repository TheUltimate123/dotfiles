% Input your problem and solution below.
% Three dashes on a newline indicate the breaking points.

---

Let $ABC$ be a triangle with circumcircle $\omega$ and circumcenter $O$. Suppose that $AB=30$, $AC=28$, and $P$ is a point in the interior of $\triangle ABC$ such that $AP=13$, $BP^2=409$, and $P$ is closer to $\overline{AC}$ than $\overline{AB}$. Let $E$, $F$ be the points where $\overline{BP}$, $\overline{CP}$ intersect $\omega$ again, and let $Q$ be the intersection of $\overline{EF}$ and the tangent to $\omega$ at $A$. Given that $AQOP$ is cyclic and that $CP^2$ is expressible in the form $a-b\sqrt c$ for positive integers $a$, $b$, $c$ where $c$ is squarefree, find $a+b+c$.

---

\begin{center}
    \begin{asy}
        size(7cm); defaultpen(fontsize(10pt));

        pair O,A,B,C,Pp,P,EE,F,Q,D,X,Y,Z,Xp,Ap;
        O=(0,0);
        A=dir(110);
        B=dir(210);
        C=dir(330);
        real t=37;
        Pp=extension(B,B+rotate(-t)*(B-A),C,C+rotate(t)*(C-A));
        P=(A+Pp)/2;
        EE=2*foot(O,B,P)-B;
        F=2*foot(O,C,P)-C;
        Q=extension(EE,F,A,A+rotate(90)*(O-A));
        D=2*foot(O,A,P)-A;
        X=reflect(O,Q)*A;
        Y=2*foot(O,B,Pp)-B;
        Z=2*foot(O,C,Pp)-C;
        Xp=2*foot(O,P,X)-X;
        Ap=-A;

        draw(EE--B--Y,gray);
        draw(F--C--Z,gray);
        draw(F--Q,gray+dashed);
        draw(X--Xp,gray+dashed);
        draw(A--Xp--Ap,gray+Dotted);
        draw(circle(O,1));
        draw(A--B--C--A);
        draw(circumcircle(A,P,Q),dashed);
        draw(D--A--Q);

        dot("$O$",O,S);
        dot("$A$",A,NW);
        dot("$B$",B,B);
        dot("$C$",C,C);
        dot("$P'$",Pp,dir(-60));
        dot("$P$",P,dir(195));
        dot("$E$",EE,dir(-75));
        dot("$F$",F,F);
        dot("$Q$",Q,N);
        dot("$D$",D,D);
        dot("$X$",X,E);
        dot("$Y$",Y,N);
        dot("$Z$",Z,W);
        dot("$X'$",Xp,Xp);
        dot("$A'$",Ap,Ap);
    \end{asy}
\end{center}
Let $P'$ be the reflection of $A$ over $P$, and let $(APOQ)$ intersect $\omega$ again at $X$. Also let $\seg{AP}$, $\seg{BP'}$, $\seg{CP'}$, $\seg{XP}$ intersect $\omega$ again at $D$, $Y$, $Z$, $X'$. Finally let $A'$ be the antipode of $A$ on $\omega$.

The key claim is this:
\begin{iclaim*}
    $\angle ABP'=\angle ACP'$.
\end{iclaim*}
\begin{proof}
    It will suffice to prove that $AYA'Z$ is a kite, i.e.\ $-1=(AA';YZ)$.

    Several observations to note:
    \begin{itemize}[itemsep=0em]
        \item $\angle OXQ=\angle OAQ=90\dg$, so $\seg{QX}$ is also tangent to $(ABC)$.
        \item Since $OA=OX$, we have $\da OPD=\da OPA=\da XPO$, thus $PD=PX$.
        \item Since $PD=PX$, we have $PX'=PA=PP'$, so $\angle AXP'=90\dg$ and $X'$, $P'$, $A'$ collinear.
    \end{itemize}
    From this, we have \[-1=(AX;EF)\stackrel P=(DX';BC)\stackrel{P'}=(AA';YZ),\]
    as claimed.
\end{proof}

The rest is routine computation: in fact $\angle ACP'=\angle ABP'=60\dg$ by median formula and Law of Cosines, and so $CP=365-28\sqrt{22}$, and the requested sum is $365+28+22=415$.

---

415
