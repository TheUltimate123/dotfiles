% Input your problem and solution below.
% Three dashes on a newline indicate the breaking points.

---

Let $\mathbb N$ be the set of all positive integers. A subset $A$ of $\mathbb N$ is \emph{sum-free} if, whenever $x$ and $y$ are (not necessarily distinct) members of $A$, their sum $x+y$ does not belong to $A$. Determine all surjective functions $f:\mathbb N\to\mathbb N$ such that, for each sum-free subset $A$ of $\mathbb N$, the image $\{f(a):a\in A\}$ is also sum-free.

---

The only solution is the obvious one --- $f(n)\equiv n$.
\setcounter{claim}0
\begin{claim}
    $f(2a)=2f(a)$.
\end{claim}
\begin{proof}
    By surjectivity there is an infinite sequence $a_0$, $a_1$, $\ldots$ such that $f(a_i)=2^if(a)$ for each $i$. Evidently the elements of $(a_i)$ are all distinct. Observe that $\{f(a_i),f(a_{i+1})\}$ is never sum-free, so $\{a_i,a_{i+1}\}$ is not sum-free and $a_{i+1}\in\{2a_i,a_i/2\}$ for each $i$.

    Since the sequence is injective, either $a_{i+1}=2a_i$ for all $i$ or $a_{i+1}=a_i/2$ for all $i$; the latter is impossible.
\end{proof}
\begin{claim}
    $f$ is bijective.
\end{claim}
\begin{proof}
    Say $f(a)=f(b)$ but $a\ne b$. Then $\{a,2b\}$ is sum-free, but $\{f(a),f(2b)\}$ is not sum-free.
\end{proof}

Let $g=f^{-1}$, so $g$ is also bijective, $g(2a)=2g(a)$ for all $a$, and $g$ maps not sum-free sets to not sum-free sets.

Let $g(1)=c$, so $g(2^i)=2^ic$. By $\{1,3,4\}$, $\{1,4,5\}$, we deduce $\{g(3),g(5)\}=\{3c,5c\}$, and by $\{1,5,6\}$, we deduce $g(3)=3c$, $g(5)=5c$.

Now we prove by induction on $n$ that $g(n)\equiv cn$. The base cases $n=1,\ldots,6$ have already been shown. For the inductive step, say $f(k)=k$ for all $k<n$. By $\{1,n-1,n\}$, we have $f(n)\in\{nc,(n-2)c\}$, but the latter case is impossible since $f(n-2)=(n-2)c$ already.

Finally $c=1$ since $g$ is bijective, so we are done.

