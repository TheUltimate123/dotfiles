% Input your problem and solution below.
% Three dashes on a newline indicate the breaking points.

---

A necklace of red, white, and blue beads is given. A set of beads is \emph{American} if it contains an equal number of beads of each color, and a pair of adjacent beads is \emph{dazzling} if they are of different colors.

Suppose that the entire necklace is American and the number of dazzling pairs is even. Prove that the necklace can be cut into two American pieces, each with at least one bead.

---

Suppose the necklace $\mathcal P$ is American, but cannot be cut into two American pieces. We will show the number of dazzling pairs is odd.

Let $\mathbf u$, $\mathbf v$, $\mathbf w$ be unit vectors with arguments $0\dg$, $120\dg$, $240\dg$ respectively. Construct a polyline as follows: go around $\mathcal P$ and increment position by $\mathbf u$ each time a red bead is encountered, by $\mathbf v$ each time a white bead is encountered, and by $\mathbf w$ each time a blue bead is encountered.
\begin{center}
    \begin{asy}
        size(4cm); dotfactor *= 1.5;
        int[] a = {0, 0, 0, 1, 0, 0, 1, 1, 2, 1, 2, 2, 1, 1, 2, 2, 0, 2};
        int n = a.length;
        a.cyclic = true;
        pair[] p = {(1, 0), (-1/2, sqrt(3)/2), (-1/2, -sqrt(3)/2)};
        pen[] col = {red, gray(0.4), blue};

        pair fi = (0, 0);
        pair se = (0, 0);

        string[] ch = {"{\textcolor{red}R}", "{W}", "{\textcolor{blue}B}"};
        string s = "";

        pair o = (-4, 1);

        for (int i = 0; i < n; ++i) {
            fi = se;
            se = fi + p[a[i]];
            draw(fi--se, col[a[i]], MidArrow);
            s += ch[a[i]];
            if (a[i] != a[i+1]) {
                dot(se);
            }
        }

        for (int i = 0; i < n; ++i) {
            fi = se;
            se = fi + p[a[i]];
            if (a[i] != a[i+1]) {
                dot(se);
            }
        }

        //for (int i = 0; i < 3; ++i) {
            //	draw(o--(o+p[i]), col[i], MidArrow);
        //}

        label(s, (2, 3.5));
    \end{asy}
\end{center}
Since $\mathbf u+\mathbf v+\mathbf w=0$, the polyline is closed, and since the necklace cannot be split into American pieces, the polyline borders a non-self-intersecting polygon $\mathcal Q$. Each angle $\mathcal Q$ is either $60\dg$ or $300\dg$. Let $a$ be the number of $60\dg$ angles and $b$ the number of $300\dg$ angles, so $n=a+b$ is the number of dazzling pairs of beads.

The sum of the angles of $\mathcal Q$ is given by $60\dg a+300\dg b=180\dg(a+b-2)$, so $a-b=3$. Thus $n$ is odd, as desired.

