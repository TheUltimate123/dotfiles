% Input your problem and solution below.
% Three dashes on a newline indicate the breaking points.

---

Is there a function $f:\mathbb R\to\mathbb R$ such that $f(f(x))=x^2-2$ for all real numbers $x$?

---

No such function exists. Say a function $g:\mathbb R\to\mathbb R$ is \emph{good} if $g$ has exactly two fixed points $\{a,b\}$ and $g\circ g$ has exactly four fixed points $\{a,b,c,d\}$.
\begin{boxlemma*}[Generalization of problem]
    If $g$ is good, then there is no function $f$ such that $f\circ f=g$.
\end{boxlemma*}
\begin{proof}
    Assume the contrary. As we did above, let the fixed points of $g$ be $\{a,b\}$ and the fixed points of $g\circ g$ be $\{a,b,c,d\}$.
    \begin{itemize}
        \item Let $x=g(c)$; then $g(x)=g(g(c))=c$ and $g(g(x))=g(c)=x$, so $x\in\{c,d\}$. But if $x=c$, then $g(c)=c$, so $c\in\{a,b\}$, absurd. Hence $g(c)=d$ and $g(d)=c$.
        \item Next note that $f(g(x))=f(f(f(x)))=g(f(x))$. Hence $f(a)=f(g(a))=g(f(a))$, so $f(a)\in\{a,b\}$. By the same argument, we have $\{f(a),f(b)\}=\{a,b\}$.
        \item Check that $f(c)=f(g(d))=g(f(d))=g(f(g(c))=g(g(f(c))$, so $f(c)\in\{a,b,c,d\}$. Similarly $f(d)\in\{a,b,c,d\}$.
    \end{itemize}

    Finally we exhaust the three possible cases for $f(c)$:
    \begin{itemize}[itemsep=0em]
        \item If $f(c)=a$, then $d=g(c)=f(f(c))=f(a)\in\{a,b\}$, absurd. Similarly $f(c)\ne a$.
        \item If $f(c)=c$, then $f(c)=f(f(c))=g(c)=d$, contradiction.
        \item If $f(c)=d$, then $c=g(d)=g(f(c))=f(g(c))=f(d)=f(f(c))=g(c)=d$, absurd.
    \end{itemize}
    Thus the lemma is proven.
\end{proof}

Finally I claim $g(x)=x^2-2$ is good. Indeed, $g(g(x))=x^4-4x^2+2$, so $\{a,b\}=\{2,-1\}$ and $\{c,d\}=\{\tfrac{-1+\sqrt5}2,\tfrac{-1-\sqrt5}2\}$. We are done by the lemma.

