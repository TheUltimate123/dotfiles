% Input your problem and solution below.
% Three dashes on a newline indicate the breaking points.

---

Let $k$ be a positive integer. Alex and Bob play a game on an infinite grid of regular hexagons. Initially all the grid cells are empty. Then the players alternately take turns with Alex moving first. In his move, Alex may choose two adjacent hexagons in the grid which are empty and place a counter in both of them. In his move, Bob may choose any counter on the board and remove it. If at any time there are $k$ consecutive grid cells in a line all of which contain a counter, Alex wins. Find the minimum value of $k$ for which Alex cannot win in a finite number of moves, or prove that no such minimum value exists.

---

The answer is $k=6$.

\bigskip

\textbf{Bob's strategy for $k=6$:} Consider the below ``honeycomb'' coloring.
Bob can ensure that at any point in time, at most one of the blue hexagons is colored; thus the longest line of labeled hexagons has length five.
\begin{center}
\begin{asy}
    size(5cm); defaultpen(fontsize(10pt));
    pen fil=lightblue+white+white+white;
    path hex(real x, real y) {
        return (x,y)--(x+1,y)--(x+3/2,y+sqrt(3)/2)--(x+1,y+sqrt(3))--(x,y+sqrt(3))--(x-1/2,y+sqrt(3)/2)--cycle;
    }
    fill(hex(3/2,-sqrt(3)/2),fil);
    fill(hex(9/2,-sqrt(3)/2),fil);
    fill(hex(15/2,-sqrt(3)/2),fil);
    fill(hex(0,sqrt(3)),fil);
    fill(hex(3,sqrt(3)),fil);
    fill(hex(6,sqrt(3)),fil);
    fill(hex(9,sqrt(3)),fil);
    fill(hex(3/2,5sqrt(3)/2),fil);
    fill(hex(9/2,5sqrt(3)/2),fil);
    fill(hex(15/2,5sqrt(3)/2),fil);
    for (real x=0; x<=9+1e-9; x+=3) {
        for (real y=0; y<=2sqrt(3); y+=sqrt(3)){
            draw(hex(x,y));
        }
    }
    for (real x=3/2; x<=10+1e-9; x+=3) {
        for (real y=-sqrt(3)/2; y<=5sqrt(3)/2; y+=sqrt(3)){
            draw(hex(x,y));
        }
    }
\end{asy}
\end{center}

\bigskip

\textbf{Alice's strategy for $k=5$:} On the contrary, consider arbitrarily-long two chains of long hexagons in a ``parallelogram'' formation as follows. Play strictly within the chain until it is no longer possible; that is, until there are no two adjacent uncovered squares.
\begin{center}
\begin{asy}
    size(6cm); defaultpen(fontsize(10pt));
    pen fil=lightred+white+white+white;
    path hex(real x, real y) {
        return (x,y)--(x+sqrt(3)/2,y+1/2)--(x+sqrt(3)/2,y+3/2)--(x,y+2)--(x-sqrt(3)/2,y+3/2)--(x-sqrt(3)/2,y+1/2)--cycle;
    }
    fill(hex(0,3/2),fil);
    fill(hex(sqrt(3),3/2),fil);
    fill(hex(3sqrt(3),3/2),fil);
    fill(hex(4sqrt(3),3/2),fil);
    fill(hex(5sqrt(3),3/2),fil);
    fill(hex(7sqrt(3),3/2),fil);
    fill(hex(8sqrt(3),3/2),fil);
    fill(hex(3sqrt(3)/2,0),fil);
    fill(hex(5sqrt(3)/2,0),fil);
    fill(hex(7sqrt(3)/2,0),fil);
    fill(hex(11sqrt(3)/2,0),fil);
    fill(hex(13sqrt(3)/2,0),fil);
    label("$x$",(2sqrt(3),5/2));
    label("$y$",(6sqrt(3),5/2));

    for (real x=0; x<=8sqrt(3); x+=sqrt(3)) {
        draw(hex(x,3/2));
    }
    for (real x=sqrt(3)/2; x<=8sqrt(3); x+=sqrt(3)) {
        draw(hex(x,0));
    }
\end{asy}
\end{center}
If we have not won already, then there are two uncovered hexagons $x$, $y$ in the top row with $k\le4$ covered hexagons between them.
\begin{itemize}
    \item If $k=4$, we just win by covering either $x$ or $y$.
    \item If $k\le3$, then consider the hexagons in the bottom row adjacent to $x$ and $y$. By hypothesis, all four are covered, and at most one of the $k-2\le2$ hexagons between them is uncovered. We can easily win by filling it.
\end{itemize}

