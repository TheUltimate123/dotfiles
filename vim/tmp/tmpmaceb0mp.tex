
% Input your problem and solution below.
% Three dashes on a newline indicate the breaking points.
% vim: tw=72

---

Let $ABC$ be an acute scalene triangle with centroid $G$ and circumcenter $O$. The reflections of $G$ and $O$ across $\overline{BC}$, $\overline{CA}$, and $\overline{AB}$ are denoted $G_1,\,G_2,\,G_3$ and $O_1,\,O_2,\,O_3$ respectively. Show that the circumcircles of triangles $AG_2G_3$, $BG_3G_1$, $CG_1G_2$, $AO_2O_3$, $BO_3O_1$, $CO_1O_2$, and $ABC$ share a common point.

---

Let $X=\overline{G_1O_1}\cap\overline{G_2O_2}\cap\overline{G_3O_3}$ be the anti-Steiner point of the Euler line $\overline{HGO}$. We claim that $X$ is the desired concurrence point.
\begin{center}
    \begin{asy}
        size(10cm);
        defaultpen(fontsize(10pt));

        pen pri=deepgreen;
        pen sec=royalblue;
        pen tri=green;
        pen qua=springgreen;
        pen qui=Cyan;
        pen fil=pri+opacity(0.05);
        pen sfil=sec+opacity(0.05);
        pen tfil=tri+opacity(0.05);
        pen qfil=qua+opacity(0.05);
        pen qifil=qui+opacity(0.05);

        pair A,B,C,G,O,GA,OA,GB,OB,GC,OC,X,XA,XB,XC;
        A=dir(125);
        B=dir(215);
        C=dir(325);
        G=(A+B+C)/2;
        O=(0,0);
        GA=2*foot(G,B,C)-G;
        GB=2*foot(G,C,A)-G;
        GC=2*foot(G,A,B)-G;
        OA=2*foot(O,B,C)-O;
        OB=2*foot(O,C,A)-O;
        OC=2*foot(O,A,B)-O;
        X=extension(GB,OB,GC,OC);
        XA=2*foot(X,B,C)-X;
        XB=2*foot(X,C,A)-X;
        XC=2*foot(X,A,B)-X;

        filldraw(circumcircle(A,B,C),fil,pri);
        filldraw(circumcircle(A,GB,GC),tfil,tri);
        filldraw(circumcircle(A,OB,OC),qfil,qua);
        draw(B--C,pri);
        draw(extension(X,XC,A,B)--A--extension(GA,OA,A,C));
        draw(XB--extension(GA,OA,B,C)--XC,sec);
        draw(extension(OA,GA,B,C)--extension(A,C,OA,GA),sec);
        draw(OC--X--GB,sec);
        draw(XA--X,qui+dashed);
        draw(XB--X--XC,qui+dashed);
        draw(GB--G--GC,tri+dashed);
        draw(OB--O--OC,qua+dashed);
        draw(GA--G,tri+dashed);
        draw(OA--O,qua+dashed);

        dot("$A$",A,N);
        dot("$B$",B,dir(185));
        dot("$C$",C,dir(5));
        dot("$G$",G,N);
        dot("$O$",O,N);
        dot("$X$",X,S);
        dot("$G_B$",GB,NE);
        dot("$G_C$",GC,dir(190));
        dot("$O_B$",OB,NW);
        dot("$O_C$",OC,dir(160));

        clip((-100,-1.2)--(100,-1.2)--(100,100)--(-100,100)-- cycle);
        clip((XB+(0,100))--(XB-(0,100))--(XB-(100,100))--(XB+(-100,100))-- cycle);
    \end{asy}
\end{center}
Notice that
\begin{align*}
    \da(\overline{G_BO_B},\overline{G_CO_C})&=\da(\overline{G_BO_B},\overline{CA})+\measuredangle CAB+\da(\overline{AB},\overline{O_CG_C})\\
    &=\da(\overline{CA},\overline{GO})+\measuredangle CAB+\da(\overline{GO},\overline{AB})\\
    &=2\measuredangle CAB.
\end{align*}
Furthermore, \[\measuredangle G_BAG_C=\measuredangle G_BAG+\measuredangle GAG_C=2\measuredangle CAG+2\measuredangle GAB=2\measuredangle CAB\]
and \[\measuredangle O_BAO_C=\measuredangle O_BAO+\measuredangle OAO_C=2\measuredangle CAO+2\measuredangle OAB=2\measuredangle CAB,\]
so $X$ lies on $(ABC)$, $(AO_BO_C)$, and $(AG_BG_C)$, so we are done by symmetry. 

