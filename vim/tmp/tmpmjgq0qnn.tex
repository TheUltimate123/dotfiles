% Input your problem and solution below.
% Three dashes on a newline indicate the breaking points.

---

For positive integers $a$ and $b$, let $r(a,b)$ denote the remainder when $a$ is divided by $b$. Determine if there exists a constant $C>0$ and a positive integer $N$ such that for all integers $n\ge N$, \[\sum_{i=1}^nr(2020^n,i)>Cn(\log n)^e.\]

---

The answer is yes. We will prove the left-hand side is at least $O\big(n(\log n)^3\big)$. For some $k$, let $S_k$ denote the set of positive integers $i\le n$ such that $\gcd(2020^n,i)=k$. The key is this estimate:
\begin{iclaim*}
    For any $k\mid2020^n$ with $k\le n/4040$, \[\sum_{i\in S_k}r(2020^n,i)\ge O(n).\]
\end{iclaim*}
\begin{proof}
    Note that $r(2020^n,i)\ge k$ for any $i$, and \[|S_k|\ge\frac{\vphi(2020)}{2020}\left(\frac nk-2020\right)=\frac{40}{101}\left(\frac nk-2020\right),\]
    hence we have \[\sum_{i\in S_k}r(2020^n,i)\ge\frac{40}{101}(n-2020k)\ge\frac{20n}{101}=O(n)\]
    since $2020k\ge n/2$.
\end{proof}

Now this holds for any $k=2^a\cdot5^b\cdot101^c$ with \[0\le a,b,c\le\frac{\log_{101}(n/4040)}3,\]
so $k$ may be chosen in $O\big( (\log n)^3\big)$ ways, and we are done.

