% Input your problem and solution below.
% Three dashes on a newline indicate the breaking points.
% vim: tw=72

---

Let $A_1A_2\cdots A_n$ be a regular $n$-gon. For $i=1,\ldots,n-1$, if $i=1$ or $i=n-1$, then $B_i$ is the midpoint of the side $A_iA_{i+1}$. Otherwise, if $S_i=\overline{A_1A_{i+1}}\cap\overline{A_nA_i}$, then $B_i$ is intersection of the bisector of $\angle A_iS_iA_{i+1}$ with $\overline{A_iA_{i+1}}$. Prove that \[\angle A_1B_1A_n+\angle A_1B_2A_n+\cdots+\angle A_1B_{n-1}A_n=180^\circ.\]

---

The pith of this problem is this lemma:
\begin{boxlemma*}
    Let $ABCD$ be an isosceles trapezoid, so that $\overline{AB}\parallel\overline{CD}$ and $BC=DA$. Let $S=\overline{AC}\cap\overline{BD}$, and suppose the angle bisector of $\angle BSC$ intersects $\overline{BC}$ at $N$. If $P$ is the midpoint of $\overline{BC}$, then $ADPN$ is cyclic.
\end{boxlemma*}
\begin{proof}
    Let $T=\overline{AD}\cap\overline{BC}$ (if they are parallel the proof is trivial). Since $\overline{ST}$ bisects $\angle ASB$, by a well-known lemma, $-1=(BC;TN)$.  By the Midpoints of Harmonic Bundles Lemma, $TN\cdot TP=TB\cdot TC=TA\cdot TD$, and the desired result follows.
\end{proof}
\begin{center}
    \begin{asy}
        size(4.5cm);
        defaultpen(fontsize(10pt));
        pair A, B, C, D, SS, NN, P, T;
        A=(-3, 4);
        B=(3, 4);
        C=(6, 0);
        D=(-6, 0);
        SS=extension(A, C, B, D);
        NN=extension(SS, SS+(1, 0), B, C);
        P=(B+C)/2;
        T=extension(A, D, B, C);
        draw(B -- T -- A -- B -- C -- D -- A -- C, red);
        draw(B -- D, red);
        draw(T -- SS -- NN, orange);
        filldraw(circumcircle(A, D, P), orange+opacity(0.05), orange);
        clip((-100, -1) -- (100, -1) -- (100, 100) -- (-100, 100) -- cycle);

        dot("$A$", A, NW);
        dot("$B$", B, NE);
        dot("$C$", C, E);
        dot("$D$", D, W);
        dot("$S$", SS, S);
        dot("$N$", NN, NE);
        dot("$P$", P, E);
        dot("$T$", T, N);
    \end{asy}
\end{center}

Let $M$ be the midpoint of $\overline{A_1A_n}$, and $M_i$ the midpoint of $\overline{M_iM_{i+1}}$. Since $A_1A_n=A_iA_{i+1}$ and $A_1A_iA_{i+1}A_n$ is cyclic, $A_1A_iA_{i+1}A_n$ must be an isosceles trapezoid, whence by our lemma, \[\sum_{i=1}^{n-1}\angle A_1B_iA_n=\sum_{i=1}^{n-1}\angle A_1M_iA_n=\sum_{i=1}^{n-1}\angle A_iMA_{i+1}=\angle A_1MA_n=180^\circ,\]and we are done.
\begin{center}
    \begin{asy}
        size(8cm);
        defaultpen(fontsize(10pt));
        pair A1, A2, A3, A4, A5, A6, S2, S3, S4, B1, B2, B3, B4, B5, B6, M, M2, M4;
        A1=dir(90); A2=dir(30); A3=dir(330); A4=dir(270); A5=dir(210); A6=dir(150);
        S2=extension(A1, A3, A6, A2);
        S3=extension(A1, A4, A6, A3);
        S4=extension(A1, A5, A6, A4);
        B1=(A1+A2)/2;
        B2=extension(A1+S4-S3, S2, A2, A3);
        B3=(A3+A4)/2;
        B4=extension(A6+S2-S3, S4, A4, A5);
        B5=(A5+A6)/2;
        M=(A1+A6)/2;
        M2=(A2+A3)/2;
        M4=(A4+A5)/2;
        pair X=A1+S4-S3;
        pair O=(X+S3)/2;

        filldraw(circle((0, 0), 1), red+opacity(0.05), red);
        filldraw(circle(O, length(O)), orange+opacity(0.05), orange);
        draw(A1 -- A2 -- A3 -- A4 -- A5 -- A6 -- A1, red);
        draw(A3 -- A1 -- A5 -- A1 -- A4 -- A6 -- A2 -- A6 -- A3, red);
        draw(B2 -- X -- B3 -- X -- B4, orange+dashed);
        //draw(B1 -- A6 -- B2 -- A1 -- B3 -- A6 -- B4 -- A1 -- B5, orange+dotted);
        //draw(A6 -- M2 -- A1 -- M4 -- A6, fuchsia+dotted);
        draw(B1 -- M -- B2 -- M -- B3 -- M -- B4 -- M -- B5, pink);
        label("$A_1$", A1, A1);
        label("$A_2$", A2, A2);
        label("$A_3$", A3, A3);
        label("$A_4$", A4, A4);
        label("$A_5$", A5, A5);
        label("$A_6$", A6, A6);
        label("$S_2$", S2, NE);
        label("$S_3$", S3, E);
        label("$S_4$", S4, W);
        label("$B_1$", B1, unit(B1));
        label("$B_2$", B2, E);
        label("$B_3$", B3, unit(B3));
        label("$B_4$", B4, unit(M4));
        label("$B_5$", B5, W);
        label("$M$", M, dir(75));
        label("$M_2$", M2, E);
        label("$M_4$", M4, unit(M4));
        label("$X$", X, unit(X));
        dot(A1); dot(A2); dot(A3); dot(A4); dot(A5); dot(A6); dot(S2); dot(S3); dot(S4); dot(B1); dot(B2); dot(B3); dot(B4); dot(B5); dot(B6); dot(M); dot(M2); dot(M4); dot(X); dot(O);
    \end{asy}
\end{center}

