% Input your problem and solution below.
% Three dashes on a newline indicate the breaking points.

---

The following operation is allowed on a finite graph: Choose an arbitrary cycle of length 4 (if there is any), choose an arbitrary edge in that cycle, and delete it from the graph. For a fixed integer $n\ge 4$, find the least number of edges of a graph that can be obtained by repeated applications of this operation from the complete graph on $n$ vertices.

---

The answer is $n$. We can check that:
\begin{itemize}
    \item The operation preserves whether the graph is connected.
    \item The operation preserves whether the graph is bipartite. (We can check this by ``undoing'' the operation.)
\end{itemize}
Hence any resulting graph must be connected and have an odd cycle, i.e.\ have at least $n$ edges.

Now I claim this operation can achieve every connected non-bipartite graph. Indeed suppose $G$ is connected, non-bipartite, and edge-maximal with respect to 4-cycles. Then:
\begin{itemize}
    \item $G$ is triangle-free. If $v_1v_2v_3$ is a triangle, since $G$ is connected, there is a vertex $u$ connected to the triangle at, say, $v_1$. Then the edge $uv_3$ will create the 4-cycle $uv_1v_2v_3$.
    \item $G$ must contain a triangle. Otherwise, if $v_1v_2\cdots v_{2k+1}$ is the smallest odd-cycle, the edge $v_1v_3$ can be drawn to make a triangle $v_1v_2v_3$.
\end{itemize}
Thus, we can construct any tree with an extra edge added to form an odd cycle.

