% Input your problem and solution below.
% Three dashes on a newline indicate the breaking points.

---

Let $n$ be a positive integer. There are $\tfrac{n(n+1)}2$ tokens, each with a black side and a white side, arranged into an equilateral triangle, with the biggest row containing $n$ tokens. Initially, each token has the black side up. An \emph{operation} is to choose a line parallel to the sides of the triangle, and flip all the tokens on that line. A configuration is called \emph{admissible} if it can be obtained from the initial configuration by performing a finite number of operations. For each admissible configuration $C$, let $f(C)$ denote the smallest number of operations required to obtain $C$ from the initial configuration. Find the maximum value of $f(C)$, where $C$ varies over all admissible configurations.

---

There are $3n$ possible moves, and each may be performed either never or once. Thus we interpret the situation as a linear map $T:\mathbb F_2^{3n}\to\mathbb F_2^{n(n+1)/2}$.

First define the unit vectors $\hat a_1$, $\ldots$, $\hat a_n$ as the $n$ lines all parallel to some direction, such that $a_i$ contains $i$ tokens; analogously define $\hat b_i$ and $\hat c_i$. Let $\mathbf a=\hat a_1+\cdots+\hat a_n$, and similarly define $\mathbf b$, $\mathbf c$. Finally let $\mathbf d$ denote the vector that contains all lines with an even number of tokens.

The key claim is as follows:
\begin{claim*}
    $\ker T$ has basis $\{\mathbf d,\mathbf a+\mathbf b,\mathbf a+\mathbf c\}$, hence $\left\lvert\ker T\right\rvert=8$.
\end{claim*}
\begin{proof}
    It is easy to verify the three aforementioned vectors are in $\ker T$. (We can verify $T(\mathbf d)=0$ by Vivani's theorem.) Now let $\mathbf v\in\ker T$ be independent of $\{\mathbf d,\mathbf a+\mathbf b,\mathbf a+\mathbf c\}$.
    Now without loss of generality, by adding a subset of $\{\mathbf d,\mathbf a+\mathbf b,\mathbf a+\mathbf c\}$, that
    $\mathbf v$ does not use $\hat a_2$, $\hat b_n$, $\hat c_n$.

    Then it is easy to inductively check $\mathbf v$ is the zero vector, by repeated spamming the argument ``if $\mathbf v$ does not use two lines through some token, then it does not use the third.''
\end{proof}

Now comes the computation. For $\mathbf u\in\mathbb F_2^{3n}$, let $\left\lvert\mathbf u\right\rvert$ denote the number of ones in the representation of $\mathbf u$. For a representation $\mathbf v\in\mathbb F_2^{3n}$ of $C$, we can express \[f(C)=\min_{\mathbf w\in\ker T}\left\lvert\mathbf v+\mathbf w\right\rvert\]
as the minimum of eight values.

Casework modulo four gives
\[\min_Cf(C)=\begin{cases}
        6k&\text{if }n=4k\\
        6k+1&\text{if }n=4k+1\\
        6k+2&\text{if }n=4k+3\\
        6k+3&\text{if }n=4k+4.
\end{cases}\]

