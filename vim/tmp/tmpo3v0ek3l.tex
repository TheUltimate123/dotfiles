% Input your problem and solution below.
% Three dashes on a newline indicate the breaking points.

---

Find all functions $f:\mathbb Z\to\mathbb Z$ such that \[f(f(x)+f(y))+f(x)f(y)=f(x+y)f(x-y)\]
for all integers $x$ and $y$.

---

The answer is $f\equiv0$ and $f(x)=\left(\frac x5\right)$. The former obviously works; to check the latter, note that both sides of the functional equation are in $[-2,2]$, so it suffices to show they are congruent modulo $5$. Since $\left(\frac x5\right)\equiv x^2\pmod5$, just note that \[\left(x^2+y^2\right)^2+x^2y^2=\left(x^2+y^2\right)^2-4x^2y^2=(x+y)^2(x-y)^2.\]
We will now show these are the only solutions.

Let $P(x,y)$ denote the assertion.
\setcounter{iclaim}0
\begin{iclaim}
    $f(0)=0$.
\end{iclaim}
\begin{proof}
    Check that
    \begin{itemize}[itemsep=0em]
        \item $P(0,0)\implies f(2f(0))=0$,
        \item $P(0,2f(0))\implies f(f(0))=0$, and
        \item $P(f(0),f(0))\implies f(0)=0$,
    \end{itemize}
    as claimed.
\end{proof}
\begin{iclaim}
    $f$ is even.
\end{iclaim}
\begin{proof}
    Note that $P(x,y)$ and $P(y,x)$ give $f(x+y)f(x-y)=f(x+y)f(y-x)$. This means that if $f(n)\ne f(-n)$, for every choice of $x$, $y$ with $x-y=n$, we have $f(x+y)=0$; in other words, $f(m)=0$ for all $m\equiv n\pmod2$.

    Then $f(n)=f(-n)=0$, contradiction.
\end{proof}
\begin{iclaim}
    For integers $n$, we have $f(nf(x))=f(x)^2\left(\frac n5\right)$.
\end{iclaim}
\begin{proof}
    This is clearly true for $n=0$, and also when $f(x)=0$ --- henceforth assume $f(x)\ne0$. We prove the hypothesis only for positive integers $n$; this is sufficient since $f$ is even.

    First $P(x,0)$ and $P(x,x)$ give $f(f(x))=f(x)^2$ and $f(2f(x))=-f(x)^2$ respectively, proving the hypothesis for $n=1$ and $n=2$. Now, we prove the hypothesis inductively for $5\nmid n$ as follows. (Here, $k$ is a nonnegative integer.)
    \begin{itemize}[itemsep=0em]
        \item For $n=5k+3$, $P(f(x),(5k+2)f(x))\implies-f(x)^4=f(x)^2f( (5k+3)f(x))$.
        \item For $n=5k+4$, $P(f(x),(5k+3)f(x))\implies-f(x)^4=-f(x)^2f( (5k+4)f(x))$.
        \item For $n=5k+6$, $P(2f(x),(5k+4)f(x))\implies-f(x)^4=-f(x)^2f( (5k+6)f(x))$.
        \item For $n=5k+7$, $P(3f(x),(5k+4)f(x))\implies-f(x)^4=f(x)^2f( (5k+7)f(x))$.
    \end{itemize}

    To prove the hypothesis for $5\mid n$, we first check that \[f(2f(x)^2)=f(2f(f(x)))=-f(f(x))^2=-f(x)^4,\]
    so $P(f(x),(5k+4)f(x))\implies0=-f(x)^2f(5f(x))$, and the claim has been proven.
\end{proof}

Finally by Claim 3, whenever $f(x),f(y)\ne0$, \[f(x)^2\left(\frac{f(y)}5\right)=f(f(x)f(y))=f(y)^2\left(\frac{f(x)}5\right)\implies\frac{\left(\frac{f(x)}5\right)}{f(x)^2}=\frac{\left(\frac{f(y)}5\right)}{f(y)^2}\]
is fixed. Since the numerator of each fraction is in $\{0,1,-1\}$, we know $f(x)\ne0$ implies $|f(x)|$ is fixed. Recalling that $f(f(x))=f(x)^2$ shows that this fixed value is $1$, so if $f\not\equiv0$, then at least one of $1$ or $-1$ is in the range of $f$. Claim 3 with $x$ obeying $|f(x)|=1$ completes the proof.

