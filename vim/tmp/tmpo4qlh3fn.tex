% Input your problem and solution below.
% Three dashes on a newline indicate the breaking points.

---

Suppose that circles $\Omega_1$ and $\Omega_2$ intersect at $P$ and $Q$, and that line $AB$ is tangent to $\Omega_1$ and $\Omega_2$ at $A$ and $B$, respectively, such that $Q$ is closer to $\overline{AB}$ than $P$. If $AB=2$, $PA=20$, and $PB=19$, then $QA\cdot QB$ can be expressed in the form $\tfrac mn$, where $m$ and $n$ are relatively prime positive integers. Find the remainder when $m+n$ is divided by $1000$.

---

\begin{center}
    \begin{asy}
        size(8cm);
        defaultpen(fontsize(10pt));

        pen pri=deepgreen;
        pen sec=springgreen;
        pen fil=chartreuse+opacity(0.05);
        pen sfil=springgreen+opacity(0.05);

        pair A, B, M, O1, O2, P, Q;
        path G1, G2;
        A=(-1, 0);
        B=(1, 0);
        M=(0, 0);
        O1=(-1, 3);
        O2=(1, 2);
        G1=circle(O1, length(O1-A));
        G2=circle(O2, length(O2-B));
        P=intersectionpoints(G1, G2)[0];
        Q=intersectionpoints(G1, G2)[1];

        filldraw(G1, sfil, sec);
        filldraw(G2, sfil, sec);

        draw(foot(O1+length(O1-A)*dir(180), A, B) -- foot(O2+length(O2-B)*dir(0), A, B), pri);
        // draw(foot(O1+length(O1-A)*dir(250), A, B) -- foot(O2+length(O2-B)*dir(0), A, B), pri);
        draw(A -- P -- B -- Q -- A, pri); draw(P -- M, pri);
        fill(P -- A -- B -- cycle, fil);

        //clip(foot(O1+length(O1-A)*dir(250), A, B) -- extension(foot(O1+length(O1-A)*dir(250), A, B), foot(O1+length(O1-A)*dir(250), A, B)+(0, 1), O1+length(O1-A)*dir(30), O1+length(O1-A)*dir(30)+(1, 0)) -- extension(foot(O2+length(O2-A)*dir(0), A, B), foot(O2+length(O2-A)*dir(0), A, B)+(0, 1), O1+length(O1-A)*dir(30), O1+length(O1-A)*dir(30)+(1, 0)) -- foot(O2+length(O2-A)*dir(0), A, B) -- cycle);

        dot("$A$", A, S);
        dot("$B$", B, S);
        dot("$M$", M, S);
        dot("$P$", P, NE);
        dot("$Q$", Q, dir(105));
    \end{asy}
\end{center}

First, boxremark that if in any triangle $XYZ$, $YZ=2$ and $N$ denotes the midpoint of $\overline{YZ}$, then by Stewart's Theorem, \[XN^2=\frac{XY^2+XZ^2}2-1.\]
Let $M=\overline{AB}\cap\overline{PQ}$. Then, $MA^2=MP\cdot MQ=MB^2$, so $M$ is the midpoint of $\overline{AB}$. By the Tangency Criterion, $\triangle MAQ\sim\triangle MPA$ and $\triangle MBQ\sim\triangle MPB$. Hence, \[\frac{PA}{QA}=\frac{MP}{MA}=\frac{MP}{MB}=\frac{PB}{QB}.\]
It follows that there exists a real number $t$ such that $QA=20t$ and $QB=19t$. By Power of a Point from $M$, $MP\cdot MQ=1$. If $u=\tfrac{20^2+19^2}2=\tfrac{761}2$, using our boxremark, \[1=MP^2\cdot MQ^2=\left(\frac{20^2+19^2}2-1\right)\left(t^2\cdot\frac{20^2+19^2}2-1\right)=(u-1)(t^2u-1),\]
whence \[1+\frac1{t^2}=u=\frac{761}2\implies t^2=\frac2{759}\implies QA\cdot QB=380t^2=\frac{760}{759},\]
and the requested remainder is $760+759\equiv 519\pmod{1000}$.

---

519
