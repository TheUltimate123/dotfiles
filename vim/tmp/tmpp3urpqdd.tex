% Input your problem and solution below.
% Three dashes on a newline indicate the breaking points.

---

Let $P$ be a nonconstant polynomial with integer coefficients and $n$ a positive integer. Let $a_0=n$, and let $a_k=P(a_{k-1})$ for all positive integers $k$. Suppose that for every positive integer $b$ there is an element of the sequence $a_0$, $a_1$, $\ldots$ that is the $b$th power of some positive integer greater than 1. Prove that $P$ is a linear polynomial.

---

Observe that for each $b$, the sequence contains infinitely many powers of $b$. (We can find powers of $b$, $2b$, $3b$, and so on.)
\setcounter{claim}0
\begin{claim}
    For all $i$,
    \[a_{i+1}-a_i\quad\text{divides}\quad a_i(a_i-1).\]
\end{claim}
\begin{proof}
    Observe that $a_{i+1}-a_i\mid a_{j+1}-a_j$ for all $j\ge i$, so
    \[a_i\equiv a_{i+1}\equiv a_{i+2}\equiv\cdots\pmod{a_{i+1}-a_i}.\]
    But there are infinitely many $b$th powers in the sequence, so for any choice of $b$ and $i$, $a_i$ is a $b$th power residue modulo $a_{i+1}-a_i$.

    Now for each $p^e\parallel a_{i+1}-a_i$, let $b=\vphi(p^e)$, so $a_i\equiv 0,1\pmod{p^e}$. The claim follows from Chinese Remainder theorem.
\end{proof}
\begin{claim}
    $\deg P\le2$.
\end{claim}
\begin{proof}
    We have by Claim 1 that $|a_{i+1}-a_i|\le|a_i(a_i-1)|$. The degree of $a_{i+1}-a_i$ as a polynomial in $n$ is $(\deg P)^{i+1}$, and the degree of $a_i(a_i-1)$ is $2(\deg P)^i$. The claim follows from
    \[(\deg P)^{i+1}\le2(\deg P)^i.\]
\end{proof}

It will suffice to show no quadratic $P$ work. Let $P(x)=cx^2+dx+e$.
\begin{claim}
    $P(x)$ is of the form $\pm x^2+dx$; that is, $e=0$ and $c=\pm1$.
\end{claim}
\begin{proof}
    We use Claim 1 again:
    \begin{itemize}
        \item Check that $\gcd(a_i,a_{i+1}-a_i)=\gcd(a_i,P(a_i))=\gcd(a_i,e)\mid e$, so Claim 1 may be strengthened to
            \[a_{i+1}-a_i\quad\text{divides}\quad e(a_i-1).\]
            By degree counting, this is absurd unless $e=0$. Henceforth $P(x)=x(cx+d)$.
        \item Now, $a_{i+1}-a_i=a_i(ca_i+d-1)$ divides $a_i(a_i-1)$, so
            \[ca_i+d-1\quad\text{divides}\quad a_i-1.\]
            It follows from $|ca_i+d-1|\le|a_i-1|$ that $c=\pm1$.
    \end{itemize}
\end{proof}

The remaining cases may be verified by hand:
\begin{itemize}
    \item If $c=-1$, then the sequence $a_1$, $a_2$, $\ldots$ is eventually negative and therefore a square finitely often.
    \item If $c=1$ and $d\ne0$, then $4P(x)=4\left(x^2+dx\right)=(2x+d)^2-d^2$ is a square finitely often, so the sequence $a_1$, $a_2$, $\ldots$ contains finitely many squares.
    \item Finally, $P(x)=x^2$ obviously fails. For instance, take $b$ an odd integer such that $x$ is not a perfect $b$th power.
\end{itemize}


