% Input your problem and solution below.
% Three dashes on a newline indicate the breaking points.

---

A pile of $2001$ pebbles is placed in a vertical column. This configuration is modified according to the following rules. A pebble can be moved if it is at the top of a column which contains at least two more pebbles than the column immediately to its right. (If there are no pebbles to the right, think of this as a column with $0$ pebbles.)

Show that if this operation is applied repeatedly, then we always reach a unique final configuration where no more moves are possible; furthermore describe this final configuration.

---

View the sizes of the columns from left-to-right as a sequence. Of course this operation must eventually terminate, since the sequence decreases lexicographically.

It is easy to check that:
\begin{itemize}[itemsep=0em]
    \item The elements of the sequence are always nonincreasing.
    \item In the final configuration, no two consecutive terms differ by at least $2$.
\end{itemize}
In addition, we claim:
\begin{iclaim*}
    For all $a$ and $i>0$, we cannot obtain the subsequence \[(a,a,a-1,a-2,\ldots,a-i+1,a-i,a-i).\]
    For $i=0$, we cannot obtain $(a,a,a)$.
\end{iclaim*}
\begin{proof}
    Induction on $i$. The base case, $i=0$, is because there is no possible sequence right before we obtain $(a,a,a)$. For the inductive step, immediately before obtaining $(a,a,\ldots,a-i,a-i)$ we must have $(a,a,\ldots,a-i+1,a-i+1)$, which is impossible by inductive hypothesis.
\end{proof}

Thus the final configuration is unique, and may be described as follows: each positive integer $n$ may be expressed uniquely as \[n=(1+2+\cdots+m)+k\quad\text{where }0\le k\le m.\]
Then the final configuration for an initial column of $n$ stones is \[(m,m-1,\ldots,k+1,k,k,k-1,\ldots,2,1).\]
(If $k=0$, then the sequence is just $(m,m-1,\ldots,2,1)$, since the last pile of size $0$ may be omitted.) For $n=2001$, we have $m=62$ and $k=48$.

