% Input your problem and solution below.
% Three dashes on a newline indicate the breaking points.

---

Jenn randomly choose a number $J$ from $1,2,3,\ldots,19,20$. Bela then randomly chooses a number $B$ from $1,2,3,\ldots,19,20$ distinct from $J$. The value of $B-J$ is at least $2$ with a probability that can be expressed in the form $\tfrac mn$, where $m$ and $n$ are relatively prime positive integers. Find $m+n$.

---

Bela must pick $J+2,J+3,\ldots,20$, so the number of successful outcomes is \[\sum_{J=1}^{18}\big(20-(J+2)+1\big)=\frac{18\cdot 19}2=9\cdot 19,\]
so the answer is $(9\cdot 19)/(20\cdot 19)=9/20$, and the requested sum is $9+20=29$.

---

029
