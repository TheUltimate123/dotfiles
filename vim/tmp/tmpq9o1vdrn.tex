% Input your problem and solution below.
% Three dashes on a newline indicate the breaking points.

---

Suppose $2019$ chicks are sitting in a circle. Suddenly, each chick randomly pecks either the chick on its left or the chick on its right with equal probability. Let $k$ be the number of chicks that were not pecked. The probability $k$ is odd can be expressed as $\frac{p}{q}$, where $p$ and $q$ are relatively prime positive integers. Find the remainder when $p+q$ is divided by $1000$. 

---

Define $p_n(k)$ as the probability $k$ chicks were not pecked if $n$ chicks were sitting in the circle, where $n$ is odd.
\begin{iclaim}
    We have \[p_n(k)=\frac{1}{2^{n-1}}\binom{n}{2k}.\]
\end{iclaim}
\begin{proof}
    Number the chicks $1$ to $n$. Suppose we have a circular string $s$ composed of L's and R's such that a L represents a peck to the left and a R represents a peck to the right. (Note circular means that the character after the last character is the first character). A chick is not pecked iff the letter to the left is L and the letter to the right is R. Now, reorder the string $s$ to a new circular string $s'$ such that the chicks are in the order $1,3,5,\ldots,n,2,4,6,\ldots,n-1$. 

    The number of unpecked chicks is then the number of times the string \enquote{LR} appears in $s'$. Note that if \enquote{LR} appears $k$ times, \enquote{RL} also appears $k$ times. This is because $s'$ is circular, so there has to be a point where the string transitions from R's to L's. Note that this reasoning is similar to the concept of prefix sums. In addition, the \enquote{LR's} and \enquote{RL's} are alternating.

    We can choose $s'$ in $2^n$ ways. Now, we can choose the starting positions of the \enquote{LR's} and \enquote{RL's} in $\binom{n}{2k}$ ways. Note that there is no overcount because even if we choose two adjacent positions, we can construct a string such as \enquote{LRL.} Consider the first position we choose in the string. It can be either \enquote{LR} or \enquote{RL,} so we must multiply by $2$. Note that the rest of the positions we choose rely on the previous, so they are determined by what we choose for the first. Therefore, \[p_n(k)=\frac{2}{2^n}\binom{n}{2k}=\frac{1}{2^{n-1}}\binom{n}{2k},\]
    and our claim has been proven.
\end{proof}

Since for $p_n(k)$ to be nonzero, $k\le\left\lfloor\frac{n}{2}\right\rfloor$, the probability $k$ is odd is
\[\sum_{i=0}^{505}p_{2019}(2n-1)=\frac{1}{2^{2018}}\left(\sum_{i=0}^{505}\binom{2019}{4n-2}\right)=\frac{m}{2^{2018}}\]
for some $m$. We can evaluate $m$ by Roots of Unity Filter. Consider the function $f(n)=(n+1)^{2019}$. Suppose $\omega=e^{\pi i/2}=i$ is a $4^\text{th}$ root of unity. Then, \[m=\frac{f\left(1\right)-f\left(\omega\right)+f\left(\omega^2\right)-f\left(\omega^3\right)}{4}=\frac{2^{2019}-(1+i)^{2019}+0^{2019}-(1-i)^{2019}}{4}.\]
Suppose $t=(1+i)^{2019}+(1-i)^{2019}$. Note that $m=\dfrac{2^{2019}-t}{4}$. Writing in Polar Form,
\begin{align*}
    t&=\raisebox{4pt}{$\left(\text{\raisebox{-4pt}{$\sqrt{2}e^{\text{\raisebox{4pt}{$\frac{\raisebox{1pt}{$\pi i$}}{\raisebox{-1.75pt}{$4$}}$}}}$}}\right)^{2019}$}+\raisebox{4pt}{$\left(\text{\raisebox{-4pt}{$\sqrt{2}e^{\text{\raisebox{4pt}{$\frac{\raisebox{1pt}{$7\pi i$}}{\raisebox{-1.75pt}{$4$}}$}}}$}}\right)^{2019}$} \\
    &=2^{\text{\raisebox{4pt}{$\frac{\raisebox{1pt}{$2019$}}{\raisebox{-1.75pt}{$2$}}$}}}\text{\raisebox{4pt}{$\left(\text{\raisebox{-4pt}{$e^{\text{\raisebox{4pt}{$\frac{\raisebox{1pt}{$2019\pi i$}}{\raisebox{-1.75pt}{$4$}}$}}}+e^{\text{\raisebox{4pt}{$\frac{\raisebox{1pt}{$2019\cdot 7\pi i$}}{\raisebox{-1.75pt}{$4$}}$}}}$}}\right)$}} \\
    &=2^{\text{\raisebox{4pt}{$\frac{\raisebox{1pt}{$2019$}}{\raisebox{-1.75pt}{$2$}}$}}}\text{\raisebox{4pt}{$\left(\text{\raisebox{-4pt}{$e^{\text{\raisebox{4pt}{$\frac{\raisebox{1pt}{$3\pi i$}}{\raisebox{-1.75pt}{$4$}}$}}}+e^{\text{\raisebox{4pt}{$\frac{\raisebox{1pt}{$5\pi i$}}{\raisebox{-1.75pt}{$4$}}$}}}$}}\right)$}} \\
    &=2^{\text{\raisebox{4pt}{$\frac{\raisebox{1pt}{$2019$}}{\raisebox{-1.75pt}{$2$}}$}}}\left(-\frac{\sqrt{2}}{2}+\frac{\sqrt{2}}{2}i-\frac{\sqrt{2}}{2}-\frac{\sqrt{2}}{2}i\right) \\
    &=-2^{\text{\raisebox{4pt}{$\frac{\raisebox{1pt}{$2019$}}{\raisebox{-1.75pt}{$2$}}$}}}\sqrt{2} \\
    &=-2^{1010}.
\end{align*}
It follows that \[m=\frac{2^{2019}+2^{1010}}{4}=2^{2017}+2^{1008}=2^{1008}\left(2^{1009}+1\right).\]
Then, our desired probability is \[\frac{m}{2^{2018}}=\frac{2^{1008}\left(2^{1009}+1\right)}{2^{2018}}=\frac{2^{1009}+1}{2^{1010}}.\]
It is easy to see that $\gcd\left(2^{1009}+1,2^{1010}\right)=1$, so we desire $2^{1009}+1+2^{1010}=3\cdot 2^{1009}+1$. Let this value be $x$. By CRT, we only seek $x$ modulo $8$ and $125$.

It is easy to see that $x\equiv 1\pmod{8}$. By Euler's Formula, $2^{\phi(125)}\equiv 2^{100}\equiv 1\pmod{125}$. Therefore, $x\equiv 3\cdot 2^9+1\equiv 3\cdot 512+1\equiv 37\pmod{125}$. We can write $x=8j+1$ for some $j$. Then, $8j+1\equiv 37\pmod{125}$, and $8j\equiv 36\pmod{125}$, so $2j\equiv 9$ and $j\equiv\frac{9}{2}\equiv\frac{567}{126}\equiv 67\pmod{125}$. It follows that $x\equiv 8\cdot 67+1\equiv 537\pmod{1000}$, and we are done.

---

537
