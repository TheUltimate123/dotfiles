% Input your problem and solution below.
% Three dashes on a newline indicate the breaking points.

---

Let $ABCD$ be a circumscribed quadrilateral. Let $g$ be a line through $A$ which meets segment $BC$ at $M$ and line $CD$ at $N$. Denote by $I_1$, $I_2$, $I_3$ the incenters of $\triangle ABM$, $\triangle MNC$, $\triangle NDA$, respectively. Prove that the orthocenter of $\triangle I_1I_2I_3$ lies on $g$.

---

\begin{center}
    \begin{asy}
        size(7cm); defaultpen(fontsize(10pt));

        pair I,WW,X,Y,Z,A,B,C,D,M,NN,I1,I2,I3,H,EE;
        I=(0,0);
        WW=dir(100);
        X=dir(170);
        Y=dir(300);
        Z=dir(30);
        A=2*WW*X/(WW+X);
        B=2*X*Y/(X+Y);
        C=2*Y*Z/(Y+Z);
        D=2*Z*WW/(Z+WW);
        M=(2*B+7*C)/9;
        NN=extension(A,M,C,D);
        I1=incenter(A,B,M);
        I2=incenter(M,NN,C);
        I3=incenter(NN,D,A);
        H=orthocenter(I1,I2,I3);
        EE=extension(I1,I3,A,M);

        draw(incircle(A,B,M),gray);
        draw(incircle(M,NN,C),gray);
        draw(incircle(NN,D,A),gray);
        draw(C--reflect(I1,I3)*foot(I1,A,M));
        draw(circumcircle(I1,I2,I3));
        draw(unitcircle);
        draw(A--B--C--D--A--NN--C);

        dot("$A$",A,NW);
        dot("$B$",B,S);
        dot("$C$",C,dir(30));
        dot("$D$",D,NE);
        dot("$M$",M,S);
        dot("$N$",NN,SE);
        dot("$I_1$",I1,W);
        dot("$I_2$",I2,SE);
        dot("$I_3$",I3,N);
        dot("$E$",EE,dir(240));
    \end{asy}
\end{center}
Let $\omega$, $\omega_1$, $\omega_2$, $\omega_3$ be the incircles of $ABCD$, $\triangle ABM$, $\triangle MNC$, $\triangle NDA$ respectively.
\setcounter{iclaim}0
\begin{iclaim}
    The other common internal tangent of $\omega_1$ and $\omega_3$ passes through $C$.
\end{iclaim}
\begin{proof}
    Let the second tangent from $C$ to $\omega_1$ intersect $g$ at $E$. Then $ABCD$ and $ABCE$ are tangential, so by Pitot's theorem,
    \begin{align*}
        AB+CD&=BC+DA,\\
        AB+CE&=BC+EA.
    \end{align*}
    Subtracting and rearranging, \[CD+EA=DA+CE,\]
    so $ADCE$ is tangential, and $E$ is the insimilicenter of $\omega_1$, and $\omega_3$, as desired.
\end{proof}
\begin{iclaim}
    $C$ lies on $(I_1I_2I_3)$.
\end{iclaim}
\begin{proof}
    Note that \[\angle I_1I_2I_3=\angle I_2NM+\da NMI_2=\frac{\angle CNM+\angle NMC}2=\frac{\angle BCD}2,\]
    and furthermore \[\angle I_1CI_2=\angle I_1CE+\ange ECI_3=\frac{\angle BCE+\angle ECD}2=\frac{\angle BCD}2,\]
    as desired.
\end{proof}

Finally $g$ is the Steiner line of $C$ with respect to $\triangle I_1I_2I_3$, so the orthocenter of $\triangle I_1I_2I_3$ lies on $g$, and we are done.

