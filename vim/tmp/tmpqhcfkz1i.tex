% Input your problem and solution below.
% Three dashes on a newline indicate the breaking points.

---

In triangle $ABC$, $AB=13$, $BC=14$, $CA=15$, and a point $P$ lies on $\overline{BC}$. Let $Q$ be the foot of the perpendicular from $P$ to $\overline{AB}$ and $R$ be the foot of the perpendicular from $P$ to $\overline{AC}$. Suppose $I_B$ and $I_C$ are the incenters of triangles $\triangle PBQ$ and $\triangle PCR$, respectively. Then the maximum possible area of $\triangle PI_BI_C$ is $\frac{m}{n}$, where $m$ and $n$ are relatively prime positive integers. Find $m+n$. 

---

\begin{center}
    \begin{asy}
        size(5cm); defaultpen(fontsize(10pt));

        pair A, B, C, P, Q, R, I1, I2;
        A=(0, 12);
        B=(-5, 0);
        C=(9, 0);
        P=(2, 0);
        Q=(2-7*144/169, 7*60/169);
        R=(2+7*16/25, 7*12/25);
        I1=incenter(P, B, Q);
        I2=incenter(P, C, R);
        dot(A); dot(B); dot(C); dot(P); dot(Q); dot(R); dot(I1); dot(I2);
        draw(A -- B -- C -- A);
        draw(Q -- P -- R);
        draw(P -- I1 -- I2 -- P);
        draw(A -- P);
        draw(rightanglemark(P, R, A, 16));
        draw(rightanglemark(P, Q, A, 16));
        draw(incircle(P, B, Q));
        draw(incircle(P, C, R));
        label("$A$", A, N);
        label("$B$", B, SW);
        label("$C$", C, SE);
        label("$P$", P, S);
        label("$Q$", Q, NW);
        label("$R$", R, NE);
    \end{asy}
\end{center}

First, we will find a closed form for the area of $\triangle PI_BI_C$. Notice that $\overline{PI_B}$ bisects $\angle BPQ$ and $\overline{PI_C}$ bisects $\angle CPR$, so \[\angle I_BPI_C=180-\frac{\left(90-\angle B\right)+\left(90-\angle C\right)}{2}=180-\frac{\angle A}{2}\]
Since $\angle I_BPI_C$ is fixed, it suffices to maximize $PI_B\cdot PI_C$.

Consider $\triangle PBI_B$. Notice that $\angle BI_BP=90+\frac{\angle BQP}{2}=135^\circ$. By the Law of Sines, \[\frac{PI_B}{\sin\angle PBI_B}=\frac{BP}{\sin\angle BI_BP}\implies PI_B=\sqrt{2}\cdot BP\cdot\sin\left(\frac{\angle B}{2}\right).\]
Similarly, we know $PI_C=\sqrt{2}\cdot CP\cdot\sin\left(\frac{\angle C}{2}\right)$. It follows that
\begin{align*}
    \left[PI_BI_C\right]&=\frac{1}{2}\cdot PI_B\cdot PI_C\cdot\sin\angle I_BPI_C \\
    &=\frac{1}{2}\cdot\sqrt{2}\cdot BP\cdot\sin\left(\frac{\angle B}{2}\right)\cdot\sqrt{2}\cdot CP\cdot\sin\left(\frac{\angle C}{2}\right)\cdot\sin\left(\frac{\angle A}{2}\right) \\
    &=BP\cdot CP\cdot \sin\left(\frac{\angle A}{2}\right)\sin\left(\frac{\angle B}{2}\right)\sin\left(\frac{\angle C}{2}\right).
\end{align*}
Since the angles are fixed, it suffices to maximize $BP\cdot CP$. Since $BP+CP=14$, by AM-GM the maximum possible value of $BP\cdot CP$ occurs when $BP=CP=7$.

It follows that \[\left[PI_BI_C\right]=49\sin\left(\frac{\angle A}{2}\right)\sin\left(\frac{\angle B}{2}\right)\sin\left(\frac{\angle C}{2}\right)\]
Suppose $X$ is the foot of the perpendicular from $A$ to $\overline{BC}$. It is well known that $AX=12$, $BX=5$, and $CX=9$. It is easy to see by the Cosine Addition Formula that $\cos\angle A=\frac{33}{65}$. Then, we can apply the Half Angle Formulas to see that
\begin{align*}
    \left[PI_BI_C\right]&=49\sqrt{\frac{\left(1-\frac{33}{65}\right)\left(1-\frac{5}{13}\right)\left(1-\frac{3}{5}\right)}{2\cdot 2\cdot 2}}\\
    &=49\sqrt{\frac{\frac{32}{65}\cdot\frac{8}{13}\cdot\frac{2}{5}}{2\cdot 2\cdot 2}}\\
    &=49\sqrt{\frac{8^2}{65^2}}=\frac{49\cdot 8}{65}=\frac{392}{65}.
\end{align*}
The answer is then $392+65=457$.

---

457
