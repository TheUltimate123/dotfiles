% Input your problem and solution below.
% Three dashes on a newline indicate the breaking points.

---

Let $ABC$ be a triangle and let $M$ and $N$ denote the midpoints of $\overline{AB}$ and $\overline{AC}$, respectively. Let $X$ be a point such that $\overline{AX}$ is tangent to the circumcircle of triangle $ABC$. Denote by $\omega_B$ the circle through $M$ and $B$ tangent to $\overline{MX}$, and by $\omega_C$ the circle through $N$ and $C$ tangent to $\overline{NX}$. Show that $\omega_B$ and $\omega_C$ intersect on line $BC$.

---

\begin{center}
    \begin{asy}
        size(8cm); defaultpen(fontsize(10pt));
        pen pri=red;
        pen sec=orange;
        pen tri=fuchsia;
        pen fil=pri+opacity(0.05);
        pen sfil=sec+opacity(0.05);
        pen tfil=tri+opacity(0.05);

        pair O,A,B,C,M,NN,X,P,Q;
        O=(0,0);
        A=dir(130);
        B=dir(220);
        C=dir(320);
        M=(A+B)/2;
        NN=(A+C)/2;
        X=A+1.8*(rotate(90)*A);
        P=reflect(X,(A+O)/2)*A;
        Q=(A+P)/2;

        filldraw(circumcircle(P,B,M),tfil,tri);
        filldraw(circumcircle(P,C,NN),tfil,tri);
        draw(Q--M--X,tri);
        filldraw(circumcircle(A,M,NN),sfil,sec);
        draw(P--X,sec);
        filldraw(circle(O,1),fil,pri);
        draw(A--B--C--A--X,pri);
        draw(A--P,pri);

        dot("$A$",A,A);
        dot("$B$",B,B);
        dot("$C$",C,C);
        dot("$M$",M,dir(160));
        dot("$N$",NN,dir(40));
        dot("$X$",X,W);
        dot("$P$",P,dir(-30));
        dot("$Q$",Q,E);
    \end{asy}
\end{center}
\begin{customenv}{First solution, by symmedians}
    By Miquel's theorem, it suffics to show $\omega_B$ and $\omega_C$ intersect on $(AMN)$. By homothety, $\seg{AX}$ is tangent to $(AMN)$. Let $\seg{XP}$ be the other tangent from $X$ to $(AMN)$. I claim $P$ is this concurrence point.

    It suffices to show that $\seg{XM}$ is tangent to $(BMP)$, from which $\omega_B=(BMP)$, and we finish by symmetry. Let $Q$ be the midpoint of $\seg{AP}$. Since $\seg{MX}$ is the $M$-symmedian of $\triangle MAP$ and $\seg{MQ}$ is a midsegment of $\triangle ABP$, \[\da XMP=\da AMP=\da ABP=\da MBP,\]
    concluding the proof.
\end{customenv}
\begin{customenv}{Second solution, by moving points}
    Animate $X$ at a linear rate on the tangent to $(AMN)$ at $A$, and let $\omega_B$ intersect $\seg{BC}$ at $R\ne B$.

    Note that since $\da XMR=\da MBR=\da ABC$ is constant, $R$ moves projectively along $\seg{BC}$. Similarly the second intersection of $\omega_C$ and $\seg{BC}$ moves projectively, so it suffices to verify the hypothesis for three choices of $X$.
    \begin{itemize}[itemsep=0em]
        \item If $X=A$, then $R$ is the point at infinity along $\seg{BC}$.
        \item If $X$ is the point at infinity along $\seg{AA}$, then $R$ is the midpoint of $\seg{BC}$.
        \item If $X=\seg{AA}\cap\seg{MN}$, then $R$ is the foot of the altitude from $A$.
    \end{itemize}
    This concludes the proof.
\end{customenv}

