% Input your problem and solution below.
% Three dashes on a newline indicate the breaking points.

---

Let $T=13$. Let $ABC$ be a triangle with incircle $\omega$. Points $E$, $F$ lie on $\seg{AB}$, $\seg{AC}$ such that $\seg{EF}\parallel\seg{BC}$ and $\seg{EF}$ is tangent to $\omega$. If $EF=T$ and $BC=T+1$, compute $AB+AC$.

---

\begin{center}
\begin{asy}
    size(4cm); defaultpen(fontsize(10pt));

    pair A,B,C,I,D,T,EE,F;
    A=dir(110);
    B=dir(220);
    C=dir(320);
    I=incenter(A,B,C);
    D=foot(I,B,C);
    T=2I-D;
    EE=extension(A,B,T,T+B-C);
    F=extension(A,C,T,EE);

    draw(A--B--C--A);
    draw(incircle(A,B,C));
    draw(EE--F);

    dot("$A$",A,N);
    dot("$B$",B,SW);
    dot("$C$",C,SE);
    dot("$E$",EE,W);
    dot("$F$",F,E);
\end{asy}
\end{center}
Since $\triangle AEF\sim\triangle ABC$, let $r:=EF/BC=AE/AB=AF/AC$. Then $BE=(1-r)AB$, $CF=(1-r)AC$. Applying Pitot theorem on $BEFC$ gives \[(1-r)(AB+AC)=(1+r)BC.\]
Setting $BC=T$, $r=(T-1)/T$, we have \[AB+AC=(T+1)\cdot\frac{1+\frac T{T+1}}{1-\frac T{T+1}}=(T+1)(2T+1).\]
With $T=13$, the answer is $378$.

