% Input your problem and solution below.
% Three dashes on a newline indicate the breaking points.

---

We have $n\ge2$ lamps $L_1$, \ldots, $L_n$ in a row, each of them being either on or off. Every second we simultaneously modify the state of each lamp as follows: if the lamp $L_i$ and its neighbours (only one neighbour for $i=1$ or $i=n$, two neighbours for other $i$) are in the same state, then $L_i$ is switched off; otherwise, $L_i$ is switched on.

Initially all the lamps are off except the leftmost one which is on.
\begin{enumerate}[label=(\alph*),itemsep=0em]
    \item Prove that there are infinitely many integers $n$ for which all the lamps will eventually be off.
    \item Prove that there are infinitely many integers $n$ for which the lamps will never be all off.
\end{enumerate}

---

The constructions for (a) and (b) are $n=2^k$ and $n=2^k+1$ respectively.

\paragraph{Solution (a)}
We will verify by induction on $k$ the following hypothesis:
\begin{quote}
    For $n=2^k$, $L_n$ will eventually turn on, and the first time it turns on, all the lamps will be on.
\end{quote}
The base case, $n=2$, is easily verified: $10\to11\to00$. Then, if the hypothesis holds for $k-1$, in the case of $n=2^k$ lamps, the state will eventually be
\[\underbrace{11\cdots1}_{2^{k-1}}\underbrace{00\cdots0}_{2^{k-1}}\to\underbrace{0\cdots0}_{2^{k-1}-1}11\underbrace{0\cdots0}_{2^{k-1}-1}.\]
Then the lamps are symmetric around the center, so applying the inductive hypothesis again, the first time $L_n$ is on, all the lamps are on.

To finish, note that
\[\underbrace{11\cdots1}_{2^k}\to\underbrace{00\cdots0}_{2^k},\]
as needed.

\paragraph{Solution (b)}
By the hypothesis in (a), the state of the lamps for $n=2^k+1$ will eventually be
\[\underbrace{11\cdots1}_{2^k}0\to\underbrace{0\cdots0}_{2^k-1}11.\]
However, the state $0\cdots011$ is merely a reflection of a state we've seen earlier: $110\cdots0$, so the states of the lamps are periodic and never all off.


