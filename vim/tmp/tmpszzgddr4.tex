% Input your problem and solution below.
% Three dashes on a newline indicate the breaking points.

---

We call a positive integer alternating if every two consecutive digits in its decimal representation are of different parity.

Find all positive integers $n$ such that $n$ has a multiple which is alternating.

---

The answer is $20\nmid n$; it is straightforward to show no multiple of $20$ is alternating.
\setcounter{lemma}0
\begin{lemma}[Powers of five]
    For each $k$, there is an odd $k$-digit alternating multiple of $5^k$.
\end{lemma}
\begin{proof}
    Induct on $k$, starting with $5$. To construct a $(k+1)$-digit multiple of $5^{k+1}$, there are five choices for the leading digit before appending the $k$-digit multiple of $5^k$; one of these must work.
\end{proof}
\begin{lemma}[Powers of two]
    For even $k$, there is an even $k$-digit alternating multiple of $2^{k+1}$.
\end{lemma}
\begin{proof}
    Repeat the above proof, but add two digits at a time.
\end{proof}
\begin{claim*}[Relatively prime with 10]
    For every $m\perp10$ and $r$, there is an even alternating number and an odd alternating number equivalent to $r$ modulo $m$.
\end{claim*}
\begin{proof}
    First we show odd alternating numbers are surjective modulo $m$. Define \[f(k):=\underbrace{1010\cdots1}_{k\text{ ones}}=1+10^2+\cdots+10^{2(k-1)}=\frac{10^{2k}-1}{99}.\]
    Then take $\vphi(99m)\mid k$ to achieve $r=0$.

    To achieve $r\equiv2j\pmod m$, take \[f\big(j\vphi(99m)\big)+2\cdot\left(10^{\vphi(99m)}+10^{2\vphi(99m)}+\cdots+10^{j\vphi(99m)}\right);\]
    that is, take $f(k)$ where $k$ is a sufficiently large multiple of $\vphi(99m)$, and turn some of the 1's into 3's.

    To show even alternating numbers are surjective modulo $m$, just take each odd alternating number and multiply by 10.
\end{proof}

Finally, we need to settle $n$ not relatively prime with 10: in what follows, $m\perp10$.
\begin{itemize}
    \item An odd multiple of $5^{\vphi(99m)}\cdot m$ is achieved by taking $10^{\vphi(99m)}A+B$, where $A$ is odd and chosen so that the expression is $0\pmod m$, and $B$ is an odd $\vphi(99m)$-digit multiple of $5^{\vphi(99m)}$.
    \item An even multiple of $2^{\vphi(99m)}\cdot m$ is achieved by taking $10^{\vphi(99m)}A+B$, where $A$ is even and chosen so that the expression is $0\pmod m$, and $B$ is an even $\vphi(99m)$-digit multiple of $2^{vphi(99m)}$.
    \item An even multiple of $2\cdot5^{\vphi(99m)+1}\cdot m$ is achieved by taking the construction for an odd multiple of $5^{\vphi(99m)}\cdot m$ and multiplying by $10$.
\end{itemize}
It is easy to see that any $20\nmid n$ has a multiple that is either of the form $5^{\vphi(99m)}\cdot m$, $2^{\vphi(99m)}\cdot m$, or $2\cdot5^{\vphi(99m)+1}\cdot m$, for some $m\perp10$, so we are done.

