% Input your problem and solution below.
% Three dashes on a newline indicate the breaking points.

---

Determine all functions $f:\mathbb N\to\mathbb N$ such that for all positive integers $a$, $b$, there exists a non-degenerate triangle with sides of lengths \[a,\;f(b),\;\text{and}\;f(b+f(a)-1).\]

---

The answer is $f(n)\equiv n$. Let $P(a,b)$ be the assertion. The key insight is that if $1$, $m$, $n$ are the sides of a triangle, then $m=n$.
\setcounter{claim}0
\begin{claim}
    $f(1)=1$.
\end{claim}
\begin{proof}
    By $P(1,b)$, we have $1$, $f(b)$, $f(b+f(1)-1)$ are the sides of a triangle; i.e.\ $f(b)=f(b+f(1)-1)$. If $f(1)\ne1$, then $f$ is periodic and therefore bounded above by some constant $M$. Taking $a>2M$ in $P(a,b)$ gives a contradiction.
\end{proof}
\begin{claim}
    $f$ is an involution; in particular, $f$ is bijective.
\end{claim}
\begin{proof}
    By $P(a,1)$, we have $a$, $1$, $f(f(a))$ are the sides of a triangle; i.e.\ $f(f(a))=a$.
\end{proof}
\begin{claim}
    If $f(2)=k$, then $f(n)\equiv(n-1)k-(n-2)$.
\end{claim}
\begin{proof}
    The proof is by strong induction on $n$ with base cases $n=1,2$ given.

    Suppose $f(n)=(n-1)k-(n-2)$; then $P(2,(n-1)k-(n-2))$ implies $f(nk-(n-1))\le n$. But $k\ne1$ and $f$ is injective, so $f(nk-(n-1))=n$. Thus $f( (n-1)k-(n-2))$ as needed.
\end{proof}

This implies the sequence $1$, $k$, $2k-1$, $3k-2$ is bijective, so $k=2$. Finally $f(n)=2(n-1)-(n-2)=n$ for all $n$, end proof.

