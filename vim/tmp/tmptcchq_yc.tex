% Input your problem and solution below.
% Three dashes on a newline indicate the breaking points.

---

Let $AXYZB$ be a convex pentagon inscribed in a semicircle of diameter $AB$. Denote by $P$, $Q$, $R$, $S$ the feet of the perpendiculars from $Y$ onto lines $AX$, $BX$, $AZ$, $BZ$, respectively. Prove that the acute angle formed by lines $PQ$ and $RS$ is half the size of $\angle XOZ$, where $O$ is the midpoint of segment $AB$.

---

\begin{center}
    \begin{asy}
        size(9cm); defaultpen(fontsize(10pt));
        pair O, A, B, X, Y, Z, P, Q, R, SS, T;
        O=(0, 0);
        A=(-1, 0);
        B=(1, 0);
        X=(Cos(144), Sin(144));
        Y=(Cos(105), Sin(105));
        Z=(Cos(27), Sin(27));
        P=foot(Y, A, X);
        Q=foot(Y, B, X);
        R=foot(Y, A, Z);
        SS=foot(Y, B, Z);
        T=foot(Y, A, B);
        dot(O); dot(A); dot(B); dot(X); dot(Y); dot(Z); dot(P); dot(Q); dot(R); dot(SS); dot(T);
        draw(arc(O, 1, 0, 180));
        draw(circumcircle(T, A, Y), dotted);
        draw(circumcircle(T, B, Y), dotted);
        draw(A -- B);
        draw(Z -- O -- X -- A -- Z -- B -- X);
        draw(A -- Y -- B);
        draw(P -- T -- SS);
        draw(P -- Y -- Q); draw(R -- Y -- SS);
        draw(X -- P); draw(Z -- SS);
        draw(Y -- T);
        draw(rightanglemark(Y, T, B, 1.25));
        draw(rightanglemark(Y, P, A, 1.25));
        draw(rightanglemark(Y, Q, X, 1.25));
        draw(rightanglemark(Y, R, Z, 1.25));
        draw(rightanglemark(Y, SS, B, 1.25));
        draw(rightanglemark(A, X, B, 1.25));
        draw(rightanglemark(A, Y, B, 1.25));
        draw(rightanglemark(A, Z, B, 1.25));
        label("$O$", O, S);
        label("$A$", A, SW);
        label("$B$", B, SE);
        label("$X$", X, (X-B)/length(X-B));
        label("$Y$", Y, Y);
        label("$Z$", Z, (Z-A)/length(Z-A));
        label("$P$", P, (P-T)/length(P-T));
        label("$Q$", Q, SW);
        label("$R$", R, SE);
        label("$S$", SS, (SS-T)/length(SS-T));
        label("$T$", T, S);
    \end{asy}
\end{center}
Let $T$ be the projection of $Y$ onto $\overline{AB}$. Notice that $T$ lies on the Simson Line $\overline{PQ}$ from $Y$ to $\triangle AXB$, and the Simson Line $\overline{RS}$ from $Y$ to $\triangle AZB$. Hence, $T=\overline{PQ}\cap\overline{RS}$, so it suffices to show that $\angle PTS=\tfrac{1}{2}\angle XOZ$.

Since $TAPY$ and $TBSY$ are cyclic quadrilaterals, \[\angle PTS=\angle PTY+\angle YTS=\angle PAY+\angle YBS=\angle XAY+\angle YBZ=\frac{1}{2}\angle XOY+\frac{1}{2}\angle YOZ=\frac{1}{2}\angle XOZ,\]
as required.
