% Input your problem and solution below.
% Three dashes on a newline indicate the breaking points.

---

Let $ABC$ be a triangle with orthocenter $H$, and define $E$ and $F$ as the intersections of $\overline{AH}$ with the perpendicular bisectors of $\overline{AB}$ and $\overline{AC}$ respectively. Furthermore, let $D$ be the intersection of $\overline{BE}$ and $\overline{CF}$. Suppose that $X$ and $Y$ lie on $\overline{AB}$ and $\overline{AC}$ respectively such that $\overline{FX}$, $\overline{EY}$, and $\overline{BC}$ are all parallel. Prove that $X$ and $Y$ lie on the exterior angle bisector of $\angle BDC$.

---

Let $O$ be the circumcenter of $ABC$, and define $P$ and $Q$ as the second intersections of $\overline{BD}$ and $\overline{CD}$ with the circumcircle of $ABC$. It follows that $$\widehat{QP}=2(\angle FCA+\angle EBA)=2(\angle FAC+\angle EAB)=2\angle A=\widehat{BC},$$$BQPC$ must be an isosceles trapezoid with bases parallel to $\overline{AO}$. From here, it follows that $\overline{DO}$ is perpendicular to $\overline{AO}$ and $\overline{DO}$ bisects $\angle EDF$.
\begin{center}
    \begin{asy}
        size(8cm);
        defaultpen(fontsize(10pt));
        pair A=(0,75), B=(-30,0), C=(60,0), D=(-14.1,16.7), E=(0,31.5), F=(0,13.5), X=(-24.6,13.5), Y=(34.8,31.5), O=(15,25.5), P=(38.3,71.7), Q=(-36.6,21.7);
        draw(A--B--C--cycle, linewidth(0.5));
        draw(A--F--X, linewidth(0.5));
        draw(E--Y, linewidth(0.5));
        draw(B--E, linewidth(0.5));
        draw(C--D, linewidth(0.5));
        draw(X--Y, linewidth(0.4)+dashed+grey);
        draw(A--O, linewidth(0.4)+grey);
        draw(E--P, linewidth(0.4)+grey);
        draw(D--Q, linewidth(0.4)+grey);
        draw(circumcircle(A,B,C), linewidth(0.5));
        draw(circumcircle(A,E,O), linewidth(0.7)+dashed);
        draw(B--Q, linewidth(0.4)+grey);
        draw(C--P, linewidth(0.4)+grey);
        dot("$A$", A, (-0.5,1));
        dot("$B$", B, SW);
        dot("$C$", C, SE);
        dot("$X$", X, (-1,0));
        dot("$Y$", Y, (1,0));
        dot("$O$", O, (0,-1));
        dot("$E$", E, (-1,0));
        dot("$F$", F, SW);
        dot("$D$", D, (-0.2,1));
        dot("$P$", P, NE);
        dot("$Q$", Q, (-1,0.5));
    \end{asy}
\end{center}
Now, let $Y'$ be the intersection of $\overline{DO}$ with $\overline{AC}$. Trivial angle chasing yields that $\measuredangle OY'A=\measuredangle ABC=\measuredangle OEA$, so $A$, $E$, $O$, and $Y'$ are concyclic and $\angle AEY'=\angle AOY'=90^\circ$; hence, $Y=Y'$, and $AEOY$ is cyclic. Similarly, $X$ lies on $\overline{DO}$, and $AFOX$ is cyclic. It follows that $\overline{XY}$ is the external angle bisector of $\angle BDC$, as desired.

