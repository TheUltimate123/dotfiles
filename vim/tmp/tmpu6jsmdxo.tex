% Input your problem and solution below.
% Three dashes on a newline indicate the breaking points.

---

Let $ABC$ be a triangle and let $M$, $N$, $P$ be the midpoints of $\seg{BC}$, $\seg{CA}$, $\seg{AB}$. Point $K$ lies on segment $NP$ such that $\seg{AK}$ bisects $\angle BKC$. Lines $MN$ and $BK$ intersect at $E$ and lines $MP$ and $CK$ intersect at $F$. Suppose that $H$ is the foot of the perpendicular from $A$ to $\seg{BC}$ and $L$ is the second intersection of the circumcircles of $\triangle AKH$ and $\triangle HEF$.

Prove that lines $MK$, $EF$, $HL$ are concurrent.

---

\begin{center}
    \begin{asy}
        size(10cm); defaultpen(fontsize(10pt));
        pen pri=red;
        pen sec=orange;
        pen tri=fuchsia;
        pen fil=pri+opacity(0.05);
        pen sfil=sec+opacity(0.05);
        pen tfil=tri+opacity(0.05);

        pair K,B,C,funny,A,M,NN,P,EE,F,T,H,D,SS;
        K=dir(125);
        B=dir(170);
        C=dir(10);
        funny=extension(B,C,K,incenter(K,B,C));
        A=2K-funny;
        M=(B+C)/2;
        NN=(C+A)/2;
        P=(A+B)/2;
        EE=extension(B,K,M,NN);
        F=extension(C,K,M,P);
        T=extension(NN,P,EE,F);
        H=foot(A,B,C);
        D=reflect(T,circumcenter(K,EE,F))*K;
        SS=extension(EE,F,M,K);

        filldraw(circumcircle(A,K,H),tfil,tri);
        draw(T--D--M,tri);
        filldraw(circumcircle(K,EE,F),sfil,sec);
        draw(EE--T,sec);
        draw(A--K,sec+dashed);
        draw(B--F,sec+dashed);
        draw(C--EE,sec+dashed);
        filldraw(circumcircle(B,C,F),fil,pri);
        filldraw(A--B--C--cycle,fil,pri);
        draw(F--M--NN--EE,pri);
        draw(NN--T,pri);

        dot("$A$",A,dir(80));
        dot("$B$",B,SW);
        dot("$C$",C,SE);
        dot("$K$",K,NE);
        dot("$M$",M,S);
        dot("$N$",NN,dir(15));
        dot("$P$",P,SW);
        dot("$E$",EE,NE);
        dot("$F$",F,W);
        dot("$T$",T,W);
        dot("$H$",H,SE);
        dot("$D$",D,NW);
        dot("$S$",SS,dir(75));
    \end{asy}
\end{center}
First by Pappus' theorem on $BKCNMF$, we have $A\in\seg{EF}$, and by Pappus' theorem on $ABCENK$ and $ACBFMK$, we have $\seg{AP}\parallel\seg{BF}\parallel\seg{CE}$. Since $\seg{AK}$ bisects $\angle BKC$ and $\angle EKF$, we have \[\da FBE=\da FBK=\da AKB=\da CKA=\da KCE=\da FCE,\]
whence $BCEF$ is cyclic.

Let $T=\seg{EF}\cap\seg{NP}$ and $S=\seg{MK}\cap\seg{EF}$. Note that \[-1=(BC;M\infty_{BC})\stackrel K=(EF;ST)\stackrel M=(NP;KT).\]
Since $\seg{BC}$ and $\seg{EF}$ are antiparallel wrt.\ $\angle K$, $\seg{KS}$ is the $K$-symmedian of $\triangle KEF$, so $\seg{KT}$ is tangent to $(KEF)$. But $\da TAK=\da FEC=\da ECB=\da AKT$, whence $TA=TK$. Since $A$ and $H$ are reflections across $\seg{NP}$, $T$ is the circumcenter of $\triangle AKH$.

Let $(AKH)$ and $(KEF)$ intersect again at $D$. Since $(AKH)$ and $(KEF)$ are orthogonal, $\seg{TD}$ is tangent to $(KEF)$, so $T$ is the pole of $\seg{KD}$ and $D\in\seg{MKS}$.

The concurrence follows from Radical Axis theorem on $(HEF)$, $(AKH)$, $(KEF)$.

