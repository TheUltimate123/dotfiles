% Input your problem and solution below.
% Three dashes on a newline indicate the breaking points.

---

A hunter and an invisible rabbit play a game in the plane.
The rabbit and hunter start at points $A_0=B_0$. In the $n^\text{th}$ round of the game ($n\ge1$), three things occur in order:
\begin{itemize}
    \item[(i)] The rabbit moves invisibly from $A_{n-1}$ to a point $A_n$ such that $A_{n-1}A_n=1$.
    \item[(ii)] The hunter has a tracking device (e.g. dog) which reports an approximate location $P_n$ of the rabbit, such that $P_nA_n\le1$.
    \item[(iii)] The hunter moves visibly from $B_{n-1}$ to a point $B_n$ such that $B_{n-1}B_n=1$.
\end{itemize}
Let $N=10^9$. Can the hunter guarantee that $A_NB_N<100$?

---

The answer is no. Consider the following procedure, occurring over $m$ moves. For convenience, we will refer to this as an $m$-move.
\begin{itemize}
    \item The rabbit moves from point $A$ to the point $A'$ such that $AA'=m$, and if $P'$ is the projection of $A'$ onto line $AB$, then $A'P'=1$. It moves $1$ unit along segment $AA'$ each round.
    \item During each round, let the dog report the projection of the rabbit's location onto line $AB$. This is always possible since the distance from the rabbit to line $AB$ never exceeds $1$ during the procedure.
    \item Because the hunter doesn't know which half-plane the rabbit is on, he must move along $\seg{AB}$. Otherwise, it is possible that the rabbit is on the other side of the line as he moves, and he moves farther and farther away from the rabbit.
    \item For convenience, assume the hunter is magically able to deduce the exact location of the rabbit at the end of the procedure.
\end{itemize}
\begin{center}
    \begin{asy}
        size(10cm);
        pair B, Bp, A, Ap, P;
        B=(0,0);
        Bp=(2,0);
        A=(5,0);
        Ap=(5,0)+2*dir(20);
        P=foot(Ap,B,A);

        draw(B--P);
        draw(Bp--Ap--P,dashed);
        draw(B--Bp,linewidth(1.5));
        draw(A--Ap,linewidth(1.5));

        dot("$B$",B,S);
        dot("$B'$",Bp,S);
        dot("$A$",A,S);
        dot("$A'$",Ap,NE);
        dot("$P'$",P,SE);
    \end{asy}
\end{center}
\begin{iclaim*}
    Suppose that at some point in time, the rabbit is at point $A$ and the hunter is at point $B$, and furthermore $AB=x$. After a $101$-move, if the rabbit moves to $A'$ and the hunter moves to $B'$, then the rabbit can guarentee that $A'B'\ge\sqrt{x^2+1/101}$.
\end{iclaim*}
\begin{proof}
    Use the same diagram as above. Recall that $A'P'=1$, whence $B'P'=x+\sqrt{m^2-1}-m$. It follows that
    \begin{align*}
        A'B'^2&\ge1+\left(x+\sqrt{m^2-1}-m\right)^2\\
        &=1+x^2+(m^2-1)+m^2+2x\sqrt{m^2-1}-2m\sqrt{m^2-1}-2xm\\
        &=x^2+2m^2-2xm+2x\sqrt{m^2-1}-2m\sqrt{m^2-1}\\
        &=x^2+2(m-x)\left(m-\sqrt{m^2-1}\right).
    \end{align*}
    Now let $m=101$. We choose $m=101$ since this forces the second term to always be positive, thus increasing the distance between the rabbit and the hunter; however the problem is quite weak, so many other (possibly more convenient) $m\ge101$ still work. Remark that \[A'B'^2\ge x^2+\frac{2(m-x)}{m+\sqrt{m^2-1}}\ge x^2+\frac{m-x}m\ge x^2+\frac1{101},\]
    as desired.
\end{proof}

Finally, we can perform a $101$-move a total of $\lfloor10^9/101\rfloor$ times. That is, \[A_NB_N\ge\left\lfloor\frac{10^9}{101}\right\rfloor\cdot\frac1{101}\ge\frac{10^9-100}{101^2}>100^2,\]
so $A_NB_N>100$, as desired.
