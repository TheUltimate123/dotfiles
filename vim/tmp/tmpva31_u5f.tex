% Input your problem and solution below.
% Three dashes on a newline indicate the breaking points.

---

Let $ABC$ be a triangle with sides $AB=5$, $BC=7$, $CA=8$. Denote by $O$ and $I$ the circumcenter and incenter of $\triangle ABC$, respectively. The incircle of $\triangle ABC$ touches $\overline{BC}$ at $D$, and line $OD$ intersects the circumcircle of $\triangle AID$ again at $K$. Then the length of $DK$ can be expressed in the form $\tfrac{m\sqrt n}p$, where $m$, $n$, $p$ are positive integers, $m$ and $p$ are relatively prime, and $n$ is not divisible by the square of any prime. Find $m+n+p$. 

---

\begin{center}
    \begin{asy}
        size(8cm); defaultpen(fontsize(10pt));
        pair A,B,C,I,O,D,EE,F,J;
        A=dir(140);
        B=dir(205);
        C=dir(335);
        I=incenter(A,B,C);
        O=(0,0);
        D=foot(I,B,C);
        EE=2*foot(circumcenter(A,I,D),O,A)-A;
        F=foot(A,B,C);
        J=2*foot(circumcenter(A,I,D),O,D)-D;

        draw(incircle(A,B,C),gray);
        draw(A--I--D--EE,gray);
        draw(circumcircle(A,I,D),dashed);
        draw(EE--O--J,Dotted);
        draw(A--F,dotted);
        draw(A--B--C--A);

        dot("$A$",A,NE);
        dot("$B$",B,SW);
        dot("$C$",C,SE);
        dot("$D$",D,SE);
        dot("$E$",EE,N);
        dot("$F$",F,S);
        dot("$I$",I,SE);
        dot("$K$",J,S);
        dot("$O$",O,SE);
    \end{asy}
\end{center}
Let $\overline{AO}$ intersect $(AIO)$ again at $E$, and let $F$ be the foot from $A$ to $\overline{BC}$. Note that $\overline{AF}$ and $\overline{AO}$ are isogonal wrt.\ $\angle BAC$, but $\overline{AF}\parallel\overline{ID}$. Consequently, \[\measuredangle AID=\measuredangle IAF=\measuredangle OAI=\measuredangle EAI,\]
whence $AIDE$ is an isosceles trapezoid. In particular, $AE=ID$.

When $AB=5$, $BC=7$, $CA=8$, we have $\angle A=60^\circ$, so the area of $\triangle ABC$ is given by $K=10\sqrt3$, the semiperimeter is $s=10$, the inradius is $r=\sqrt3$, and the circumradius is $R=\tfrac7{\sqrt3}$. Plugging in the numbers, \[OD\cdot OK=OA\cdot OE=R(R+r)=\frac{70}3.\]
If $M$ denotes the midpoint of $\overline{BC}$, we can compute $BD=s-b=2$, so $DM=\tfrac32$. Since $\angle BOM=\angle A=60^\circ$ and $BM=\tfrac72$, we have $OM=\tfrac7{2\sqrt3}$. Thus \[OD=\sqrt{\left(\frac32\right)^2+\left(\frac7{2\sqrt3}\right)^2}=\frac{\sqrt{57}}3.\]
It follows that $OK=\tfrac{70}{\sqrt{57}}$ and $DK=\tfrac{17\sqrt{57}}{19}$. The requested sum is $17+57+19=93$.

---

093
