% Input your problem and solution below.
% Three dashes on a newline indicate the breaking points.

---

A crazy physicist discovered a new kind of particle which he called an \emph{imon}, after some of them mysteriously appeared in his lab. Some pairs of imons in the lab can be entangled, and each imon can participate in many entanglement relations. The physicist has found a way to perform the following two kinds of operations with these particles, one operation at a time.
\begin{enumerate}[label=(\roman*),itemsep=0em]
    \item If some imon is entangled with an odd number of other imons in the lab, then the physicist can destroy it.
    \item At any moment, he may double the whole family of imons in the lab by creating a copy $I'$ of each imon $I$. During this procedure, the two copies $I'$ and $J'$ become entangled if and only if the original imons $I$ and $J$ are entangled, and each copy $I'$ becomes entangled with its original imon $I$; no other entanglements occur or disappear at this moment.
\end{itemize}
Prove that the physicist may apply a sequence of much operations resulting in a family of imons, no two of which are entangled.

---

Given a graph $G$ of chromatic number $\chi(G)=k$ colorable by $c_1$, \ldots, $c_k$, I claim that if $k\ge2$, then we can obtain a graph with chromatic number at most $k-1$.
\begin{itemize}
    \item Keep on deleting vertices with odd degree until you can't anymore. Let the resulting graph be $G_1$. The same coloring works, so $\chi(G_1)\le k$.
    \item Apply (ii) to obtain $G_2$. In the first copy of $G_1$, let any vertex colored $c_i$ remain colored $c_i$. In the second copy of $G_1$, let any vertex colored $c_i$ be colored $c_{i+1}$ instead (indices modulo $k$). Then the resulting graph may still be colored in $k$ colors, so $\chi(G_2)\le k$.
    \item Now, delete all vertices of color $c_1$. This is permitted, since the degrees of all vertices of $G_2$ are odd, and by definition, no two adjacent vertices are colored $c_1$. Let the resulting graph be $G_3$. Since the same coloring works, but no vertices of color $c_1$ remain, $\chi(G_3)\le k-1$, as claimed.
\end{itemize}
Repeat this process until the graph has chromatic number 1. Then no edges exist, the end.

