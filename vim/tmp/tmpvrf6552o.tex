% Input your problem and solution below.
% Three dashes on a newline indicate the breaking points.

---

Construct a tetromino by attaching two $2\times1$ dominoes along their longer sides such that the midpoint of the longer side of one domino is a corner of the other domino. This construction yields two kinds of tetrominos with opposite orientations. Let us call them $S$- and $Z$-tetrominos, respectively.

Assume that a lattice polygon $P$ can be tiled with $S$-tetrominos. Prove that no matter how we tile $P$ using only $S$- and $Z$-tetrominos, we always use an even number of $Z$-tetrominos.

---

Consider the following (easily generalizable) coloring of the plane.
\begin{center}
\begin{asy}
    size(5cm); defaultpen(fontsize(10pt));
    draw( (0,0)--(10,0)--(10,10)--(0,10)--cycle);
    for (int i=0; i<10; i += 1) {
        for (int j=0; j<10; j += 1) {
            if ( (i-j) == (i-j)#4 * 4)
                fill( (i,j)--(i+1,j)--(i+1,j+1)--(i,j+1)--cycle,lightred+white+white);
            else fill( (i,j)--(i+1,j)--(i+1,j+1)--(i,j+1)--cycle,lightblue+white+white);
        }
    }
    for (int i=1; i<10; i += 1) {
        draw( (0,i)--(10,i),gray);
        draw( (i,0)--(i,10),gray);
    }
\end{asy}
\end{center}
We can verify that $S$-tetrominos always contain an even number of red squares, whereas $Z$-tetrominos always contain an even number of red squares. Thus, if $P$ can be covered with $S$-tetrominos, it contains an even number of red squares, so any covering contains an even number of $Z$-tetrominos.
\begin{remark}
    It is not very hard to find this covering. The key is to consider each instance of the following:
    \begin{center}
    \begin{asy}
        size(3cm); defaultpen(fontsize(10pt));
        draw( (0,0)--(2,0)--(2,1)--(3,1)--(3,2)--(-1,2)--(-1,1)--(0,1)--cycle);
        draw( (0,2)--(0,1)--(2,1)--(2,2),gray);
        draw( (1,0)--(1,2),gray);
        label("$A$",(-.5,1.5));
        label("$B$",(.5,1.5));
        label("$C$",(1.5,1.5));
        label("$D$",(2.5,1.5));
        label("$E$",(.5,.5));
        label("$F$",(1.5,.5));
    \end{asy}
    \end{center}
    If every $S$-tetromino contains an even number of red squares but every $Z$-tetromino contains an even number of blue squares, then in particular, the number of red squares among ${A,B,E,F}$, ${C,D,E,F}$ differ in parity.

    Thus, for every four consecutive squares $A$, $B$, $C$, $D$, the number of red squares among ${A,B}$ should differ in parity from the number of red squares among ${C,D}$. Then, repeatedly apply this until you obtain a coloring of the plane with the desired property.
\end{remark}

