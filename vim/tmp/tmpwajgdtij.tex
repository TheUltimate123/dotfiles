% Input your problem and solution below.
% Three dashes on a newline indicate the breaking points.

---

Circles $\omega_1$ and $\omega_2$ intersect at points $X$ and $Y$. Line $\ell$ is tangent to $\omega_1$ and $\omega_2$ at $A$ and $B$, respectively, with line $AB$ closer to point $X$ than to $Y$. Circle $\omega$ passes through $A$ and $B$ intersecting $\omega_1$ again at $D \neq A$ and intersecting $\omega_2$ again at $C \neq B$. The three points $C$, $Y$, $D$ are collinear, $XC = 67$, $XY = 47$, and $XD = 37$. Find $AB^2$.

---

(Diagram not in scale for obvious reasons.)
\begin{center}
    \begin{asy}
        size(8cm);
        pair Y, C, D, O1, O2, Q, A, B, P, X;
        Y=(0, 0);
        C=(sqrt(67^2-67/37*47^2), 0);
        D=(-sqrt(37^2-37/67*47^2), 0);
        O1=D/2+(0, length(D/2)/tan(acos(47*52/(37*67))))/5;
        O2=C/2+(0, length(C/2)/tan(acos(47*52/(37*67))))/5;
        Q=extension(O1, O2, C, D);
        A=intersectionpoints(circle((Q+O1)/2, length(Q-O1)/2), circle(O1, length(Y-O1)))[0];
        B=intersectionpoint(A -- (A+(A-Q)*100), circle(O2, length(Y-O2)));
        P=extension(B, C, D, A);
        X=intersectionpoints(circle(O1, length(Y-O1)), circle(O2, length(Y-O2)))[0];

        draw(circle(O1, length(Y-O1)), red);
        draw(circle(O2, length(Y-O2)), red);
        draw(C -- Q -- B, red); draw(circumcircle(A, B, C), orange);
        draw(C -- P -- D, red); draw(Y -- P, red); draw(circumcircle(A, B, Y), orange);
        draw(A -- Y -- B, orange); draw(C -- X -- D, orange);

        dot("$P$", P, N);
        dot("$Q$", Q, SW);
        dot("$A$", A, dir(160));
        dot("$B$", B, N);
        dot("$C$", C, SE);
        dot("$D$", D, SW);
        dot("$X$", X, NE);
        dot("$Y$", Y, S);
    \end{asy}
\end{center}
Let $Q=\overline{AB}\cap\overline{CD}$. Then, by Power of a Point, \[QA^2=QD\cdot QY\text{ and }QB^2=QC\cdot QD,\]
whence \[QA^2\cdot QB^2=QY^2\cdot QC\cdot QD=QY^2\cdot QA\cdot QB,\]
so $QY^2=QA\cdot QB$. It follows that $\overline{CD}$ is tangent to $(ABX)$. Then, in the language of directed angles, \[\measuredangle ADY=\measuredangle ADQ=-\measuredangle YAQ=-\measuredangle YAB\]
and \[\measuredangle DYA=\measuredangle QYA=-\measuredangle ABY,\]
and by symmetry $\triangle DAY\sim\triangle AYB\sim\triangle YBC$. It follows that $\frac{DY}{AB}=\frac{AY}{BY}=\frac{AB}{CY}$, so $AB^2=CY\cdot DY$. However,\[$\measuredangle DXY=\measuredangle DAY=\measuredangle YBC=\measuredangle YXC,\]
so $\overline{XY}$ is the angle bisector of $\triangle XDC$. Since $XC=67,XY=47,XD=37$, \[AB^2=CY\cdot DY=XC\cdot XD-XY^2=37\cdot 67-47^2=\boxed{270},\]
and we are done.
