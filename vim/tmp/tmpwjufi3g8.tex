% Input your problem and solution below.
% Three dashes on a newline indicate the breaking points.
% vim: tw=72
Let $ABC$ be a triangle with incenter $I$ whose incircle is tangent to $\seg{BC}$, $\seg{CA}$, $\seg{AB}$ at $D$, $E$, $F$. Point $P$ lies on $\seg{EF}$ such that $\seg{DP}\perp\seg{EF}$. Ray $BP$ meets $\seg{AC}$ at $Y$ and ray $CP$ meets $\seg{AB}$ at $Z$. Point $Q$ is selected on the circumcircle of $\triangle AYZ$ so that $\seg{AQ}\perp\seg{BC}$. Prove that $P$, $I$, $Q$ are collinear.
---

\begin{center}
\begin{asy}
    size(10cm);
    defaultpen(fontsize(9pt));
    pair A=dir(130);
    pair B=dir(210);
    pair C=dir(330);
    pair I=incenter(A,B,C);
    pair O=(0,0);
    pair D=foot(I,B,C);
    pair EE=foot(I,C,A);
    pair F=foot(I,A,B);
    pair P=foot(D,EE,F);
    pair T=2*foot(I,D,P)-D;
    pair Y=extension(B,P,A,C);
    pair Z=extension(C,P,A,B);
    pair SS=extension(B,C,Y,Z);
    pair J=extension(P,I,A,SS);
    pair X=foot(A,B,C);
    pair Q=extension(A,X,P,I);

    draw(circle(O,1));
    draw(incircle(A,B,C),gray);
    draw(circumcircle(A,Y,Z));
    draw(A--B--C--A);
    draw(EE--F);
    draw(EE--D--F,gray);
    draw(B--SS--Y);
    draw(A--SS,gray);
    draw(B--Y,gray);
    draw(C--Z,gray);
    draw(D--T);
    draw(J--I);
    draw(A--X,gray);

    dot("$A$",A,N);
    dot("$B$",B,SW);
    dot("$C$",C,SE);
    dot("$D$",D,S);
    dot("$E$",EE,NE);
    dot("$F$",F,W);
    dot("$I$",I,S);
    dot("$P$",P,dir(245));
    dot("$T$",T,dir(70));
    dot("$Y$",Y,NE);
    dot("$Z$",Z,2*dir(190));
    dot("$S$",SS,SW);
    dot("$J$",J,dir(160));
    dot("$X$",X,S);
    dot("$Q$",Q,SW);
\end{asy}
\end{center}
Let $T$ be the second intersection of $\seg{DP}$ and the incircle, and let the tangent at $T$ intersect $\seg{AC}$ at $Y'$ and $\seg{AB}$ at $Z'$. Note that $BCY'Z'$ is tangential by definition, and also 
\begin{align*}
\angle B+\angle CY'Z'&=(180^\circ-\angle DIF)+(180^\circ-\angle TIE)\\
&=360^\circ-2\angle DPF=180^\circ,
\end{align*}
thus $BCY'Z'$ is cyclic. By Brianchon's Theorem on $BDCY'TZ'$ and $BCEY'Z'F$, lines $DT$, $EF$, $BY'$, $CZ'$ concur, so $Y'=Y$ and $Z'=Z$.

Let $J$ be the Miquel point of $BCYZ$, and $\ell$ the line through $J$ perpendicular to $\seg{AJ}$. By Brokard's Theorem on $BCYZ$, $\ell=\seg{PI}$, so it suffices to show that $Q$ lies on $\ell$. But this is obvious: Since $\seg{YZ}$ and $\seg{BC}$ are antiparallel wrt.\ $\angle A$, and the orthocenter and circumcenter are isogonal conjugates, $\seg{AQ}$ is a diameter of $(AYZ)$, so $\angle AJQ=90^\circ$, completing the proof.

