% Input your problem and solution below.
% Three dashes on a newline indicate the breaking points.

---

Let $\mathcal E$ be an ellipse and let $\mathcal E_1$, $\mathcal E_2$, $\mathcal E_3$ be ellipses inside $\mathcal E$ such that $\mathcal E_n$ is tangent to $\mathcal E$ at points $S_n$ and $T_n$ for $n=1,2,3$. Assume that $S_1$, $S_2$, $S_3$, $T_1$, $T_2$, $T_3$ lie on $\mathcal E$ in that order. For $n=1,2,3$, let the four points in $\mathcal E_{n-1}\cap\mathcal E_{n+1}$ (where indices are taken modulo $3$) form convex quadrilateral $A_nB_nC_nD_n$. Show that $\overline{A_1C_1}\cap\overline{A_2C_2}\in\overline{A_3C_3}\cup\overline{B_3D_3}$.

---

\begin{center}
    \begin{asy}
        size(9cm); defaultpen(fontsize(10pt));

        // tangency points
        pair S1 = dir(-30);
        pair S2 = dir(40);
        pair S3 = dir(115);
        pair T1 = dir(160);
        pair T2 = dir(240);
        pair T3 = dir(260);

        //pair S1 = dir(65);
        //pair S2 = dir(100);
        //pair S3 = dir(205);
        //pair T1 = dir(220);
        //pair T2 = dir(340);
        //pair T3 = dir(350);

        // distance from foci of E_n to center of E
        real d1 = 0.985;
        real d2 = 0.970;
        real d3 = 0.920;

        //real d1 = 0.990;
        //real d2 = 0.968;
        //real d3 = 0.920;

        // final affine transformation
        transform t = yscale(1.2);

        // labels A_n correctly
        int n1 = 3;
        int n2 = 1;
        int n3 = 1;

        // returns an ellipse given the two foci and a point on the ellipse
        guide ellipse(pair F1, pair F2, pair P) {
            real c = 1/2 * abs(F2 - F1);
            real a = 1/2 * (abs(P - F1) + abs(P - F2));
            real b = sqrt(a^2 - c^2);
            return shift((F1 + F2) / 2) * rotate(degrees(F2 - F1)) * scale(a, b) * circle((0, 0), 1);
        }

        // returns an ellipse tangent to the unit circle given the two tangency points and the distance to the foci. This constructs the ellipses based on the fact that S_n, T_n, O, and the foci are concyclic.
        guide tangentellipse(pair S, pair T, real d) {
            path g = circle((0, 0), d);
            path h = circumcircle(S, T, (0, 0));
            pair f[] = intersectionpoints(g, h);
            return ellipse(f[0], f[1], S);
        }

        // determines the affine transformation that maps three points to three other points
        transform mapthreepoints(pair A, pair B, pair C, pair Ap, pair Bp, pair Cp) {
            real[][] a = {
                {1, 0, A.x, A.y, 0, 0},
                {0, 1, 0, 0, A.x, A.y},
                {1, 0, B.x, B.y, 0, 0},
                {0, 1, 0, 0, B.x, B.y},
                {1, 0, C.x, C.y, 0, 0},
            {0, 1, 0, 0, C.x, C.y}};

            real[] b = {Ap.x, Ap.y, Bp.x, Bp.y, Cp.x, Cp.y};

            real[] x = solve(a, b);

            return (x[0], x[1], x[2], x[3], x[4], x[5]);
        }

        guide aexcircle(pair A, pair B, pair C) {
            pair I = incenter(A, B, C);
            pair M = extension(A, I, circumcenter(A, B, C), (B + C) / 2);
            pair IA = 2 * M - I;
            real a = abs(B - C);
            real b = abs(C - A);
            real c = abs(A - B);
            real s = 1/2 * (a + b + c);
            return circle(IA, sqrt(s * (s - b) * (s - c) / (s - a)));
        }

        // determines the point A' where A'P', BP', CP' concur at the Gergonne point
        transform inellipsetransform(pair A, pair B, pair C, pair P) {
            pair D = extension(A, P, B, C);
            pair E = extension(B, P, C, A);
            pair F = extension(C, P, A, B);

            if (intersectionpoints(A -- P, B -- C).length == 1) {
                path c1 = circle(B, abs(F - B) / abs(D - B) * abs(C - B));
                path c2 = circle(A, abs(F - A) / abs(E - A) * abs(C - A));

                pair Cp = intersectionpoints(c1, c2)[1];

                return mapthreepoints(A, B, Cp, A, B, C);
            }

            if (intersectionpoints(B -- P, C -- A).length == 1 || intersectionpoints(C -- P, A -- B).length == 1) {
                return inellipsetransform(B, C, A, P);
            }

            path c1 = circle(B, abs(D - B) / abs(F - B) * abs(B - A));
            path c2 = circle(C, abs(D - B) / abs(F - B) * abs(F - A) + abs(D - C));

            pair Ap = intersectionpoints(c1, c2)[0];

            return mapthreepoints(Ap, B, C, A, B, C);
        }

        // returns an ellipse tangent to the feet of three cevians that concur at P
        guide inellipse(pair A, pair B, pair C, pair P) {
            pair D = extension(A, P, B, C);
            pair E = extension(B, P, C, A);
            pair F = extension(C, P, A, B);

            if (intersectionpoints(A -- P, B -- C).length == 1) {
                path c1 = circle(B, abs(F - B) / abs(D - B) * abs(C - B));
                path c2 = circle(A, abs(F - A) / abs(E - A) * abs(C - A));

                pair Cp = intersectionpoints(c1, c2)[1];

                return mapthreepoints(A, B, Cp, A, B, C) * aexcircle(A, B, Cp);
            }

            if (intersectionpoints(B -- P, C -- A).length == 1 || intersectionpoints(C -- P, A -- B).length == 1) {
                return inellipse(B, C, A, P);
            }

            path c1 = circle(B, abs(D - B) / abs(F - B) * abs(B - A));
            path c2 = circle(C, abs(D - B) / abs(F - B) * abs(F - A) + abs(D - C));

            pair Ap = intersectionpoints(c1, c2)[0];

            return mapthreepoints(Ap, B, C, A, B, C) * incircle(Ap, B, C);
        }

        // returns an ellipse arc given the two foci and two points on its perimeter
        guide ellipsearc(pair F1, pair F2, pair P, pair Q) {
            real c = 1/2 * abs(F2 - F1);
            real a = 1/2 * (abs(P - F1) + abs(P - F2));
            real b = sqrt(a^2 - c^2);

            transform t = shift((F1 + F2) / 2) * rotate(degrees(F2 - F1)) * scale(a, b);

            transform tp = inverse(t);

            pair Pp = tp * P;
            pair Qp = tp * Q;

            real dPp = degrees(Pp);
            real dQp = degrees(Qp);

            if (dPp < dQp) {
                return t * arc((0, 0), 1, dPp, dQp);
            }
            return t * arc((0, 0), 1, dPp - 360, dQp);

        }

        //returns an arc of an ellipse tangent to two points on the circle
        guide tangentellipsearc(pair S, pair T, real d) {
            path g = circle((0, 0), d);
            path h = circumcircle(S, T, (0, 0));
            pair f[] = intersectionpoints(g, h);
            return ellipsearc(f[0], f[1], S, T);
        }

        path E = circle((0, 0), 1);

        path E1 = tangentellipse(S1, T1, d1);
        path E2 = tangentellipse(S2, T2, d2);
        path E3 = tangentellipse(S3, T3, d3);

        path E1a = tangentellipsearc(S1, T1, d1);
        path E1b = tangentellipsearc(T1, S1, d1);

        path E2a = tangentellipsearc(S2, T2, d2);
        path E2b = tangentellipsearc(T2, S2, d2);

        path E3a = tangentellipsearc(S3, T3, d3);
        path E3b = tangentellipsearc(T3, S3, d3);

        pair u1[] = intersectionpoints(E2, E3);
        pair A1 = u1[(n1-0) % 4];
        pair B1 = u1[(n1-1) % 4];
        pair C1 = u1[(n1-2) % 4];
        pair D1 = u1[(n1-3) % 4];

        pair u2[] = intersectionpoints(E3, E1);
        pair A2 = u2[(n2-0) % 4];
        pair B2 = u2[(n2-1) % 4];
        pair C2 = u2[(n2-2) % 4];
        pair D2 = u2[(n2-3) % 4];

        pair u3[] = intersectionpoints(E1, E2);
        pair A3 = u3[(n3-0) % 4];
        pair B3 = u3[(n3-1) % 4];
        pair C3 = u3[(n3-2) % 4];
        pair D3 = u3[(n3-3) % 4];

        pair X1 = extension(A1, C1, B1, D1);
        pair X2 = extension(A2, C2, B2, D2);
        pair X3 = extension(A3, C3, B3, D3);

        pair P = extension(B1, D1, B2, D2);
        pair K1 = extension(A2, C2, A3, C3);
        pair K2 = extension(A3, C3, A1, C1);
        pair K3 = extension(A1, C1, A2, C2);

        path IE = inellipse(K1, K2, K3, P);

        draw(t * (E ^^ E1 ^^ E2 ^^ E3));
        draw(t * (A1 -- C1 ^^ A2 -- C2));
        draw(t * (A3 -- C3 ^^ B3 -- D3), dotted);
        draw(t * (C1 -- K3 -- A2));
        draw(t * (B3 -- K3), dotted);

        dot("$S_1$", t * S1, dir(S1));
        dot("$S_2$", t * S2, dir(S2));
        dot("$S_3$", t * S3, dir(S3));
        dot("$T_1$", t * T1, dir(T1));
        dot("$T_2$", t * T2, dir(T2));
        dot("$T_3$", t * T3, dir(T3));

        dot("$A_1$", t * A1, 1.5 * dir(345));
        dot("$B_1$", t * B1, dir(345));
        dot("$C_1$", t * C1, dir(315));
        dot("$D_1$", t * D1, dir(345));

        dot("$A_2$", t * A2, dir(45));
        dot("$B_2$", t * B2, dir(210));
        dot("$C_2$", t * C2, dir(210));
        dot("$D_2$", t * D2, dir(200));

        dot("$A_3$", t * A3, 1.5 * dir(110));
        dot("$B_3$", t * B3, 1.5 * dir(285));
        dot("$C_3$", t * C3, 1.2 * dir(10));
        dot("$D_3$", t * D3, 1.2 * dir(285));
    \end{asy}
\end{center}

\paragraph{First solution (Jaedon Whyte)} Suppose $\mathcal E$ is the level set $e(x,y)=0$, where $e$ is some convex quadratic polynomial in $x$, $y$. Since $\mathcal E_1$, $\mathcal E_2$, $\mathcal E_3$ are tangent to $\mathcal E$, we can represent them as the level sets
\begin{align*}
    \mathcal E_1&:e(x,y)+a(x,y)^2&=0\\
    \mathcal E_2&:e(x,y)+b(x,y)^2&=0\\
    \mathcal E_3&:e(x,y)+c(x,y)^2&=0\\
\end{align*}<++>
