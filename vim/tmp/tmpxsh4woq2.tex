% Input your problem and solution below.
% Three dashes on a newline indicate the breaking points.
% vim: tw=72

---

The Bank of Bath issues coins with an $H$ on one side and a $T$ on the other. Harry has $n$ of these coins arranged in a line from left to right. He repeatedly performs the following operation: if there are exactly $k>0$ coins showing $H$, then he turns over the $k^\text{th}$ coin from the left; otherwise, all coins show $T$ and he stops. For example, if $n=3$ the process starting with the configuration $THT$ would be $THT\to HHT\to HTT\to TTT$, which stops after three operations.
\begin{itemize}
    \item[(a)] Show that, for each initial configuration, Harry stops after a finite number of operations.
        \vspace{-0.5em}
    \item[(b)] For each initial configuration $C$, let $L(C)$ be the number of operations before Harry stops. For example, $L(THT)=3$ and $L(TTT)=0$. Determine the average value of $L(C)$ over all $2^n$ possible initial configurations $C$.
\end{itemize}

---

The answer is $\tfrac14n(n+1)$. Denote by $a_n$ the answer. Consider a graph $G_n$ of degree $2^n$, where each node represents one of the possible coin arrangements. For each node, draw a directed edge to the state that must immediately follow.

We will show that (i) this graph is a tree, and (ii) $a_n=\tfrac14n(n+1)$; this obviously solves the problem. To this end, we use induction. Remark that the base case, $n=1$, is trivial; in fact the even easier case of $n=0$ is a valid base case as well. $G_n$ is clearly unique, so we construct $G_n$ in terms of $G_{n-1}$.

Say that the root of $G_n$ is $\text{TTT}\ldots\text{TT}$, and that for any node $M$, $\seg M$ denotes the node that results from flipping over every coin, $M^r$ the node the results from reversing the ordering of the coins, and $M_1M_2$ the node that results from concatenating nodes $M_1$ and $M_2$. Also define these operators on graphs, and apply them to each node of the graph. To construct $G_n$, I claim that the following procedure works:
\begin{itemize}
    \item Construct graphs $A=G_{n-1}T$ and $B=\seg{G_{n-1}^r}H$.
        \vspace{-0.5em}
    \item Draw an edge from the root of $B$ (i.e. $\text{HHH}\ldots\text{HH}$) to $\text{HHH}\ldots\text{HT}$.
\end{itemize}
This is not hard to prove. For each node $M$ in $B$, applying the operation to $M$ will flip the coins in position $1+(n-1-k)=n-k$. Since the corresponding node in $A$ has its first $n-1$ coins reversed, this operation is identical, so $B$ is a valid subgraph. Finally, applying the operation to $\text{HHH}\ldots\text{HH}$ obviously yields $\text{HHH}\ldots\text{HT}$, so the claim has been proven.

Obviously $G_n$ is a tree, and furthermore note that $\text{HHH}\ldots\text{HH}$ is at a distance of $n$ from the root; this is because from there, we simply sweep from the right to the left, flipping each bit. Thus it takes $n$ operations to reach the end of the process. Half of the tree retains its value of $L(C)$, and the other half has $L(C)$ incremented my $n$. It follows that \[a_n=a_{n-1}+\frac12n=\frac12(1+2+\cdots+n)=\frac14n(n+1).\]
This completes the proof.

