% Input your problem and solution below.
% Three dashes on a newline indicate the breaking points.

---

Let $f:\mathbb R\to\mathbb R$ be a bijective function. Does there always exist an infinite number of functions $g:\mathbb R\to\mathbb R$ such that $f(g(x))=g(f(x))$?

---

The answer is yes, infinitely many $g$ always exist. Note that if $f$ doesn't have finite order, then $f$ commutes with $f^0$, $f^1$, $f^2$, $\ldots$, which are all distinct. Henceforth $f^n\equiv\opname{id}$ for some $n$.

If there are infinitely many fixed points, swap any two of them and fix everything else. In what follows, we will assume $f$ has finite order and fixes finitely many points.

\paragraph{First, official solution} Draw an arrow $a\to b$ whenever $f(a)=b$. This forms a graph comprised of disjoint cycles. Each cycle has length $d\mid n$, so for some $d$ there are infinitely many cycles with length $d$.

Select a countably infinite number of the cycles and label them $x_{r,1}\to x_{r,2}\to\cdots\to x_{r,d}\to x_{r,1}$. For each integer $s$, consider the function $g$ sending $x_{r,a}\to x_{r+s,a}$ for each $r$, $a$, and fixing every other point not of the form $x_{k,\ell}$ for some $k$, $\ell$.

There are infinitely many such $g$, and each works, the end.

\paragraph{Second solution} Again draw arrows $a\to b$ whenever $f(a)=b$, so that we have a graph comprised of disjoint cycles. By assumption there are infinitely many cycles of length greater than $1$.

For each cycle $x_1\to x_2\to\cdots\to x_d\to x_1$ with $d>1$, let $g(x_i)=f(x_i)$ for each $i$, and let $g$ fix everything else. There are infinitely many ways to choose this cycle, and each such $g$ works, end proof.

