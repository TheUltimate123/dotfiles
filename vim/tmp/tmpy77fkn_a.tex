% Input your problem and solution below.
% Three dashes on a newline indicate the breaking points.

---

For a positive integer $a$, define a sequence of integers $x_1$, $x_2$, $\ldots$ by letting $x_1=a$ and $x_{n+1}=2x_n+1$ for $n\ge1$. Let $y_n=2^{x_n}-1$. Determine the largest possible $k$ such that for some positive integer $a$, the numbers $y_1$, $\ldots$, $y_k$ are all prime.

---

The answer is $2$, achieved by $x_1=2$ (thus $y_1=3$, $y_2=31$, but $y_3=2047$, which is composite). Suppose that $y_1$, $y_2$, $y_3$ are all prime, and $x_1>2$. Then $x_1$, $x_2$, $x_3$ must be prime (in particular, $a=2b+1$ for some $b$), as otherwise $y_1$, $y_2$, $y_3$ can be factored.

The key claim is this, which suffices.
\begin{iclaim*}
    $x_3\mid y_2$.
\end{iclaim*}
Let $x=x_3$ be $7\pmod8$ since $x_3=8b+7$. Then \[\left(\frac2x\right)=(-1)^{\frac18(x^2-1)}=1,\]
so $2$ is a quadratic residue and $y_2+1\equiv2^{(x-1)/2}\equiv1\pmod x$.

