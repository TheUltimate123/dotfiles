% Input your problem and solution below.
% Three dashes on a newline indicate the breaking points.

---

Prove for all $k>1$ equation $(x+1)(x+2)\cdots(x+k)=y^2$ has finite solutions.

---

We cite the following well-known lemma:
\begin{lemma*}
    If $P(X)\in\mathbb Z[X]$ has even degree and its leading coefficient is a perfect square, then either $P(X)$ is a square finitely often or $P$ is the square of a polynomial.
\end{lemma*}
\begin{proof}
    Bound between two squares.
\end{proof}

The lemma immediately solves the problem for $k$ even, so henceforth $k$ is odd. For each $i\in[k]$, let
\[x+i=a_i\cdot b_i^2,\quad\text{where }a_i\text{ is squarefree.}\]
Observe that each prime $p\ge k$ can divide at most one of $x+1$, \ldots, $x+k$, so it must have even exponent. It follows that all primes $p$ dividing $a_i$ satisfy $p<k$.

I claim:
\begin{claim*}
    For sufficiently large $x$, the product $\prod_{i\in S}a_i$ is distinct over all $S\subseteq[k]$.
\end{claim*}
\begin{proof}
    Assume for contradiction $\prod a_S=\prod a_T$. We can assume $S\cap T=\varnothing$, since we may delete elements in $S\cap T$ from both products.

    Then
    \[\prod_{i\in S\cup T}(x+i)=a_i^2\cdot b_i^2\cdot b_j^2\]
    is a perfect square. Moreover,
    \[\prod_{i\in[k]\setminus(S\cup T)}(x+i)\]
    is also a perfect square.

    Since $k$ is odd, one of these has an even number of linear terms, so by the lemma, it is not a square for sufficiently large $x$.
\end{proof}

Take $x$ sufficiently large (so the claim holds). There are $2^{\pi(k-1)}$ possible products $\prod a_S$ using primes $p<k$, but we have exhibited $2^k$ distinct products. This is a contradiction.
