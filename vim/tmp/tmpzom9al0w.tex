% Input your problem and solution below.
% Three dashes on a newline indicate the breaking points.

---

In the diagram below, the circle with center $A$ is congruent to and tangent to the circle with center $B$. A third circle is tangent to the circle with center $A$ at point $C$ and passes through point $B$. Points $C$, $A$, $B$ are collinear. The line segment $\overline{CDEFG}$ intersects the circles at the indicated points. Suppose that $DE=6$ and $FG=9$. Find $AG$.
\begin{center}
    \begin{asy}
        size(5cm);
        defaultpen(fontsize(9pt));
        pair A = (-9 sqrt(3), 0);
        pair B = (9 sqrt(3), 0);
        pair C = (-18 sqrt(3), 0);
        pair D = (-4 sqrt(3) / 3, 10 sqrt(6) / 3);
        pair E = (2 sqrt(3), 4 sqrt(6));
        pair F = (7 sqrt(3), 5 sqrt(6));
        pair G = (12 sqrt(3), 6 sqrt(6));
        real r = 9sqrt(3);
        draw(circle(A, r),linewidth(0.5));
        draw(circle(B, r),linewidth(0.5));
        draw(circle((B + C) / 2, 3r / 2),linewidth(0.5));
        draw(C -- D,linewidth(0.5));
        draw("$6$", E -- D,linewidth(0.5));
        draw(E -- F,linewidth(0.5));
        draw("$9$", F -- G,linewidth(0.5));
        dot(A);
        dot(B);
        label("$A$", A, plain.E);
        label("$B$", B, plain.E);
        label("$C$", C, W);
        label("$D$", D, dir(250));
        label("$E$", E, dir(300));
        label("$F$", F, SSW);
        label("$G$", G, N);
    \end{asy}
\end{center}

---

Let's evaluate everything on $\seg{CDEFG}$ first. Let $\omega_1$ be the circle centered at $A$, $\omega_2$ the circle centered at $B$, and $\omega$ the big circle. Since $\seg{BC}$ is a diameter of $\omega$, $\angle BFC=90^\circ$. Then this tells us that $F$ is the midpoint of $\seg{EG}$, so $EG=18$ and $DG=24$.

Let $O$ be the center of $\omega$. Note that $CO=BC/2=3/2\cdot AC$. But by homothety $\triangle CAD\sim\triangle COF$, so $CD/CF=CA/CO=2/3$. Since $CF=CD+15$, it can be seen that $CD=30$.

Let $M$ be the midpoint of $\seg{CD}$, so that $CM=15$, and denote by $r$ the radius of $\omega_1$ and $\omega_2$. Let $AM=x$, so that $BF=3x$. By the Pythagorean Theorem on $\triangle CAM$, $r^2=225+x^2$, but by the Pythagorean Theorem on $\triangle EBF$, $r^2=81+9x^2$. Thus $r^2=243$.

Finally note that $AG^2=\pow(G,\omega_1)+r^2=GC\cdot GD+r^2=24\cdot54+243=1539$. It follows that $AG=9\sqrt{19}$, and we are done.

---

$9\sqrt{19}$
